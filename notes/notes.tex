\documentclass[10pt]{article}

% Packages
% ---------- Basic Packages ----------
\usepackage[utf8]{inputenc}
\usepackage{array}
\usepackage{geometry}
\usepackage{setspace}
\usepackage{multicol}
\usepackage{xcolor}
\usepackage{graphicx}
\usepackage{float}
\usepackage{caption}
\usepackage{booktabs}
\usepackage{url}
\usepackage{hyperref}
\usepackage{lipsum}
\usepackage{fancybox}
\usepackage{todonotes}
\usepackage[dvipsnames]{xcolor}
\usepackage{mdframed} % boxed theorems
\usepackage{thmtools} % couples well with mdframed
\usepackage{tikz} \usetikzlibrary{arrows.meta, calc, decorations.pathmorphing, positioning}
\usepackage{listings}

% --------- Code Listings ----------
\captionsetup[lstlisting]{font={small}, labelfont={bf}, labelsep=colon}
\renewcommand{\lstlistingname}{Code}

% ---------- Color Commands ----------
\DeclareRobustCommand{\yellow}[1]{\begingroup\color{Yellow}\ignorespaces#1\endgroup\ignorespacesafterend}
\DeclareRobustCommand{\red}[1]{\begingroup\color{Red}\ignorespaces#1\endgroup\ignorespacesafterend}
\DeclareRobustCommand{\blue}[1]{\begingroup\color{Blue}\ignorespaces#1\endgroup\ignorespacesafterend}
\DeclareRobustCommand{\green}[1]{\begingroup\color{Green}\ignorespaces#1\endgroup\ignorespacesafterend}
\DeclareRobustCommand{\gray}[1]{\begingroup\color{Gray}\ignorespaces#1\endgroup\ignorespacesafterend}
\newenvironment{redblock}{\begingroup\color{Red}\ignorespaces}{\endgroup\ignorespacesafterend}


% ---------- Math and Theorem Environments ----------
\usepackage{algorithm}
\usepackage{algpseudocode}
\usepackage{amsmath, amssymb, amsthm, amsfonts, mathtools, bm, amsopn}
\usepackage{witharrows}
\usepackage[dvipsnames]{xcolor}
\usepackage{mdframed}       % boxed theorems
\usepackage{thmtools}       % couples well with mdframed
\usepackage{capt-of}        % if you prefer \captionof

% ================== THEOREM ENVIRONMENTS ==================
\numberwithin{equation}{subsection} % optional

% Main shared counter “theorem” per section:
\newtheorem{theorem}{Theorem}[subsection]

% All others share the same counter as “theorem”:
\newtheorem{lemma}[theorem]{Lemma}
\newtheorem{proposition}[theorem]{Proposition}
\newtheorem{corollary}[theorem]{Corollary}

\theoremstyle{definition}
\newtheorem{definition}[theorem]{Definition}
\newtheorem{example}[theorem]{Example}
\newtheorem{prop}[theorem]{Proposition}

\theoremstyle{remark}
\newtheorem{remark}[theorem]{Remark}

% Boxed Assumption using the custom style, sharing the same counter:
\declaretheorem[
  name=Assumption,
  style=boxed,
  sibling=theorem
]{assumption}
% % Define a theorem style that uses mdframed for the box
% \declaretheoremstyle[
%   headfont=\normalfont\bfseries,
%   mdframed={
%     linewidth=0.5pt,
%     roundcorner=4pt,
%     backgroundcolor={gray!5},
%     innertopmargin=6pt,
%     innerbottommargin=6pt,
%     innerleftmargin=8pt,
%     innerrightmargin=8pt
%   }
% ]{boxed}
% % Create the “assumption” environment, numbered within each section
% \declaretheorem[name=Assumption, style=boxed, numberwithin=section]{assumption}
% % Other logical environments
% \newenvironment{solution}
%   {\renewcommand\qedsymbol{$\blacksquare$}\begin{proof}[\textbf{Solution}]}
%   {\end{proof}}
% \newtheorem{lemma}{Lemma}
%   \renewcommand{\thelemma}{\Alph{lemma}}
% \newtheorem*{lem}{Lemma}
% \newtheorem{corollary}[lemma]{Corollary}
% \newtheorem*{theorem}{theorem}
% \theoremstyle{definition}
% \newtheorem{definition}{Definition}
% \theoremstyle{remark}
% \newtheorem*{remark}{Remark}
% \theoremstyle{claim}
% \newtheorem*{claim}{Claim}

% ---------- Page setup ----------
\geometry{letterpaper, margin=1in}
\setlength{\parindent}{0pt}
\linespread{1.5}
\renewcommand{\arraystretch}{1.25} % Adjust as needed

% ---------- Shaded box's to tag ideas ----------
\newcommand{\td}[1]{\todo[inline, color=red!20, size=\small]{#1}}
\newcommand{\note}[1]{\todo[inline, color=blue!20, size=\small]{#1}}
\newcommand{\remeber}[1]{\todo[inline, color=green!20, size=\small]{#1}}

% ---------- Sets and number systems ----------
\newcommand{\Z}{\mathbb{Z}}
\newcommand{\Q}{\mathbb{Q}}
\newcommand{\C}{\mathbb{C}}
\newcommand{\R}{\mathbb{R}}
\newcommand{\N}{\mathbb{N}}
\newcommand{\La}{\mathcal{L}}

% ---------- Summations ----------
\newcommand{\sumforall}{\sum_{\forall i}}
\newcommand{\sumfor}[1]{\sum_{\forall #1}}

% ---------- Generic Domains ----------
\newcommand{\Omg}{\Omega}
\newcommand{\OmgBar}{\overline{\Omega}}
\newcommand{\Omgt}{\Omega(t)}
\newcommand{\pOmg}{\partial \Omega}
\newcommand{\pOmgt}{\partial \Omega(t)}


%---------- Blood Domain ----------
\newcommand{\OmgB}{\Omega_B}
\newcommand{\OmgBt}{\Omega_B(t)}
\newcommand{\OmgBtBar}{\overline{\Omega}_B(t)}
\newcommand{\OmgBBar}{\overline{\Omega}_B}
\newcommand{\pOmgB}{\partial \Omega_B}
\newcommand{\pOmgBt}{\partial \Omega_B(t)}

%---------- Vessel Wall Domain ----------
\newcommand{\OmgW}{\Omega_W}
\newcommand{\OmgWt}{\Omega_W(t)}
\newcommand{\pOmgW}{\partial \Omega_W}

% ---------- Symbols ----------
\newcommand{\e}{\epsilon}
\newcommand{\p}{\rho}
\newcommand{\eps}{\epsilon}
\newcommand{\la}{\lambda}

% ---------- Linear Algebra ----------
\newcommand{\matr}[1]{\begin{pmatrix}#1\end{pmatrix}} % For matrices
\newcommand{\vect}[1]{\begin{bmatrix}#1\end{bmatrix}} % For column vectors
\newcommand{\vct}[1]{\bm{#1}} % For bold vectors
\newcommand{\transp}{^{\mathsf{T}}} % Transpose symbol
\newcommand{\sym}{\text{sym}} % Symmetric part of a matrix
\DeclarePairedDelimiter\ip{\langle}{\rangle} % Inner product
\DeclarePairedDelimiter\norm{\lVert}{\rVert} % Norm
\newcommand{\zero}{\mathbf{0}} % Zero vector/matrix
\newcommand{\ident}{\mathbf{I}} % Identity matrix
\newcommand{\id}[1]{\operatorname{id}_{#1}} % Identity map on set #1
\DeclareMathOperator{\tr}{tr} % Trace
\DeclareMathOperator{\detm}{det} % Determinant
\DeclareMathOperator{\rank}{rank} % Rank
\DeclareMathOperator{\nullity}{nullity} % Nullity

% Block Matrices and Vectors ----------
\newcommand{\blockmat}[2]{\begin{pmatrix} #1 & #2 \end{pmatrix}} % Block matrix with 2 columns
\newcommand{\augmatr}[2]{\left(\begin{array}{#1|#1}#2\end{array}\right)}
\newcommand{\bmfour}[4]{\left( \begin{array}{c|c} #1 & #2 \\ \hline #3 & #4 \end{array} \right)} % 2x2 block matrix with lines
\newcommand{\blockvect}[2]{\begin{bmatrix} #1 \\ #2 \end{bmatrix}} % Block vector with 2 rows

% Unit vectors ----------
\newcommand{\nhat}{\hat{\vct{n}}}
\newcommand{\that}{\hat{\vct{t}}}
\newcommand{\ihat}{\hat{\vct{i}}}
\newcommand{\jhat}{\hat{\vct{j}}}
\newcommand{\khat}{\hat{\vct{k}}}

% Generic vectors and tensors. (bold notation) ---------
\newcommand{\vu}{\vct{u}}
\newcommand{\vv}{\vct{v}}
\newcommand{\vT}{\vct{T}}
\newcommand{\vt}{\vct{t}}
\newcommand{\vTx}{\vct{T}(\vct{x})}
\newcommand{\vx}{\vct{x}}
\newcommand{\vxi}{\vct{\xi}}
\newcommand{\vfb}{\vct{f_b}}
\newcommand{\vf}{\vct{f}}
\newcommand{\vF}{\vct{F}}
\newcommand{\vA}{\vct{A}}
\newcommand{\vFhat}{\widehat{\vct{F}}}
\newcommand{\vJ}{\vct{J}}
\newcommand{\vJhat}{\widehat{\vct{J}}}
\newcommand{\Jdet}{J}
\newcommand{\Jhatdet}{\widehat{J}}
\newcommand{\phihat}{\widehat{\phi}}
\newcommand{\vw}{\vct{w}}
\newcommand{\va}{\vct{a}}
\newcommand{\vahat}{\widehat{\vct{a}}}
\newcommand{\vuhat}{\widehat{\vct{u}}}
\newcommand{\vub}{\widehat{\vct{u_b}}}
\newcommand{\vepsilon}{\vct{\epsilon}}
\newcommand{\vtheta}{\vct{\theta}}
\newcommand{\ej}{\vct{e_j}}
\newcommand{\eo}{\vct{e_1}}
\newcommand{\ef}{\vct{e_1}} % first basis vector
\newcommand{\el}{\vct{e_n}} % nth basis vector

% Learning Stuff
\newcommand{\vW}[1]{\vct{W_{#1}}} % Weight matrix
\newcommand{\vN}[1]{\vct{N_{#1}}} % Neural Network Layer output}
\newcommand{\vb}[1]{\vct{b_{#1}}} % Bias vector
\newcommand{\vsigma}{\vct{\sigma}} % Activation function
\newcommand{\vmL}{\mathcal{L}_{\text{obj}}} % Loss function
\newcommand{\vs}{\vct{s}} % Step direction
\newcommand{\vH}{\vct{H}} % Hessian matrix
\newcommand{\vd}{\vct{d}} % Direction vector 1

% Basis vectors
\newcommand{\generalbasis}{\{\vct{v_1}, \ldots, \, \vct{v_n}\}}
\newcommand{\stdbasis}{\{\vct{e_1}, \ldots, \, \vct{e_n}\}}

% ---------- Operators ----------
\newcommand{\equivto}{\; \equiv \;}
\newcommand{\defined}{\; := \;}
\newcommand{\equals}{\; = \;}
\newcommand{\ol}{\overline}
\newcommand{\pd}{\partial}
\newcommand{\dt}{\frac{d}{dt}}
\newcommand{\pdt}{\frac{\partial}{\partial t}}
\newcommand{\dx}{\, d\vct{x}}
\newcommand{\dxi}{\, d\vct{\xi}}
\newcommand{\dSx}{\, dS_{\vct{x}}}
\newcommand{\argmin}{\operatorname{argmin}}
\DeclareMathOperator{\col}{col}
\DeclareMathOperator{\spn}{span}
\DeclareMathOperator{\range}{range}
\DeclareMathOperator{\kernel}{ker}
\newcommand{\inv}[1]{#1^{-1}}
\DeclareMathOperator{\dia}{diag}

\newcommand{\del}{\Delta}
\newcommand{\Div}{\text{div}}
\newcommand{\Divbf}{\textbf{div}}
\newcommand{\inp}[2]{\langle #1, \; #2 \rangle}
\newcommand{\forallin}[2]{\forall #1 \in #2}
\newcommand{\forallsubsets}[2]{\forall \; #1 \subset #2}
\newcommand{\units}[1]{\; \big[ \; #1 \; \big]}
\newcommand{\blinform}[2]{a\big( #1, \; #2 \big)}

% ---------- Define custom colors ----------
\definecolor{bg}{rgb}{0.95,0.95,0.92}
\definecolor{codeblue}{rgb}{0.25,0.5,0.75}
\definecolor{keyword}{rgb}{0.5,0,0.5}
\definecolor{string}{rgb}{0,0.5,0}

% ---------- Define the style for IPython ----------
\lstdefinelanguage{ipython}{
  basicstyle=\ttfamily\footnotesize\color{black},
  keywordstyle=\color{keyword}\bfseries,
  stringstyle=\color{string},
  backgroundcolor=\color{bg},
  morekeywords={In,Out},
  frame=lines,
  numbers=left,
  numbersep=5pt
}

% ---------- Define the style for Julia ----------
\lstdefinelanguage{julia}{
  keywords={struct, abstract, primitive, typealias, bitstype, mutable, module, baremodule, import, importall, using, export, if, else, elseif, for, while, break, continue, function, return, type, global, local, const, let, do, try, catch, finally, end, quote, macro, where},
  basicstyle=\ttfamily\footnotesize\color{black},
  keywordstyle=\color{keyword}\bfseries,
  stringstyle=\color{string},
  backgroundcolor=\color{bg},
  frame=lines,
  numbers=left,
  numbersep=8pt,
  inputencoding=utf8,
  extendedchars=true,
  escapeinside={(*}{*)},
  literate={~}{{$\sim$}}1
           {₁}{{\textsubscript{1}}}1
           {₂}{{\textsubscript{2}}}1
           {₃}{{\textsubscript{3}}}1
           {ᵀ}{{$^{\mathrm{T}}$}}1
           {⋅}{{$\cdot$}}1
           {∈}{{$\in$}}1
           {ℝ}{{$\mathbb{R}$}}1
           {→}{{$\to$}}1
           {π}{{$\pi$}}1
           {≤}{{$\leq$}}1
           {≥}{{$\geq$}}1
           {≈}{{$\approx$}}1
           {ϵ}{{$\epsilon$}}1
           {κ}{{$\kappa$}}1
           {Δ}{{$\Delta$}}1
           {λ}{{$\lambda$}}1
}

% ---------- Define the style for Mathematica ----------
\lstdefinelanguage{Mathematica}{
  keywords={Module,With,Block,If,Then,Else,Which,Switch,For,While,Return,Do,Table,Plot,Map,Apply,Function,Set,SetDelayed,Clear,Quit,NonlinearModelFit,FindMinimum,FindMaximum,FindRoot,NIntegrate,Integrate,Simplify,FullSimplify,DSolve,RSolve,NSolve,NDSolve,Limit,Series,Assuming,Expand,Factor,TableForm,MatrixForm,Part,Length,Dimensions,Transpose,MapThread,MapAt,Flatten,Thread,Join,Outer,ConstantArray,Riffle,ArrayPlot,Plot3D,ContourPlot,ParametricPlot,MatrixPlot,Graphics,Graphics3D,Manipulate,Evaluate,Sin,Cos,Exp,Log,Sqrt},
  keywordstyle=\color{keyword}\bfseries,
  sensitive=true,
  morecomment=[l]{(*}, morecomment=[r]{*)},
  commentstyle=\itshape\color{gray},
  morestring=[b]",
  stringstyle=\color{string},
  basicstyle=\ttfamily\footnotesize\color{black},
  backgroundcolor=\color{bg},
  frame=lines,
  numbers=left,
  numbersep=5pt,
  showstringspaces=false,
  escapeinside={(*}{*)}
}

% ---------- Optionally define a style for Mathematica code cells ----------
\lstdefinestyle{mathematicaStyle}{
  language=Mathematica,
  frame=lines,
  backgroundcolor=\color{bg},
  basicstyle=\ttfamily\footnotesize,
  keywordstyle=\color{keyword}\bfseries,
  commentstyle=\itshape\color{gray},
  stringstyle=\color{string},
  numbers=left,
  numbersep=5pt,
  showstringspaces=false
}

\usepackage{csquotes} % recommended with biblatex
\usepackage[backend=biber,style=numeric]{biblatex}
\addbibresource{citations.bib}


% ----------------- Document TITLE
\title{Modeling Blood Flow in Macrocirculatory System}
\author{Names Here}
\date{\today}



% ------------------ DOCUMENT START
% -----------------------------------------------------------------
% ------------------
\begin{document} \maketitle

    % ------------------ COVER PAGE
    % -----------------------------------------------------------------
    \noindent Notes pertaining to mathematical models of blood flow in
    the macrocirculatory system.

    \medskip \tableofcontents \medskip
    \newpage

    % ------------------ SECTION: Preliminaries
    % -----------------------------------------------------------------
    \section{Preliminaries}
        \label{sec:preliminaries}
        \medskip
        \subsection*{Mathematical Notation}
            \begin{minipage}[t]{0.48\textwidth}
                \begin{tabular}{@{}p{0.4\linewidth} p{1.8\linewidth}@{}}
                    $ \R$                             & set of real numbers \\
                    $ \R^{+}$                         & set of positive real numbers \\
                    $ \R^{-}$                         & set of negative real numbers \\
                    $ \R^{n}$                         & n-dimensional real vector space \\
                    $ \Omega \subset \R^{n}$          & a connected open subset of $\R^{n}$ \\
                    $ \OmegaBar$                      & the closure of $\Omega$ \\
                    $ \partial \Omega$                & the boundary of $\Omega$ \\
                    $ dx $                            & Lebesgue measure on $\R^{n}$ \\
                    $ dS $                            & surface measure on $\partial \Omega$ \\
                    $ dV $                            & volume measure on $\Omega$ \\
                    $ \nabla $                        & gradient operator \\
                    $ \del $                          & Laplace operator \\
                    $ \Div $                          & divergence of a vector field \\
                    $ \Divbf $                        & divergence of a tensor \\
                    $ \vct{v}_i$                      & $i$-th component of vector $\vct{v}$ \\
                    $ \langle \cdot, \cdot \rangle_X$ & inner product on vector space $X$ \\
                    $ \langle \vct{u}, \vct{v} \rangle$ & inner product of vectors $\vct{u}, \vct{v} \in \R^n$ \\
                    $ \dfrac{\partial}{\partial \hat{\vct{n}}} = \langle \nabla, \hat{\vct{n}} \rangle$ & normal derivative on $\partial \Omega$ \\
                    $\| \cdot \|$                     & $L^2$-norm \\
                    $C^{k}(\Omega)$                   & space of $k$ times continuously differentiable functions on $\Omega$ \\
                    $C^{k}_{0}(\Omega)$               & space of $k$ times continuously differentiable functions with compact support in $\Omega$ \\
                    $C^{k}_{0}(\overline{\Omega})$    & space of $k$ times continuously differentiable functions which have bounded and uniformly
                                                        continuous derivatives up to order $k$ with compact support in $\Omega$ \\
                    $C^{\infty}_{0}(\Omega)$          & space of smooth functions with compact support in $\Omega$ \\
                    $L^{p}(\Omega)$                   & Lebesgue space of $p$-integrable functions on $\Omega$ \\
                \end{tabular}
            \end{minipage}\hfill
        \newpage
        
        \subsection*{Symbols and Abbreviations}
            \label{subsec:symbols}
            \medskip
            \begin{minipage}[t]{0.48\textwidth}
                \begin{tabular}{@{}p{0.4\linewidth} p{1.8\linewidth}@{}}
                    $\therefore$ & consequently\\
                    $\because$ & because\\ 
                    $\implies$ & implies \\
                    $\iff$ & if and only if \\
                    $:=$ & defines\\
                    $\equiv$ & equivalent \\
                    s.t. & such that \\
                    w.r.t. & with respect to\\
                    r.t.a. & referred to as \\
                    a.e. & almost everywhere\\
                    wlog & without loss of generality \\
                    i.e. & "id est" (that means) \\
                    e.g. & "exempli gratia" (for example)\\
                    ODE & Ordinary Differential Equation \\
                    PDE & Partial Differential Equation \\
                    IC & Initial Condition \\
                    BC & Boundary Condition \\
                    0D & Zero dimensional \\
                    1D & One dimensional \\
                    2D & Two dimensional \\
                    3D & Three dimensional \\
                    FSI & Fluid-Structure Interaction \\
                    WHO & World-Health Organization \\
                    SB & Stenotic Bloodflow \\
                    bpm & beats per minute \\
                \end{tabular}
            \end{minipage}
        \medskip

        \subsection*{Parameters and Units}
            \label{subsec:params}
            \medskip
            \begin{minipage}[t]{0.48\textwidth}
                \begin{tabular}{@{}p{0.4\linewidth} p{1.8\linewidth}@{}}
                    $\rho$ & density of blood  $\quad \big[ \frac{kg}{m^3} \big]$\\
                    $\eta$ & dynamic viscosity $\quad \big[Pa \cdot s]$ \\
                    $\mu$ & kinematic viscosity $\quad \big[ \frac{m^2}{s} \big]$\\
                    $\tau$ & shear stress \\
                    $\dot{\gamma}$ & shear rate \\
                    $W_0$ & Wormsley number $\big[ - \big]$ \\
                    $Re$ & Reynolds number $\big[ - \big]$ \\
                    $Pe$ & Peclet number $\big[ - \big]$ \\
                \end{tabular}
            \end{minipage}
        \medskip
    \newpage

    % ------------------ SECTION: Introduction
    % -----------------------------------------------------------------
    \section{Introduction}
        \label{sec:introduction}
        \medskip

        \subsection{Physiological Background}
            \label{subsec:physio-background}

            The circulatory system consists of a human's heart,
            vascular network, lungs and organs. The heart is the source,
            transporting Oxygen-rich blood to the organs and
            deoxygenated (and carbon dioxide enriched) blood back to the lungs;
            which discharge $CO2$ and enrich blood with Oxygen. We refer to these
            respective processes as the \emph{pulmonary circulation} and
            the \emph{systemic circulation} (resp.). The \emph{macrocirculatory
            system}, consists of the heart and the large vessels in the systemic circulation.
            Particularly, the arteries of the macrocirculatory system transport
            oxygenated blood from the heart, driving the return of deoxygenated blood in
            large vessels back to the heart. Hemodynamics refers to the study of blood flow in
            the circulatory system.

            \medskip

            Cardiovascular disease (CVD) is the leading cause of death in developed nations.
            According to the World Health Organization (WHO), CVD accounts for approximately
            30\% of all global deaths in 2012. Understanding the hemodynamics of the macrocirculatory system is crucial for
            diagnosing, treating, and preventing CVD. We aim to develop mathematical models
            that accurately describe blood flow in the macrocirculatory system. Our motivation
            is to use these models to simulate and analyze various cardiovascular conditions,
            such as arterial stenosis, aneurysms, and heart valve disorders.

            \medskip

            A single beat of the heart propels blood through the macrocirculatory system,
            the "lub-dub" sound. We refer to the beat and the sequence of events until the
            successive beat as the \emph{cardiac cycle}. The cardiac cycle consists of
            two main phases: systole and diastole, during which the heart chamber is
            accumulating blood and releasing blood (resp.). The beat can be recognized as a pulse
            wave in large vessels, characterized by the Wormsley number $W_0$:
            \[ 
                W_0 := D \cdot \sqrt{\frac{\omega \rho}{\eta}},
            \]
            a dimensionless parameter comparing the frequency $\omega$
            of the pulse wave to the blood viscosity $\eta$ and the vessel diameter $D$.
            The Reynolds number $Re$ characterizes flow in blood vessels:
            \[
                Re := \frac{\rho U D}{\eta},
            \]
            where $U$ is the mean flow velocity. Low $Re$ indicates laminar flow,
            while high $Re$ suggests turbulent flow. 

            \medskip

            \subsubsection*{Observed Cardiac Cycle Characteristics}
            Normal resting heart rate is considered to be $\omega = 70$ bpm,
            so the cardiac cycle is approximately $0.86 s.$, consisting of:\\
            1. Systole (ventricular contraction) $\approx 0.3$ seconds.\\
            2. Diastole (ventricular relaxation) $\approx 0.7$ seconds.

            \medskip

            The blood volume of human is approximately 5.7-6.0 liters of blood,
            flowing a full cycle approximately every minute. The energy driving
            the flow comes from oxygen and nutrients absorbed from food, creating waste
            products that must be removed; the \emph{coronary artery}'s responsibility.
            The buildup of waste products results in Arteriosclerosis, a narrowing of
            the coronary artery, leading to reduced and turbulent blood flow. (Add citations
            here of turbulence in the presence of stenotic arteries)

            \medskip

            Note \autocite[Table~1.1, p.~10, §1.1]{MathematicalEngineering}
            shows $W_0 \propto D$ and $Re \propto D^{-1}$; we
            observe large pulses and turbulent flow in large vessels and small pulses
            and laminar flow in small vessels.
            
            \medskip


            \subsubsection*{Properties of Blood}
                Blood consists of plasma and cells. The red blood cells
                comprise $\approx 97\%$ of the cell volume, which is $\approx 45\%$
                of blood volume. The remaining blood volume, plasma, is $\approx 90\%$
                water. The ratio of red blood cells and the total blood volume is called the \emph{hematocrit
                value}, a metric describing the influence of red blood cells on the flow.
                \begin{definition}
                    \label{def:newt-fluid}
                    Let $V$ be a fluid volume $V$ with viscosity $\eta = \frac{\tau}{\dot{\gamma}} \equiv$ const.,
                    where \emph{shear stress} $\tau := \frac{F}{A}$ (s.t. $F$ is parallel to it's surface $S$ with area $A$)
                    and \emph{shear rate} $\dot{\gamma}$. Such fluids are \emph{Newtonian fluids}.
                \end{definition}
                A shear stress $\tau$ deforms $V$ by an angle $\gamma ~\implies ~\dot{\gamma}$ is 
                an angular velocity (const. for Newtonian fluids).
                Our models assume blood is Newtonian. We use a 
                relative viscosity $\eta_r$, which depends on the absolute 
                viscosity $\eta$, scaled vessel diameter $D : = D/(1.0 \mu m)$, and
                hematocrit value $H$. In particular, an empirical model from \cite{bviscosity} gives:
                \begin{equation}
                    \label{relative_viscosity}
                    \eta_r = 1 + (\eta_{0.45} - 1) \frac{(1 - H)^C - 1}{(1 - 0.45)^C - 1} ~ : ~ \begin{cases*}
                        \eta_{0.45} = 6 e^{-0.085 D} + 3.2 - 2.44 e^{-0.06 D^{0.645}} \\
                        C = (0.8 + e^{-0.075 D}) \cdot (-1 + \frac{1}{1 + 10^{-11} D^{12}}) + \frac{1}{1 + 10^{-11} D^{12}} \quad
                    \end{cases*}.
                \end{equation}
                In large vessels, $\eta_r$ is observed constant ($\implies$ Newtonian behavior).
                
                \medskip 

                Our concern is the flow of blood in the
                macrocirculatory system. We develop mathematical
                models describing the hemodynamics of arterial and
                large venous segments, then we extend our models to macrocirculation networks
                using a domain decomposition approach.
            \medskip

        \newpage

        \subsection{Navier Stokes System}
            \label{subsec:intro-navier-stokes}

            In \textit{continuum mechanics}, matter is modeled by
            \textit{continuous fields}. For instance, to model the
            behavior of an \textit{incompressible} fluid in time,
            one could use a \textbf{pressure field} and a \textbf{velocity field}.
            At \textit{atomic scales}, this model breaks down—fluids
            as fluids are discrete collections of molecules, not continuous
            fields. However, at the macroscopic scale, one can hope that
            such models remain accurate.
            
            The fundamental mathematical model for incompressible fluids (e.g., water) is the \textbf{viscous incompressible Navier-Stokes equations} (NS):

            \[
            \begin{cases}
                \frac{\partial}{\partial t} \mathbf{u} + (\mathbf{u} \cdot \nabla) \mathbf{u} = \nu \Delta \mathbf{u} - \nabla p, & \text{(Momentum equation: Newton's law } F = ma\text{)} \\
                \nabla \cdot \mathbf{u} = 0, & \text{(Incompressibility condition)} \\
                \mathbf{u}(0, x) = \mathbf{u}_0(x), & \text{(Initial conditions)}
            \end{cases}
            \]

            where $ \mathbf{u} : [0, +\infty) \times \mathbb{R}^3 \to \mathbb{R}^3 $ is the velocity field and
            $ p : [0, +\infty) \times \mathbb{R}^3 \to \mathbb{R} $ is the pressure field.
            The initial velocity field is given by:
            
            \[
                \mathbf{u}_0 : \mathbb{R}^3 \to \mathbb{R}^3, \quad \nabla \cdot \mathbf{u}_0 = 0.
            \]
            
            The \emph{global regularity problem} for (NS) asks:
            \begin{center}
                \textit{For any smooth, spatially localized initial data $ \mathbf{u}_0 $, 
                does there exist a global smooth solution $ (\mathbf{u}, p) $ to (NS)?}
            \end{center}
            This problem is one of the seven Millennium Prize Problems posed 
            by the Clay Mathematics Institute in 2000, with a prize of one million 
            dollars for a correct solution.

            It is known that (NS) has a \emph{local existence and uniqueness theorem}:

            \begin{theorem}
                For any smooth, spatially localized initial data $ \mathbf{u}_0 $, 
                there exists a \textbf{maximal time of existence} $ 0 < T_* \leq +\infty $
                and a solution.
            \end{theorem}

            If $ T_* \neq +\infty $, then \textbf{blowup} occurs in the sense that

            \[
                \sup_{x \in \mathbb{R}^3} |\mathbf{u}(t, x)| \to +\infty \quad \text{as} \quad t \to T_*.
            \]

            If $ T_* = +\infty $, no blowup occurs, and $ |\mathbf{u}| \to 0 $ as $ t \to +\infty $.

            Numerical simulations suggest that global regularity is generally 
            \textbf{true}—for \textit{most} choices of initial data $ \mathbf{u}_0 $, 
            one has a global smooth solution to (NS).

            However, for large initial data, turbulence can occur. Energy moves from 
            coarse to fine scales in a complex way before dissipating due to viscous effects.
            The NS equations contain two competing effects:

            \begin{itemize}
                \item The \textbf{transport term}: $ \mathbf{u} \cdot \nabla \mathbf{u} $
                \item The \textbf{dissipation term}: $ \nu \Delta \mathbf{u} $
            \end{itemize}


            Heuristically, if $ \nu \Delta \mathbf{u} \gg (\mathbf{u} \cdot \nabla) \mathbf{u} $, one expects linear, non-turbulent behavior (\textit{global regularity}).
            If \( (\mathbf{u} \cdot \nabla) \mathbf{u} \gg \nu \Delta \mathbf{u} \), one expects \textbf{nonlinear turbulence} (and possibly \textbf{blowup}).

            \subsubsection*{Kinetic Energy Considerations}

                The kinetic energy is given by:

                \[
                E(t) = \frac{1}{2} \int_{\mathbb{R}^3} |\mathbf{u}(t, x)|^2 \, dx.
                \]

                This quantity is \textit{decreasing in time} due to dissipation. Heuristically:

                \[
                E(t) \gtrsim V^2 L^3.
                \]

                This leads to a bound:

                \[
                    V = O(L^{-3/2}).
                \]
            \medskip

            \subsubsection*{Blowup Scenarios}

                This suggests a \textbf{possible blowup scenario}. If the velocity field concentrates in a ball of radius $ L_1 $, we estimate:

                \[
                    |\mathbf{u}| \approx L_1^{-3/2}.
                \]

                At a later time $ t_2 $, the energy has concentrated further into a ball of radius $ L_2 = L_1 / 2 $, so:

                \[
                    |\mathbf{u}| \approx L_2^{-3/2}.
                \]

                This leads to \textbf{potential infinite velocity} at a single spatial point:

                \[
                    \sum_{n} L_n^{r_2} < \infty.
                \]

                At each stage, the dissipative forces are negligible compared to the transport effects. This results in an approximately self-similar solution.
            \medskip
        \medskip
    \newpage

    \section{Mathematical Models of Blood Flow}
        \label{sec:math-models}
        \medskip

        Mathematical models of blood flow in the macrocirculatory system
        can be classified into different categories based on their dimensionality
        and complexity. The choice of model depends on the specific application,
        computational resources, and desired accuracy.

        \bigskip

    

    \section{Numerical Methods for Blood Flow Simulation}
        \label{sec:num-methods}
        \medskip

        Numerical methods play a crucial role in simulating blood flow in various vascular geometries.
        These methods allow for the approximation of complex fluid dynamics equations and enable the study of
        hemodynamics in realistic scenarios.

        \bigskip

        \todo{
            Provide an overview of the different numerical methods used for blood flow simulation,
            including finite element methods (FEM), finite volume methods (FVM), and computational fluid dynamics (CFD) approaches.
            Discuss the advantages and limitations of each method.
        }



    % ------------------ SECTION: Coronary Artery Stenosis (CAS)
    % -----------------------------------------------------------------
    \section{Misc}
        \label{sec:cas}

        Coronary artery stenosis (CAS) is the narrowing of the
        coronary arteries due to the buildup of plaque. This
        narrowing can restrict blood flow to the heart muscle,
        leading to various cardiovascular problems, including
        chest pain (angina), heart attacks, and other serious
        complications. Current methods for predicting coronary
        artery stenosis are rudimentary; and often prediction
        does mean coronary artery stenosis obstruction (\cite{ehab332}).
        . There is a need for more accurate and reliable methods
        to predict and assess the severity of CAS.
        narrowing can restrict blood flow to the heart muscle,
        leading to various cardiovascular problems, including
        chest pain (angina), heart attacks, and other serious
        complications. Current methods for predicting coronary
        artery stenosis are rudimentary; and often prediction
        does mean coronary artery stenosis obstruction.
       . There is a need for more accurate and reliable methods
        to predict and assess the severity of CAS.

        \bigskip

        We perform a literature survey of arterial blood flow using known
        methods from the literature, with the hope of understanding the
        computational challenges and tradeoffs of various \emph{mathematical models}.

        \bigskip


        \todo{
            Make some comment about "correct terminology for describing the
            types of coronary arterial stenosis is "coronary artery stenosis morphology." and ref
            figures in DOI: 10.1056
        }

        \todo{
            Understand Dr. Zhou's statement: Regarding the assumption about the absence
            of a vortex, I cannot definitively say whether it is correct or not. Please review
            the references provided. If needed, we can discuss this further with interventional
            cardiologists.
        }
        \bigskip


    % ------------------ SECTION: Appendix
    % -----------------------------------------------------------------
    \newpage
    \section{Appendix}
        \label{sec:appendix}
        \medskip

        \printbibliography


        % ------------------ BIBLIOGRAPHY
        \subsection*{References}
            \label{sec:bibliography}
            \medskip

        \medskip

        % ------------------- Code Listings
        \subsection*{Code Listings}

        \td{Optional Space for sumplementary code listings of computations done while investigating}
        \lstset{language=julia}
        \begin{lstlisting}[language=julia, caption={Algorithm 16.5}, label={lst:problem-sampling}]
        function foo()
            println("Hello World)
        end
        \end{lstlisting}



\end{document}
