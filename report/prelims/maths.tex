A \emph{domain} will r.t. an open and bounded subset of $\R^N$ with nonempty interior $\Omg^\circ$,
for all $N\in\{1,2,3\}$. By the Heine-Borel theorem, every domain $\Omg$ has a well-defined boundary
$\partial \Omg$ with compact closure $\OmgBar = \Omg \cup \pOmg$. The compact domains of $\R^N$ are
precisely the closed and bounded subsets of $\R^N$. When every path between two points in $\Omg$
may be continously contrated to a point without leaving $\Omg$, we say that $\Omg$ is \emph{simply connected}.
A \emph{Lipschitz domain} is a domain $\Omg$ such that for every point $x\in\pOmg$, there exists a
neighborhood $U_x$ of $x$ such that $\Omg \cap U_x$ is the region above the graph of a Lipschitz
continuous function (after a suitable rotation and translation of coordinates). A \emph{convex domain}
is a domain $\Omg$ such that for every $x,y\in\Omg$, the line segment connecting $x$ and $y$ is
contained in $\Omg$.
