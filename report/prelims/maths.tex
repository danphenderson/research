% domain disscussion
We assume Zermelo-Fraenkel set theory with the axiom of choice (ZFC), and according to
Cohen\href{https://en.wikipedia.org/wiki/Paul_Cohen}, it's consistent to assume the
the continuum hypothesis (CH); a necessary posulate in continuum mechanics.
A \emph{domain} r.t. an open and bounded subset of $\R^{N+1}$ with nonempty interior $\Omg^\circ$,
for $N\in\{1,2,3\}$. By the Heine-Borel theorem, every domain $\Omg$ has a well-defined boundary
$\partial \Omg$ with compact closure $\OmgBar = \Omg \cup \pOmg$. The compact domains of $\R^N$ are
precisely the closed and bounded subsets of $\R^N$.
When every path between two points in $\Omg$ may be continously contrated to a point without leaving
$\Omg$, we say that $\Omg$ is \emph{simply connected}. A \emph{Lipschitz domain} is a domain $\Omg$
s.t for every point $x\in\pOmg$, there exists a neighborhood $U_x$ of $x$ s.t $\Omg \cap U_x$ is the
region above the graph of a Lipschitz continuous function (after a suitable rotation and translation
of coordinates).

\medskip

\td{
    TODO:\\
    ~- Metric sp, normed sp., inner-product sp. defs and notation; include stmt. that
    Norm's induce metrics, metrics induce norms.\\
    ~- Completeness, Banach sp., Hilbert sp. defs and notation. Finite dim sp.s are complete,
    hence banach, and hilbert under a particular choice of inner-product; state such inner-product
    for our sp.s. \\
}

\medskip

% function space discussion
Given domain $\Omg$, the function space $\mathcal{F}(\Omg)$ is the vector space (vec. sp.)
$(F(\Omg), \, K)$ of scalar valued functions $f:\Omg \to K$. E.g. $C^0(\Omg)$ r.t.
$(C(\Omg), \, \R)$, the vec. sp. of continuous real valued functions on $\Omg$.
Let $C^{k}(\Omg)$ be the sp. of continuous and $k$-times continuously differentiable
functions on $\Omg$. Then $C^k(\OmgBar)$ is the sp. of functions $f \in C^k(\Omg)$ s.t $f$
and it's derivatives up to order $k$ may be continuously extended to the boundary $\pOmg$.
The space $L^p(\Omg)$ denotes the Lebesgue sp. of $p$-integrable functions on $\Omg$.
\todo{compact support, locally integrable, lipschitz sp., etc.}.

\medskip

\td{
    TODO:
    ~- Stmt that we forgoe defining the def. of derivative,
    and what it means for func $f$ to be integrable on domain $\Omg$. \\
    For now assume integrable means $\int_\Omg | f | \dx$ is defined and finite. \\
    Assume deriv. is std. unless otherwise sated.
    Reiz represenation theorem. \\
}

\medskip

% basic feild notation
Let $n = \dim(\Omg)$, then we say
\begin{flalign*}
    \quad & \phi:\Omg\to\R \;\text{ r.t. a \emph{scalar field}}, & \\
    \quad & \vf:\Omg\to\R^{n} \;\text{ r.t. a \emph{vector field}}, &&\\
    \quad & \vT:\Omg\to\R^{n\times n} \;\text{ r.t. a \emph{(second-order) tensor field}},&&
\end{flalign*}
and we say that $\phi(\vx)$, $\vf(\vx)$, and $\vTx$ are the values of the fields
at point $\vx \in \Omg$.\footnote{
    The  $\vT \vx : \R^n \to \R^n$ is a linear map,
    and once a basis is fixed, $\exists ~ A \in \R^{n\times n}$ s.t. $\vTx = A \, \vx$ for all $\vx \in \Omg$.}

\medskip

% operators on scalar feilds
If $f \in C^1(\Omg)$, the partial derivative w.r.t. coordinate
$x_i$ is denoted
\begin{flalign*}
    \quad \pd_{x_i} f = \frac{\pd f}{\pd x_i}, &&
\end{flalign*}
the gradient of $f$ is $\nabla f = \left(\pd_{x_i} f\right)_{i\in[n]}$ and
the Laplacian of $f$ is $\del f = \sum_{i\in[n]} \pd_{x_i}^2 f.$
\footnote{In coordinate free-terms the gradient $\nabla f(\vx)$ is defined s.t.
$df = \inp{\nabla f}{\dx}$ for all infinitesimal displacements $\dx$.}

\medskip

For Lebesgue measure $\dx$ on $\R^n$ the integral of $f$ in $\Omg$ is
\begin{flalign*}
    \quad \int_{\Omg} f(\vx) \dx \; \equiv \; \int_\Omg f. ~ &&
\end{flalign*}
If $\pOmg \in C^1$, then $f \in C^{1}(\OmgBar) \supset L^1(\pOmg)$ and the integral at the boundary is
\begin{flalign*}
    \quad \int_{\pOmg} f(\vx) \dSx \; \equiv \; \int_{\pOmg} f. &&
\end{flalign*}
% \href{https://en.wikipedia.org/wiki/Gradient#Definition}
% \todo{Add def for operatorsin cylindrical, spherical coord?}

\newpage

For vector valued $\vf \in C^1(\Omg; \R^N)$, the divergence operator is
\begin{flalign*}
    \quad \Div \, \vf \defined \nabla \cdot \vf = \inp{\vf}{\nabla}_{\R^n} = \sum_{i=1}^N \pd_{x_i} f_i, &&
\end{flalign*}
and the Laplacian and integral are defined component-wise as
\begin{flalign*}
    \quad  \del \, \vf \defined \left( \del f_i \right)_{i\in[N]}
    \quad \text{ and } \quad
    \int_\Omg \, \vf \defined \bigg( \int_\Omg f_i \dx \bigg)_{i\in[N]}. &&
\end{flalign*}

\medskip

For tensor valued $\vT \in C^1(\Omg; \R^{N\times N})$, the divergence operator is
\begin{flalign*}
    \quad \Divbf \, \vT \defined \left( \Div \, \vT_{i\cdot} \right)_{i\in[N]}, &&
\end{flalign*}
where $\vT_{i\cdot}$ is the $i$-th row of $\vT$. The Laplacian and integral are defined
component-wise as
\begin{flalign*}
    \quad  \del \, \vT \defined \left( \del T_{ij} \right)_{i,j\in[N]}
    \quad \text{ and } \quad
    \int_\Omg \, \vT \defined \bigg( \int_\Omg T_{ij} \dx \bigg)_{i,j\in[N]}. &&
\end{flalign*}

\td{
    Work In Progress (WIP)
}
