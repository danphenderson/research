% domain disscussion
\red{Assume Zermelo-Fraenkel set theory with the axiom of choice (ZFC), and according to
Cohen\href{https://en.wikipedia.org/wiki/Paul_Cohen}, it's consistent if we take
the continuum hypothesis (CH); a necessary posulate in continuum mechanics.}
A \emph{domain} r.t. an open and bounded subset of $\R^n$ with nonempty interior $\Omg^\circ$,
for $N\in[3]$. By the Heine-Borel theorem, every domain $\Omg$ has a well-defined boundary
$\partial \Omg$ with compact closure $\OmgBar = \Omg \cup \pOmg$. The compact domains of $\R^N$ are precisely the closed and bounded subsets of $\R^N$.
When every path between two points in $\Omg$ may be continously contrated to a point without leaving $\Omg$, we say that $\Omg$ is \emph{simply connected}.

\medskip

% function sp. discussion
Given domain $\Omg$, the function sp. $\mathcal{F}(\Omg)$ is the vector sp.
$(F(\Omg), \, K)$ of scalar valued functions $f:\Omg \to K$. E.g. $C^0(\Omg)$ r.t.
$(C(\Omg), \, \R)$ all continuous real valued functions on $\Omg$.
Let sp. $C^{k}(\Omg)$ be the continuous and $k$-times continuously differentiable
functions on $\Omg$, then $C^k(\OmgBar)$ r.t. $f \in C^k(\Omg)$ s.t $f$
and it's derivatives up to order $k$ may be continuously extended to the boundary $\pOmg$.
The sp. $C^\infty(\Omg)$ is the intersection of all $C^k(\Omg)$ for $k \geq 0$,
i.e. the infinitely differentiable functions on $\Omg$.
And $C^k(\Omg;\; \R^m)$ denotes the functions $\vf:\Omg \to \R^m$ with components in
$C^k(\Omg)$, i.e., $\vf = (f_i)_{i\in[m]}$ with $f_i \in C^k(\Omg)$ for all $i \in[m]$.

\medskip

% basic feild notation
Let $n = \dim(\Omg)$, then we say
\begin{flalign*}
    \quad & \phi:\Omg\to\R \;\text{ r.t. a \emph{scalar field}}, & \\
    \quad & \vf:\Omg\to\R^{n} \;\text{ r.t. a \emph{vector field}}, &&\\
    \quad & \vT:\Omg\to\R^{n\times n} \;\text{ r.t. a \emph{(second-order) tensor field}},&&
\end{flalign*}
and we write $\phi(\vx)$, $\vf(\vx)$, and $\vTx$\footnote{
    Note, for tensor $\vTx \in \R^{n \times n}$, $\exists$ a linear map  $A : \R^n \to \R^n$ s.t.
    $\vTx(\vu) \mapsto A \, \vu$ for all $\vu \in \R^n$.}
for the values at a point $\vx \in \Omg$.
The $k$-th partial derivative of $\phi$ w.r.t. coordinate $x_i$ is
\begin{flalign*}
    \quad & \pd_{x_i}^k \phi \equivto \frac{\pd^k \phi}{\pd x_i^k}. &&
\end{flalign*}
The generalized derivative of order $|\alpha|$ of scalar field $\phi$ is
\begin{flalign*}
    \quad & D^\alpha \phi \equivto \frac{\pd^{|\alpha|} \phi}{\pd x_1^{\alpha_1} \, \pd x_2^{\alpha_2} \, \ldots \, \pd x_n^{\alpha_n}}, &&
\end{flalign*}
where $\alpha = (\alpha_1, \alpha_2, \ldots, \alpha_n)$ is a multi-index with
$|\alpha| = \alpha_1 + \alpha_2 + \ldots + \alpha_n$. For $|\alpha| = 1$,
$D^\alpha \phi$ is a first-order partial derivative of $\phi$, e.g. $\pd_{x_i} \phi$
where $\alpha_i = 1$ and $\alpha_j = 0 ~ \forall ~ j \neq i.$

\medskip

% basic operators and notation
For real-valued $\phi \in C^1(\Omg)$, the partial derivative w.r.t. coordinate
$x_i$ is
\begin{flalign*}
    \quad \pd_{x_i} \phi = \frac{\pd \phi}{\pd x_i}. &&
\end{flalign*}
In coordinate-free notation, for $\vx \in \Omg$ and $\nhat \in \R^n$, the normal derivative is\todo{define as declared in notation sec}
\begin{flalign*}
    \quad \pd_{\nhat} \phi(\vx) \defined \inp{\nabla \phi(\vx)}{\nhat}_{\R^n}. &&
\end{flalign*}
For Lebesgue measure $\dx$ on $\R^n$ the integral of $\phi$ in $\Omg$ is
\begin{flalign*}
    \quad \int_{\Omg} \phi(\vx) \dx \; \equiv \; \int_\Omg \phi. ~ &&
\end{flalign*}
If $\pOmg \in C^1$, then $\phi \in C^{1}(\OmgBar) \supset L^1(\pOmg)$ and
the integral at the boundary is
\begin{flalign*}
    \quad \int_{\pOmg} \phi(\vx) \dSx \; \equiv \; \int_{\pOmg} \phi. &&
\end{flalign*}
The gradient and Laplacian of $\phi$ are (respectively):
\begin{flalign*}
    \quad & \phi \; \mapsto \; \vct{\nabla}(\phi) \equivto  \nabla \phi
        \equals \big( \pd_{x_1}, \, \ldots, \, \pd_{x_N} \big)^\top \equals \left(\pd_{x_i} \phi\right)_{\forall{i}{[n]}}.&& \\
    \quad & \phi \; \mapsto \; \del \phi \equivto \nabla\cdot\nabla\phi \equals \sum_{\forall{i}{[n]}} \pd_{x_i}^2 \phi. &&
\end{flalign*}

\medskip

For vector-valued $\vf \in C^1(\Omg; \R^n)$ with components $\vf = (f_i)_{\forall{i}{[n]}}$, the divergence is
\begin{flalign*}
    \quad \Div \, \vf \defined \nabla \cdot \vf \equals \sum_{\forall{i}{[n]}} \pd_{x_i} f_i \equals \inp{\vf}{\nabla}_{\R^n}&&
\end{flalign*}
The Laplacian and domain integral of $\vf$ are defined component-wise:
\begin{flalign*}
    \quad &  \vf \; \mapsto \; \del \, \vf \defined \left( \del f_i \right)_{\forall{i}{[n]}} &&\\
    \quad &  \vf \; \mapsto \; \int_\Omg \, \vf \defined \bigg( \int_\Omg f_i \dx \bigg)_{\forall{i}{[n]}}. &&
\end{flalign*}

\medskip

For tensor-valued $\vT \in C^1(\Omg; \R^{n\times n})$ with entries
$\vT = (T_{ij})_{i,j=1}^n$, the divergence operator is\begin{flalign*}
    \quad \Divbf \, \vT \defined \left( \Div \, \vt_{i} \right)_{\forall{i}{[n]}} ~ : ~ \vt_i \text{ is the ith row of } \vT &&
\end{flalign*}
The Laplacian and domain integral of $\vT$ are again defined component-wise:
\begin{flalign*}
    \quad  \del \, \vT \defined \left( \del T_{ij} \right)_{\forall{i,j}{[n]}}
        \quad\text{ and } \
        \int_\Omg \, \vT \defined \bigg( \int_\Omg T_{ij} \dx \bigg)_{\forall{i,j}{[n]}}. &&
\end{flalign*}

\medskip

Let $K$ be compact (e.g. $\OmgBar$), then the sp. $(C^k(K), \, \R)$ is complete with norm
\begin{flalign*}
    \quad \norm{\phi}_{C^k(K)} &\defined \sum_{\forallin{|\alpha|}{[k]}} \sup_{\vx \in K}
        \left| D^\alpha \phi(\vx) \right|. &&
\end{flalign*}
meaning that every Cauchy sequence $\{ \phi_j \}_{\forall j \in \N} \subset C^k(K)$ converges
to a limit in $C^k(K)$. So the complete normed sp. $(C^k(K), \, \norm{\cdot}_{C^k(K)})$ is banach,
by the definition of banach sp. In particular, for $k = 0$, we have
\begin{flalign*}
    \quad \norm{\phi}_{C(\OmgBar)} &\defined \sup_{\vx \in \OmgBar} \left| \phi(\vx) \right|\equivto \max_{\vx \in \OmgBar} \left| \phi(\vx) \right| \equals \norm{\phi}_\infty \quad
    (\text{since $\OmgBar$ is compact}). &&
\end{flalign*}

\medskip

% linear sp. discussion
Let $\phi \in L^p(\Omg)$ be a real-valued measurable function on $\Omg$, then the
Lebesgue sp. $L^p(\Omg)$ is defined as
\begin{flalign*}
    \quad L^p(\Omg) \defined \left\{ \phi : \Omg \to \R ~ \Big| ~ \norm{\phi}_{L^p(\Omg)} < \infty \right\} \quad \text{ s.t. } \quad \norm{
        \cdot }_{L^p(\Omg)} : \phi \mapsto \left( \int_\Omg |\phi(\vx)|^p \dx \right)^{1/p} \quad 1 \leq p < \infty. &&
\end{flalign*}
For $p = \infty$, the sp. $L^\infty(\Omg)$ is defined as
\begin{flalign*}
    \quad L^\infty(\Omg) \defined \left\{ \phi : \Omg \to \R ~ \Big| ~ \norm{\phi}_{L^\infty(\Omg)} < \infty \right\} \quad \text{ s.t. } \quad \norm{
        \cdot }_{L^\infty(\Omg)} : \phi \mapsto \sup_{\vx \in \Omg} |\phi(\vx)|. &&
\end{flalign*}
where $\sup$ here r.t. the essential supremum. It follows that $\vf \in L^p(\Omg;\; \R^m)$ denotes the functions whose components
are in $L^p(\Omg)$, i.e., $\vf = (f_i)_{i\in[m]}$ with $f_i \in L^p(\Omg)$ for all $i \in[m]$. The sp $(L^p(\Omg), \, \norm{\cdot}_{L^p(\Omg)})$ is complete,
hence a banach sp. for all $1 \leq p \leq \infty$.

\medskip
When refering to a physical quantity, it's measure units are indicated as $\big[ \cdot \big]$;
whereas dimensionless quantities are indicated as $\big[ - \big]$.
\medskip

\td{
    TODO:\\
    ~- forgoe until we work with a Lipschitz boundary: Hölder spaces, then Hilbert and Sobolev spaces. defs and notation,
    then trace thm., Poincare inequality, Rellich-Kondrachov comp. thm., etc.
    Distributional derivatives, test functions, multi-index notation, weak derivatives, weak
    formulation, etc.\\
    ~- Add def. and notation for Lipschitz continuous functions/sp., $Lip(\Omg)$ and
    it's norm.\\
    ~- Add relation of $Lip(\Omg)$ to $C^0(\Omg)$, brenner's book. Don't state why,
    just cite.\\
    ~- Proceed into .\\
}

\medskip

\td{
    Work In Progress (WIP): Misc items.\\
    ~- Write cylindrical operators\\ % \href{https://en.wikipedia.org/wiki/Gradient#Definition}
    ~- Write material on directional derivatives and gradients\\
    ~- Write material on Jacobian and Hessian matrices\\
    ~- More detail on integration? Fubini's thm., change of variables, etc.\\
    ~- Scaling argument for completeness of $C^k(K)$ sp. norm and nondimensionalization\\
    ~- Define metric sp, normed sp., inner-product sp. defs and notation; include stmt. that
       norm's induce metrics, metrics induce norms.\\
    ~- Finite dim sp.s are complete, hence banach, and hilbert under a particular choice of
       inner-product; state such inner-product for their sp.s. \\
}
