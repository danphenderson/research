% domain disscussion
\red{Assume Zermelo-Fraenkel set theory with the axiom of choice (ZFC), and according to
Cohen\href{https://en.wikipedia.org/wiki/Paul_Cohen}, it's consistent if we take
the continuum hypothesis (CH); a necessary posulate in continuum mechanics.}
A \emph{domain} r.t. an open and bounded subset of $\R^{N+1}$ with nonempty interior $\Omg^\circ$,
for $N\in\{1,2,3\}$. By the Heine-Borel theorem, every domain $\Omg$ has a well-defined boundary
$\partial \Omg$ with compact closure $\OmgBar = \Omg \cup \pOmg$. The compact domains of $\R^N$ are
precisely the closed and bounded subsets of $\R^N$.
When every path between two points in $\Omg$ may be continously contrated to a point without leaving
$\Omg$, we say that $\Omg$ is \emph{simply connected}. A \emph{Lipschitz domain} is a domain $\Omg$
s.t for every point $x\in\pOmg$, there exists a neighborhood $U_x$ of $x$ s.t $\Omg \cap U_x$ is the
region above the graph of a Lipschitz continuous function (after a suitable rotation and translation
of coordinates).

\medskip

\td{
    TODO:\\
    ~- Metric sp, normed sp., inner-product sp. defs and notation; include stmt. that
    Norm's induce metrics, metrics induce norms.\\
    ~- Completeness, Banach sp., Hilbert sp. defs and notation. Finite dim sp.s are complete,
    hence banach, and hilbert under a particular choice of inner-product; state such inner-product
    for our sp.s. \\
    ~- Forgoe until we work with a Lipschitz boundary: Sobolev sp. defs and notation,
    trace thm., Poincare inequality, Rellich-Kondrachov comp. thm., etc.
    Distributional derivatives, test functions, multi-index notation
}

\medskip

% function space discussion
Given domain $\Omg$, the function space $\mathcal{F}(\Omg)$ is the vector space (vec. sp.)
$(F(\Omg), \, K)$ of scalar valued functions $f:\Omg \to K$. E.g. $C^0(\Omg)$ r.t.
$(C(\Omg), \, \R)$, the vec. sp. of continuous real valued functions on $\Omg$.
Let $C^{k}(\Omg)$ be the sp. of continuous and $k$-times continuously differentiable
functions on $\Omg$. Then $C^k(\OmgBar)$ is the sp. of functions $f \in C^k(\Omg)$ s.t $f$
and it's derivatives up to order $k$ may be continuously extended to the boundary $\pOmg$.
The space $L^p(\Omg)$ denotes the Lebesgue sp. of $p$-integrable functions on $\Omg$.
\todo{compact support, locally integrable, lipschitz sp., etc.}.

\medskip

\td{
    TODO:\\
    ~- Add relation of $Lip(\Omg)$ to $C^\infty$, brenner's book. Don't state why,
    just cite.\\
    ~- Stmt that we forgoe defining the def. of derivative,\\
    and what it means for func $f$ to be integrable on domain $\Omg$. \\
    For now assume integrable means $\int_\Omg | f | \dx$ is defined and finite. \\
    Assume deriv. is std. unless otherwise sated.
    Reiz represenation theorem. \\
}

\medskip

% basic feild notation
Let $n = \dim(\Omg)$, then we say
\begin{flalign*}
    \quad & \phi:\Omg\to\R \;\text{ r.t. a \emph{scalar field}}, & \\
    \quad & \vf:\Omg\to\R^{n} \;\text{ r.t. a \emph{vector field}}, &&\\
    \quad & \vT:\Omg\to\R^{n\times n} \;\text{ r.t. a \emph{(second-order) tensor field}},&&
\end{flalign*}
and we write $\phi(\vx)$, $\vf(\vx)$, and $\vTx$ for the values at a point $\vx \in \Omg$.
at point $\vx \in \Omg$.\footnote{
    Note, for tensor $\vTx \in \R^{n \times n}$, $\exists$ a linear map  $A : \R^n \to \R^n$ s.t.
    $\vTx(\vu) \mapsto A \, \vu$ for all $\vu \in \R^n$.


    so $\exists ~ A \in \R^n \to \R^n$ is a linear map,
    and once a basis is fixed, $\exists ~ A \in \R^{n\times n}$ s.t. $\vTx = A \, \vx$ for all $\vx \in \Omg$.}

\medskip

% operators on scalar feilds
For real-valued $\phi \in C^1(\Omg)$, the partial derivative w.r.t. coordinate
$x_i$ is
\begin{flalign*}
    \quad \pd_{x_i} \phi = \frac{\pd \phi}{\pd x_i}. &&
\end{flalign*}
In coordinate-free notation, for $\vx \in \Omg$ and $\nhat \in \R^n$, the normal derivative is\todo{define as declared in notation sec}
\begin{flalign*}
    \quad \pd_{\nhat} \phi(\vx) \defined \inp{\nabla \phi(\vx)}{\nhat}_{\R^n}. &&
\end{flalign*}
For Lebesgue measure $\dx$ on $\R^n$ the integral of $\phi$ in $\Omg$ is
\begin{flalign*}
    \quad \int_{\Omg} \phi(\vx) \dx \; \equiv \; \int_\Omg \phi. ~ &&
\end{flalign*}
If $\pOmg \in C^1$, then $\phi \in C^{1}(\OmgBar) \supset L^1(\pOmg)$ and
the integral at the boundary is
\begin{flalign*}
    \quad \int_{\pOmg} \phi(\vx) \dSx \; \equiv \; \int_{\pOmg} \phi. &&
\end{flalign*}
The gradient and Laplacian of $\phi$ are (respectively):
\begin{flalign*}
    \quad & \phi \; \mapsto \; \vct{\nabla}(\phi) \equivto  \nabla \phi \equals \big( \pd_{x_1}, \, \ldots, \, \pd_{x_N} \big)^\top \equals \left(\pd_{x_i} \phi\right)_{i\in[n]}.&& \\
    \quad & \phi \; \mapsto \; \del \phi \equivto \nabla\cdot\nabla\phi \equals \sum_{i\in[n]} \pd_{x_i}^2 \phi. &&
\end{flalign*}
\todo{write operators in cylindrical coordinates}

\medskip


% \href{https://en.wikipedia.org/wiki/Gradient#Definition}
% \todo{Add def for operatorsin cylindrical, spherical coord?}

\medskip

For vector-valued $\vf \in C^1(\Omg; \R^N)$ with components $\vf = (f_i)_{i\in[N]}$, the divergence is
\begin{flalign*}
    \quad \Div \, \vf \defined \nabla \cdot \vf \equals \sum_{i=1}^N \pd_{x_i} f_i \equals \inp{\vf}{\nabla}_{\R^n}z &&
\end{flalign*}
The Laplacian and domain integral of $\vf$ are defined component-wise:
\begin{flalign*}
    \quad &  \vf \; \mapsto \; \del \, \vf \defined \left( \del f_i \right)_{i\in[N]} &&\\
    \quad &  \vf \; \mapsto \; \int_\Omg \, \vf \defined \bigg( \int_\Omg f_i \dx \bigg)_{i\in[N]}. &&
\end{flalign*}

\medskip

For tensor-valued $\vT \in C^1(\Omg; \R^{n\times n})$ with entries
$\vT = (T_{ij})_{i,j=1}^n$, the divergence operator is\begin{flalign*}
    \quad \Divbf \, \vT \defined \left( \Div \, \vt_{i} \right)_{i\in[N]} ~ : ~ \vt_i \text{ is the ith row of } \vT &&
\end{flalign*}
The Laplacian and domain integral of $\vT$ are again defined component-wise:
\begin{flalign*}
    \quad  \del \, \vT \defined \left( \del T_{ij} \right)_{i,j\in[N]}
    \quad \text{ and } \quad
    \int_\Omg \, \vT \defined \bigg( \int_\Omg T_{ij} \dx \bigg)_{i,j\in[N]}. &&
\end{flalign*}

\td{
    Work In Progress (WIP)
}
