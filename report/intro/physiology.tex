\label{subsec:physio-background}
The circulatory system is the human heart, vascular network, lungs, and organs. The systems
source is the heart, transporting oxygen-rich blood to the organs and deoxygenated (and
carbon dioxide-enriched) blood back to the lungs. Lungs discharge $CO2$ and enrich the blood
with Oxygen. These processes are r.t. the \emph{pulmonary circulation} and
the \emph{systemic circulation} (resp.). The \emph{macrocirculatory system} consists of the
heart and the large vessels in the systemic circulation. Particularly, the arteries of the
macrocirculatory system transport oxygenated blood from the heart, driving the return
of deoxygenated blood in large vessels back to the heart.

\medskip

A single beat of the heart propels blood through the macrocirculatory system, the
"lub-dub" sound. The beat and the following sequence of events until the successive
beat as the \emph{cardiac cycle}. The cardiac cycle consists of two main phases:
systole and diastole, during which the heart chamber is accumulating blood and
releasing blood (resp.). The beat can be recognized as a pulse wave in large vessels,
characterized by the Wormersley number
\begin{flalign*}
        \quad & W_0 := 2R \cdot \sqrt{\frac{\omega \rho}{\eta}},  \quad : ~ \rho \text{ is fixed blood density} &&
\end{flalign*}
a dimensionless parameter comparing the frequency $\omega$ of the pulse wave to the
blood's dynamic viscosity $\eta$ and the vessel diameter $2R$.
The Reynolds number $Re$ characterizes flow in blood vessels
\begin{flalign*}
        \quad & Re := 2R \cdot \frac{\rho U}{\eta}, \quad : ~ \rho \text{ is fixed blood density} &&
\end{flalign*}
Low $Re$ indicates laminar flow, while high $Re$ suggests turbulent flow.
\todo{Double check $W_0$ and $Re$ definitions with book1 and hemodynamics refs}

\medskip

\subsubsection*{Observed Cardiac Cycle Characteristics}
Normal resting heart rate is considered to be $\omega = 70$ bpm,
so the cardiac cycle is approximately $0.86 s.$, consisting of:\\
1. Systole (ventricular contraction) $\approx 0.3$ seconds.\\
2. Diastole (ventricular relaxation) $\approx 0.7$ seconds.

\medskip

The blood volume of a human is approximately 5.7-6.0 liters of blood,
flowing a full cycle roughly every minute. The energy driving
the flow comes from oxygen and nutrients absorbed from food, creating waste
products that must be removed; the \emph{coronary artery}'s responsibility.
The buildup of waste products results in Arteriosclerosis, a narrowing of
the coronary artery, leading to reduced and turbulent blood flow. (Add citations
here of turbulence in the presence of stenotic arteries).

\medskip

Note \autocite[Table~1.1, p.~10, §1.1]{book1}
shows $W_0 \propto 2R$ and $Re \propto (2R)^{-1}$; we
observe large pulses and turbulent flow in large vessels and small pulses
and laminar flow in small vessels.

\medskip

\paragraph{Constituents and hematocrit.}
Blood consists of plasma and formed elements which we call cells. Red blood cells (RBCs) comprise
$\approx 97\%$ of the cellular volume, and cellular volume is approximately $\approx 45\%$ of the blood volume.
The remaining $\approx 55\%$ of blood volume is plasma, which is $\approx 90\%$ water. The ratio of RBC
volume to total blood volume is the \emph{hematocrit value} $H$, a key metric governing apparent viscosity
$\eta$: as $H$ increases, $\eta$ typically increases (cf.\ S6.5.1~\cite{hemodynamics}).
The formed elements suspended in plasma include white blood cells (WBCs) and platelets.
