W.a.t. simulate blood flow in a time-dependent fluid domain $\OmgB \subset \R^{N+1}$;
we'll take $N \equivto 3$, the natural setting for the derivations that follow.
For each $t \in I_T \defined [0, T] \subset \R$, with $T > 0$, we define
\begin{flalign*}
    \quad \OmgBt &:= \{\vx \in \R^N : \vx \text{ lies inside the vessel at time } t\}. &&
\end{flalign*}
Formally, the fluid region is a time-dependent family of open sets
$\{\OmgBt\}_{\forallin{t}{I_T}} \subset \R^N$ occupied by blood at time $t$.
Let $\OmgB(0)$ r.t. the \emph{reference} configuration and $\OmgBt$
r.t. the \emph{current} configuration.
The fluid's computational domain r.t. all possible configurations of $\OmgBt$, i.e.,
the spatial-temporal lumen
\begin{flalign*}
    \quad & \OmgB \defined I_T \times \OmgBt \equivto \{ (t, \vx) \in \R^{N+1} ~:~ x \in \Omgt, ~ t \in I_T \}. &&
\end{flalign*}

\medskip

For each material particle initially at $\vxi \in \Omega_B(0)$,
we assume its motion in $\Omega_B$ is governed by a velocity field
$\vu : \Omega_B \to \R^N$.  Our aim here is to derive the governing PDE system
for $\vu$—the Navier–Stokes equations.
\medskip

\subsubsection{Blood Dynamics}
    \label{subsubsec:blood-dynamics}

    % Coordinate defs
    Let the flow map be
    \begin{flalign*}
        \quad & \vx: I_T \times \Omega_B(0) \to \R^N, \quad (t, \vxi) \mapsto \vx(t, \vxi) &&
    \end{flalign*}
    For each fixed $t \in I_T$, the \emph{Lagrangian map} is from the reference configuration to the current configuration, i.e.
    \begin{flalign*}
        \quad & \La_t : \Omega_B(0) \to \Omega_B(t), \quad \vxi \mapsto \vx(t, \vxi) \quad \text{ s.t. }
        \quad \La_t(\vxi) \equals \vx(t, \vxi). &&
    \end{flalign*}
    The variables $(t,\vx)$ and $(t,\vxi)$ are referred to as the Eulerian and Lagrangian
    coordinates, resp..

    \medskip

    \begin{remark}[Eulerian vs. Lagrangian]
        \label{rmk:eulerian-lagrangian}
        Informally, the Eulerian approach focuses our attention to $\vx \in \OmgBt$, namely,
        some fluid particle located at $\vx$ at a particular time $t$.
        Whereas, the Lagrangian approach tracks an individual fluid particle $\vxi \in \OmgB(0)$
        along it's trajectory $\vT_{\vxi} \defined \{ \vx(t,\vxi) : t \in I_T \}$ as it moves along a
        characteristic curve in $\OmgB$.
        \medskip
        For $\phi \in C^1(\OmgB)$ s.t. $(t, \vx) \mapsto \phi(t, \vx)$ in the Eulerian variables,
        we'll let $\hat{\phi}(t,\vxi) \defined \phi\bigl(t,\La_t(\vxi)\bigr)$ s.t. $\vx(t) = \La_t(\vxi)$. I.e.,
        we'll let $\phihat \equals \phi \circ \La_t$ and it follows that $\phi \equals \phihat \circ \inv{\La_t}$.
    \end{remark}

    \medskip

    % Principle kinematic quantity: velocity def. via Lagrangian Reference Frame
    \begin{definition}
        \label{def:lagrangian-velocity}
        Let the Lagrangian velocity field be defined as
        \begin{flalign*}
            \quad \vuhat \defined \pd_t \vx(t, \vxi) \mapsto \vuhat(t, \vxi)
                \quad \forallin{(t, \vxi)}{I_T \times \OmgB(0)} &&
        \end{flalign*}
    \end{definition}
    A change of coordinates in~\ref{def:lagrangian-velocity} yields the velocity field $\vu$
    in the Eulerian frame, i.e.,
    \begin{flalign*}
        \quad \vu \equals \vuhat \circ \inv{\La_t} \quad \iff \quad \vu(t, \vx)
            \equals \vuhat(t, \inv{\La_t}(\vx)) &&
    \end{flalign*}
    Then if $\vuhat$ and $\OmgB(0)$ are known, we arrive at the Cauchy problem
    for the trajectory $T_{\vxi}$ of each material particle $\vxi \in \OmgB(0)$:
    \begin{flalign*}
        \quad \begin{cases*}
            \pd_t{\vx}(t, \vxi) \equals \vuhat(t, \vxi), \forallin{t}{I_T} \\
            \vx(0, \vxi) \equals \vxi.
        \end{cases*} &&
    \end{flalign*}

    \td{State conditions for well-posedness in the sense of hadamard? Explain existence, uniqueness, stability of solutions of the resulting
    ODE system above via the Picard-Lindelof theorem.}


    \medskip

\subsubsection*{Material Derivatives}

    \begin{definition}
        \label{def:material-derivative}
        Let $\phi \in C^k(\OmgB)$ and $\phihat \equals \phi \circ \La_t$, then the \emph{material derivative}
        $D_t \phi$ is the derivative of $\phi$ w.r.t. $t$ in the Lagrangian frame expressed as a function in the Eulerian frame, i.e.,
        \begin{flalign*}
            \quad D_t \big( \phi(t, \vx) \big) &\defined \pd_t \phihat(t, \vxi), \quad \text{ s.t. } \vxi \equals \inv{\La_t}(\vx) &&\\
                & \equals \dfrac{d}{dt}\phi(t, \vx(t, \vxi)), \quad \forallin{\vxi}{\OmgB(0)}. &&
        \end{flalign*}
    \end{definition}
    By the multivariable chain rule,
    \begin{flalign*}
        \quad D_t \big( \phi(t, \vx) \big) &\equals \pd_t \phi(t,\vx) + \nabla \phi(t,\vx)\cdot \vu(t,\vx) &&\\
            &\equals \pd_t \phi + \inp{\nabla \phi}{\vu}_{\R^N} \quad \text{ in } \OmgB &&
    \end{flalign*}
    where one may see that the material derivative $D_t \phi$ measures the rate of variation of $\phi$ along trajectory $T_{\vxi}$
    expressed in the Eulerian frame. The literature also r.t. $D_t$ as the \emph{substantial derivative}, \emph{advective derivative}, \emph{lagrangian derivative}, or
    \emph{convective derivative}.

\medskip

Now differentiating $\vuhat$ w.r.t. $t$ yields the acceleration vector field $\vahat$ in the Lagrangian and Eulerian frame
repsectively:
\begin{flalign*}
    \quad \vahat(t, \vxi) &\defined \pd_t \vuhat(t, \vxi) = \pd_t^2 \vx(t, \vxi) \quad  \forallin{\vxi}{\OmgB(0)} &&\\
        &\iff \va \equals D_t \vu \equals \pd_t \vu + (\vu \cdot \nabla) \vu &&\\
        \va(t, \vx) &\equals \pd_t \vu(t, \vx) +  (\vu(t,\vx)\cdot\nabla)\vu(t,\vx) &&\\
        \va(t, \vx) &\equals \pd_t \vu(t, \vx) + \sum_{i=1}^{N} u_i(t, \vx) \pd_{x_i} \vu(t, \vx) \quad \forallin{(t, \vx)}{\OmgB}. &&
\end{flalign*}

\medskip
Let's denote the deformation gradient tensor $\vFhat_t$ and its counterpart $\vF_t$ be the gradient of $\La_t$ and $\inv{\La_t}$ in,
defined respectively as:
\begin{flalign*}
    \quad \vFhat_t(\vxi) &\defined \nabla_{\vxi} \La_t(\vxi) \quad \forallin{\vxi}{\OmgB(0)}. &&\\
        &\iff \vF_t(\vx) \equals \nabla_{\vxi} \La_t(\inv{\La_t}(\vx)) \equals \pd_{\vxi} \vx(t, \vxi) \quad \forallin{(t, \vx)}{\OmgB}. &&
\end{flalign*}
The tensor $\vF_t$ measures the spatial deformation of infinitesimal fluid elements as they move from
the reference configuration to the current configuration under $\La_t$ along trajectory $T_{\vxi}$.
We assume that, for each fixed $t$, the mapping
$\La_t : \OmgB(0) \to \OmgB(t)$ is a $C^1$–diffeomorphism with $\det \vFhat_t(\vxi) > 0 \forallin{\vxi}{\OmgB(0)}$\footnote{
    Recall the $\det \vFhat$ may be interpreted as the signed volume change of an infinitesimal parallelepiped
    spanned by the column vectors of $\vFhat$; thus, $\det \vFhat_t(\vxi) > 0$ ensures that the orientation of fluid elements
    is preserved under the deformation mapping $\La_t$ at time $t$. Moreover, it prevents unphysical scenarios
    such as interpenetration of matter consisting of negative volumes.
}.
Then $\vFhat_t(\vxi)$ is invertible, and we define
\begin{flalign*}
    \quad \Jhatdet_t(\vxi) & \defined \det\big(\vFhat_t(\vxi)\big) \quad \forallin{\vxi}{\OmgB(0)} &&\\
        &\iff \Jdet_t(\vx) \equals \det\big(\vFhat_t(\inv{\La_t}(\vx))\big) \quad \forallin{(t, \vx)}{\OmgB} &&
\end{flalign*}
We r.t. such measure of spatial deformation $\Jhatdet_t(\vxi)$ as the Jacobian determinant of $\La_t$ at
time $t$. The following lemma shows that $\pd_t \Jdet_t$ relates to the fluids divergence $\Div(\vu)$.

\medskip

\begin{lemma}
    \label{lem:jacobian-determinant-derivative}
    The Jacobian determinant $\Jdet_t(\vx)$ satisfies
    \begin{flalign*}
        \quad & D_t \Jdet_t(\vx) \equals \Jdet_t(\vx) \, \Div(\vu(t, \vx)) \quad \forallin{(t, \vx)}{\OmgB}. &&
    \end{flalign*}
    \begin{proof}
        Ref~\cite{book2}, lemma 2.1, pg. 20, for proof.
    \end{proof}
\end{lemma}

Literature often r.t. relation in lem.~\ref{lem:jacobian-determinant-derivative} as the
\emph{Euler expansion formula}, or \emph{Jacobi's equation}.

\medskip

% Reynolds Transport
Let $V(t)$ denote a measurable material volume at time $t$ that moves with the fluid flow, i.e.,
\begin{flalign*}
    \quad V(t) &\subset \OmgBt, \quad \text{for each } t \in I_T &&
\end{flalign*}
Equivalently, $V(t)$ is the image of some reference volume $V(0) \subset \OmgB(0)$ under the
Lagrangian mapping $\La_t$, i.e.,
\begin{flalign*}
    \quad V(t) & = \La_t\big(V(0)\big). &&
\end{flalign*}
The following theorem, known as the Reynolds Transport Theorem, relates the time derivative of an integral
over the moving volume $V(t)$ to integrals over $V(t)$ and its boundary $\p V(t)$.
\begin{theorem}[Reynolds Transport Theorem]
    \label{thm:reynolds-transport}
    Let $V(0) \subset \OmgB(0)$ be a material volume in the reference configuration,
    and $V(t) \subset \OmgBt$ its image in the current configuration under $\La_t$. For any
    sufficiently smooth scalar field $\phi: \OmgB \to \R$,
    \begin{flalign*}
        \quad & \frac{d}{dt} \int_{V(t)} \phi(t, \vx) \, dV \defined \int_{V(t)} \pd_t \phi(t, \vx) \, dV
            + \int_{\p V(t)} \phi(t, \vx) \, \inp{\vu}{\nhat} \, dS, &&
    \end{flalign*}
    where $\vu$ is the fluid velocity field and $\nhat$ the outward unit normal on $\p V(t)$.
\end{theorem}
\begin{proof}
    Ref~\cite{book2}, theorem 2.2. pg. 21.
\end{proof}

\subsection{NS-derivation}
In the following derivations we let $N \equiv 3$ so that $\OmgBt \subset \R^3$. We assume blood fluid is
a continuum that deforms continuously, i.e., at every point $\vx \in \OmgBt$ and time $t \in I_T$, the
blood's kinematic quantities are described by sufficiently smooth fields. At microscopic scales this
continuum hypothesis breaks down, since matter is a discrete collection of molecules, but at macroscopic
scales empirical evidence suggests such models remain accurate.

\medskip

For each $t \in I_T$, herein assume $\OmgBt$ is simply connected and boundary $\pOmgB(t) \in C^1$,
so, any closed curve in $\OmgBt$ can be continuously contracted to a point within $\OmgBt$; such an
assumption is reasonable in a healthy vessel with a smooth wall. This regularity allows us to globally
define geometric quantities such as surface differential operators on the fluids boundary $\pOmgB$, then
we may apply stokes theorem and Green's identities on the $C^1$ boundaries, useful for obtaining weak
integro-differential and variational formulations.

\medskip

\begin{remark}
    A weaker regularity condition is that $\pOmgB$ is Lipschitz. Here we must
    rely on standard trace theorems for Sobolev spaces (e.g.\ $H^1(\OmgBt)$). In particular,
    the outward unit normal is then defined a.e. on $\pOmgB(t)$, so we can meaningfully speak of
    normal and tangential components of vector fields on the boundary.
\end{remark}

\medskip
Let blood's velocity, thermodynamic pressure, and density fields be
\begin{flalign*}
    \quad & \vu: \OmgB \to \R^3,\;
        (t, x, y, z) \mapsto (u_1(t, x, y, z),\, u_2(t, x, y, z),\, u_3(t, x, y, z))^\top,
        \quad \units{\frac{cm}{s}} && \\
    \quad & p: \OmgB \to \R^+, \;
        (t, x, y, z) \mapsto p(t, x, y, z),
        \quad \units{\frac{N}{cm^2}} && \\
    \quad & \rho: \OmgB \to \R^+,\;
        (t, x, y, z) \mapsto \rho(t, x, y, z),
        \quad \units{\frac{kg}{cm^3}}. &&
\end{flalign*}
\begin{definition}[Incompressible fluid]
    \label{def:incompressible-fluid}
    A fluid is \emph{incompressible} if for each material element $V(0)\subset \Omega_B(0)$,
    the density of that element remains constant in time, i.e.
    \begin{flalign*}
        \quad \rho\bigl(t,\vx(t,\vxi)\bigr) \equals \rho_{V(0)} \in \R^+, \quad  \forallin{(t, \vxi)}{I_T \times V(0)}, &&
    \end{flalign*}
    where constant $\rho_{V(0)}$ can depend on initial fluid element $V(0)$.
\end{definition}
\begin{definition}[Incompressible flow]
    \label{def:incompressible-flow}
    A flow is \emph{incompressible} in $\OmgB$ if the material derivative of the density vanishes,
    \begin{flalign*}
        \quad D_t \rho = 0 \quad \text{in } \OmgB. &&
    \end{flalign*}
\end{definition}
In our model we further assume a uniform initial density, so def~\ref{def:incompressible-fluid} may be written as
\begin{flalign*}
    \quad & \rho_{V(0)} \equals \rho_0 \in \R^+ \quad \forallsubsets{V(0)}{\OmgB(0)}, &&
\end{flalign*}
for some constant $\rho_0$ independent of material element $V(0)$.
Therefore uniform density in the reference configuration implies constant density along
trajectory $T_{\vxi}$ to any current configuration, i.e.,
\begin{flalign*}
    \quad  \rho\bigl(t,\vx(t,\vxi)\bigr) & \equals \rho_0 \in \R^+, \quad  \forallin{(t, \vxi)}{I_T \times \OmgB(0)}. && \\
    \quad & \implies \rho(t, \vx) \equals \rho_0 \in \R^+, \quad  \forallin{(t, \vx)}{\OmgB}. &&
\end{flalign*}
Taking the material derivative yields $D_t \rho = 0$ in $\OmgB$ -- In summary, we've shown:
\begin{flalign*}
    \quad \text{Incompressibile Fluid~\ref{def:incompressible-fluid} }
        ~ \implies ~ \text{Incompressible Flow~\ref{def:incompressible-flow}}. &&
\end{flalign*}
Note, the converse is not generally true: an incompressible flow ($D_t\rho=0$) only preserves density
along $T_{\vxi}$, allowing $\rho=\rho(\vx)$ to vary spatially in the Eulerian frame. But particularly
here, uniform initial density in $\OmgB(0)$ implies constant density in $\OmgB$ under incompressible flow.
\newpage
\begin{remark}[Divergence-free condition]
    \label{rmk:incompressibility-divergence-free}
    Mass conservation for a fluid with density $\rho$ and velocity $\vu$
    yields the continuity equation
    \begin{flalign*}
        \quad \partial_t \rho + \Div(\rho \vu) = 0
        \quad \Longleftrightarrow \quad
        D_t \rho + \rho\,\Div \vu = 0
        \quad \text{in } \Omega_B. &&
    \end{flalign*}
    For an incompressible fluid with constant density $\rho \equiv \rho_0$,
    Definition~\ref{def:incompressible-fluid} implies $D_t \rho = 0$, so
    \begin{flalign*}
        \quad 0 = D_t \rho + \rho_0 \Div \vu
        \quad \Longrightarrow \quad \Div \vu = 0
        \quad \text{in } \Omega_B. &&
    \end{flalign*}
    Combining this with Lemma~\ref{lem:jacobian-determinant-derivative},
    \begin{flalign*}
        \quad D_t J_t = J_t \Div \vu, &&
    \end{flalign*}
    we obtain $D_t J_t = 0$, hence $J_t(\vx)$ is constant along material
    trajectories. With the natural choice $\La_0 = \mathrm{Id}$ we have
    $J_0 \equiv 1$, so incompressible flow is equivalent to
    \begin{flalign*}
        \quad & J_t(\vx) \equiv 1 \quad \text{and} \quad \Div \vu = 0. &&
    \end{flalign*}
    We refer to $\nabla \cdot \vu = \Div \vu = 0$ as the \emph{divergence-free condition} in $\Omega_B(t)$.
\end{remark}
By assumption blood is incompressible and newtonian, linear momentum balance follows from Newton's 2nd Law ($F = ma$)
for an infinitesimal fluid element $V(t) \subset \OmgBt$.
\begin{definition}
    \label{def:linear-momentum-balance}
    The linear momentum balance for a fluid element $V \in \OmgB$ under flow $\vu$ is
    \begin{flalign*}
        \quad & \frac{d}{dt} \int_{V(t)} \rho \, \vu \, dV \defined \int_{V(t)} \Divbf(\vT) \, dV + \int_{V(t)} \rho \, \vfb \, dV,
        \quad (\because ~ \rho \equiv \rho_0) &&
    \end{flalign*}
    where $\vT$ is the Cauchy stress tensor describing the fluids deformation, $\vfb$ may be some body force per unit mass $\units{\dfrac{cm}{s^2}}$.
\end{definition}
By applying the Reynolds Transport Theorem~\ref{thm:reynolds-transport} to~\ref{def:linear-momentum-balance},
we obtain the strong form of linear momentum balance in $\OmgB$:
\begin{flalign*}
    \quad & \int_{V(t)} \rho \, \pd_t \vu \, dV + \int_{\p V(t)} \rho \, \vu \, \inp{\vu}{\nhat} \, dS \defined \int_{V(t)} \Divbf(\vT) \, dV + \int_{V(t)} \rho \, \vfb \, dV, &&
\end{flalign*}
for any material volume $V(t) \subset \OmgBt$.
By the divergence theorem, we may rewrite the surface integral as a volume integral
\begin{flalign*}
    \quad & \int_{V(t)} \rho \, \pd_t \vu \, dV + \int_{V(t)} \Div(\rho \, \vu \otimes \vu) \, dV \defined \int_{V(t)} \Divbf(\vT) \, dV + \int_{V(t)} \rho \, \vfb \, dV, &&
\end{flalign*}
Since $V(t)$ is arbitrary, the integrands must be equal a.e. in $\OmgBt$, yielding differential form of linear momentum balance:
\begin{flalign*}
    \quad & \rho \, \pd_t \vu + \Div(\rho \, \vu \otimes \vu) \defined \Divbf(\vT) + \rho \, \vfb, \qquad \text{in} ~ \OmgBt &&
\end{flalign*}
By incompressibility assumption of blood fluid~\ref{def:incompressible-fluid} we have $\rho \equiv \rho_0$ constant in $\OmgBt$,
so the linear momentum balance simplifies to
\begin{flalign*}
    \quad & \rho \, \pd_t \vu + \rho \, \Div(\vu \otimes \vu) \defined \Divbf(\vT) + \rho \, \vfb, \qquad \text{in} ~ \OmgBt &&
\end{flalign*}
Using the vector calculus identity $\Div(\vu \otimes \vu) = (\nabla \cdot \vu) \, \vu + (\vu \cdot \nabla) \vu$ we have
\begin{flalign*}
    \quad \Div(\vu \otimes \vu) &= \Divbf\big(\vu \vu^\top\big) &&\\
    \quad &\equals \big(\nabla \cdot \vu\big) \, \vu + (\vu \cdot \nabla) \vu &&\\
    \quad &\equals 0 \cdot \vu + (\vu \cdot \nabla) \vu \quad (\because \Div \, \vu \equals 0 \text{ by remark~\ref{rmk:incompressibility-divergence-free}}) &&\\
    \quad \implies& \rho \, \pd_t \vu + \rho \, (\vu \cdot \nabla) \vu \defined \Divbf(\vT) + \rho \, \vfb, \qquad \text{in} ~ \OmgBt &&
\end{flalign*}
so the linear momentum balance~\ref{def:linear-momentum-balance} for incompressible fluid simplifies to the following form.
\begin{definition}
    \label{def:linear-momentum-balance-strong}
    The differential form of linear momentum balance for an incompressible fluid element $V \in \OmgB$ under flow $\vu$ is
    \begin{flalign*}
        \quad & \rho(\pd_t \vu + (\vu\cdot\nabla)\vu ) \defined \Divbf(\vT) + \rho \vfb, \qquad \text{in} ~ \OmgBt &&
    \end{flalign*}
\end{definition}

\medskip

The linear momentum balance equation above is not closed since the Cauchy stress tensor $\vT$ is
unknown. To close the system, we need a constitutive law relating $\vT$ to kinematic quantities such as
the velocity field $\vu$ and its derivatives.

\medskip

Let the rate-of-deformation tensor be defined as the symmetric
part of the velocity gradient, i.e.,
\begin{flalign*}
    \quad \vct{D}(\vu) \defined \tfrac{1}{2}\big(\nabla\vu+(\nabla\vu)^\top\big)
    \quad \text{s.t.} \quad \nabla\vu \defined \vect{
        \dfrac{\pd u_1}{\pd x} & \dfrac{\pd u_1}{\pd y} & \dfrac{\pd u_1}{\pd z} \\
        \dfrac{\pd u_2}{\pd x} & \dfrac{\pd u_2}{\pd y} & \dfrac{\pd u_2}{\pd z} \\
        \dfrac{\pd u_3}{\pd x} & \dfrac{\pd u_3}{\pd y} & \dfrac{\pd u_3}{\pd z}
    }. &&
\end{flalign*}
Note that $\vu \mapsto \vct{D}(\vu)$ captures spatial deformations of element $V$ in $\OmgB$ under flow $\vu$.
\begin{definition}
    \label{def:newtonian-fluid}
    A fluid is \emph{Newtonian} if its Cauchy stress tensor $\vT\;$ depends
    linearly on the rate-of-deformation tensor $\vct{D}(\vu)$.
\end{definition}
\begin{definition}
    \label{def:isotropic}
    A fluid is \emph{isotropic} if its constitutive response is independent of the coordinate system.
    Writing the Cauchy stress as $\vT\;=\vT\;(\vct{D})$,
    isotropy means that for every orthogonal rotator $\vct{Q}\in\mathrm{SO}(3)$,
    \begin{flalign*}
        \quad & \vct{Q}\, \vT(\vct{D})\,\vct{Q}^{\top} \defined \vT\;\!\big(\vct{Q}\,\vct{D}\,\vct{Q}^{\top}\big). &&
    \end{flalign*}
\end{definition}
Where $SO(3) \defined \{\vct{Q} \in \R^{3 \times 3} : \vct{Q}\,\vct{Q}^{\top} = \mathrm{I}, ~ \det(\vct{Q}) = 1\}$ is the special orthogonal group in $\R^3$.
\begin{remark}[Isotropy implication]
    \label{rmk:isotropy-implication}
    By \ref{def:isotropic}, the Cauchy stress $\vT$ of an isotropic fluid depends only on
    the invariants of $\vct{D}(\vu)$, i.e., the eigenvalues of $\vct{D}(\vu)$.
\end{remark}
\begin{remark}[Simple vs. complex fluids]
    In the terminology of continuum mechanics, a \emph{simple} fluid is one whose
    constitutive response at a point depends only on the instantaneous values of
    kinematic quantities (such as the rate-of-deformation tensor $\vct{D}(\vu)$),
    and not on their past history. The Newtonian, isotropic model in
    Definition~\ref{def:fluidsmodel} is a prototypical simple fluid.
    More complex models (e.g.\ viscoelastic fluids) incorporate memory effects
    so that the stress depends on the deformation history.
\end{remark}
\begin{remark}[Blood isotropy justification]
    \label{rmk:blood-isotropy-justification}
    Blood is often modeled as an isotropic fluid since its microstructure (RBCs, WBCs, platelets)
    are suspended in plasma and distributed uniformly in all directions at macroscopic scales.
    This uniform distribution leads to isotropic mechanical properties, meaning blood's response
    to deformation is independent of direction.
    \red{[\cite{book1}, sec. 3.1]} (Check source)
\end{remark}
\medskip
\begin{definition}[Newtonian, isotropic constitutive law]
    \label{def:fluidsmodel}
    For a Newtonian, isotropic fluid the Cauchy stress is
    \begin{flalign*}
        \quad & \vT \equals -\,p\,\mathrm{I} + 2\,\eta\,\vct{D}(\vu) + \lambda\,\Div(\vu)\,\mathrm{I}, &&
    \end{flalign*}
    where the bulk viscosity $\lambda \in \R^+$ is a material parameter quantifying the fluid's resistance
    to uniform compression.
\end{definition}
So newtonian isotropic fluids at rest (quiescent state $\vu\equiv\vct{0}$) sustains only hydrostatic stress:
\begin{flalign*}
    \quad \implies \vT \equals -\,p\,\mathrm{I} \quad \text{when} ~ \vu \equiv \vct{0}. &&
\end{flalign*}
Consequently, the constitutive law for newtonian fluids~\ref{def:fluidsmodel} simplifies under incompressibity assumption,
because then $\Div(\vu) \equals 0$ in $\OmgB$ (refer to~\ref{rmk:incompressibility-divergence-free}).
\begin{definition}[Incompressible stress tensor]
    \label{def:improssible-stress-tensor}
    The Cauchy stress~\ref{def:fluidsmodel} simplifies to
    \begin{flalign*}
        \quad \vT ~ = ~ -\,p\,\mathrm{I} + 2\,\eta\,\vct{D}(\vu). &&
    \end{flalign*}
\end{definition}
\begin{remark}[Dynamic vs. Kinematic Viscosity]
    In our model assumptions, \emph{dynamic viscosity} $\eta \in \R^+$ and  \emph{kinematic viscosity} $\mu \in \R^+$
    relate as
    \begin{flalign*}
        \quad \mu\; \defined\; \frac{\eta}{\rho} ~ = ~  \frac{\eta}{\rho_0} \; \in \; \R^+. &&
    \end{flalign*}
    Here $\eta$ quantifies the internal resistance of blood to shear deformation, i.e., $\eta \defined \dfrac{\tau}{\dot{\gamma}}$,
    with units $\units{Pa \cdot s}$. Moreover $\mu$ adjusts $\eta$ by the density $\rho$, capturing the viscous diffusion of momentum per unit mass, with units $\units{\frac{cm^2}{s}}$.
    Intuitively, $\eta$ measures how "thick" or "sticky" the fluid is, while $\mu$ measures how quickly momentum diffuses through the fluid due to viscosity.
\end{remark}

\medskip

\begin{remark}[Newtonian Blood Justification]
    \label{rmk:newtonian-justification}
    When diameter $d$ and hematocrit effects are needed, one may use a Non-Newtonian
    model with relative viscosity $\eta_r(H,d)$ that scales an absolute baseline $\eta$:
    \begin{flalign*}
        \quad \eta_{\mathrm{eff}} \defined \eta_r(H,d)\,\eta \quad \text{(effective viscosity)}&&
    \end{flalign*}
    An empirical fit from \cite{bviscosity}
    \begin{flalign*}
        \quad \eta_r \defined 1 + (\eta_{0.45}-1)\,
        \frac{(1-H)^{C}-1}{(1-0.45)^{C}-1}
        \text{ s.t. }
        \begin{cases}
            \eta_{0.45} = 6\,e^{-0.085\,d} + 3.2 - 2.44\,e^{-0.06\,d^{0.645}},\\[3pt]
            C = \big(0.8 + e^{-0.075\,d}\big)\!\left(\dfrac{1}{1 + 10^{-11}\,d^{12}} - 1\right)
                + \dfrac{1}{1 + 10^{-11}\,d^{12}},
        \end{cases} &&
    \end{flalign*}
    where $d \defined 2R/(1.0 \mu m)$ is the (scaled) vessel diameter.
    In large vessels, $\eta_r$ is often constant, justifying the Newtonian assumption.
    [\cite{book1}, sec. 3.1]
\end{remark}
\medskip
When modeling non-Newtonian effects (when $\eta \neq \text{ constant}$), the kinematic viscosity
$\mu(\cdot)$ is often chosen by Carreau model \cite{hemodynamics}
\begin{flalign*}
    \quad 2\,\mu(|\vct{D}|^2) = \eta_\infty + (\eta_0 - \eta_\infty) \cdot \big(1 + \kappa |\vct{D}|^2\big). &&
\end{flalign*}
Where $\eta_0$ and $~\eta_\infty$ are chosen to be the viscosity for very small
and very large shear rates, resp., and $\kappa \in \R^+$ and $n \in (-0.5, 0)$ are model parameters.
According to [\cite{book1}, pg. $38$], we often set
\begin{flalign*}
    \quad \eta_0 = 65.7 \cdot 10^{-3} ~Pa \cdot s, ~ \eta_\infty = 4.45\cdot10^{-3} ~Pa \cdot s,
        \kappa = 212.2 ~s^2, ~ \text{ and } n = -0.325 &&
\end{flalign*}
In the Newtonian case, we choose $\eta = \eta_\infty$ which allows us to determine $\mu$
as $\mu(|\vct{D}|^2) = \eta$.
