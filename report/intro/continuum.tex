W.a.t. simulate blood flow in a time-dependent fluid domain $\OmgB \subset \R^{N+1}$;
we'll take $N \equivto 3$, the natural setting for the derivations that follow.
For each $t \in I_T \defined [0, T] \subset \R$, with $T > 0$, we define
\begin{flalign*}
    \quad \OmgBt &:= \{\vx \in \R^N : \vx \text{ lies inside the vessel at time } t\}. &&
\end{flalign*}
Formally, the fluid region is a time-dependent family of open sets
$\{\OmgBt\}_{\forallin{t}{I_T}} \subset \R^N$ occupied by blood at time $t$.
Let $\OmgB(0)$ r.t. the \emph{reference} configuration and $\OmgBt$
r.t. the \emph{current} configuration.
The fluids computational domain r.t. all possible configurations of $\OmgBt$, i.e.,
the spatial-temporal lumen
\begin{flalign*}
    \quad & \Omg_B \defined I_T \times \OmgBt \equivto \{ (t, \vx) \in \R^{N+1} ~:~ x \in \Omgt, ~ t \in I_T \}. &&
\end{flalign*}

\medskip

For each material particle $\vxi \in \OmgB(0)$, let's assume $\vu(t, \vxi(t))$ is the flow
feild determined from a system of $PDEs$ governing $\vxi$'s motion in $\OmgB$.
Our aim here is to derive such PDEs -- the Navier-Stokes system.

\medskip

\subsubsection*{Blood Dynamics}
% Coordinate defs
The spatial configuration $\OmgBt$ is determined by a Lagrangian mapping
\begin{flalign*}
    \quad & \La_t: \OmgB(0) \mapsto \OmgBt, \quad \vxi \mapsto \vx(t, \vxi) \equals \La_t(\vxi) &&
\end{flalign*}
Let's impose that $\La_t$ is a continuous \red{bijection} $\forall t \in I_T$ in $\OmgBBar$, then
$\exists ~ \inv{\La_t}: \ol{\OmgBt} \mapsto \ol{\OmgB(0)}$ s.t. $\vxi = \inv{\La_t}(\vx)$.
I.e., there is a one-to-one correspondence between material fluid particles in the reference
configuration and spatial points in the current configuration at each time $t$.
The variables $(t, \vx)$ and $(t, \vxi)$ are refered to as the Eulerian and Lagrangian coordinates, resp..
\begin{remark}[Eulerian vs. Lagrangian]
    \label{rmk:eulerian-lagrangian}
    Informally, the Eulerian approach focuses our attention to $\vx \in \OmgBt$, namely,
    some fluid particle located at $\vx$ at a particular time $t$.
    Whereas, the Lagrangian approach tracks an individual fluid particle $\vxi \in \OmgB(0)$
    along it's trajectory $\vT_{\vxi} \defined \{ (t, \vxi) : t \in I_T \}$ as it moves
    through space and time along a characteristic curve defined by the velocity field $\vu$.

    \medskip

    For $\phi \in C^k(\OmgB)$ s.t. $(t, \vx) \mapsto \phi(t, \vx)$ in the Eulerian variables,
    we'll let $\hat{\phi}(t, \xi) \defined \phi(t, \vx)$ s.t. $\vx = \La_t(\vxi)$. Or equivalently,
    we'll let $\hat{\phi} \equals \phi \circ \La_t$ and it follows that $\phi \equals \hat{\phi} \circ \inv{\La_t}$.
\end{remark}

\medskip

% Principle kinematic quantity: velocity def. via Lagrangian Reference Frame
\begin{definition}
    \label{def:lagrangian-velocity}
    Let the Lagrangian velocity feild be defined as as the time derivative
    along the trajectory $T_{\vxi}$ of a flulid particle $\vxi \in \OmgB(0)$
    \begin{flalign*}
        \quad \hat{\vu} \defined \pd_t \vx(t, \vxi) \mapsto \hat{\vu}(t, \vxi) \quad \text{ in } ~ \OmgBt &&
    \end{flalign*}
\end{definition}
A change of corrdinates in~\ref{def:lagrangian-velocity}  yeilds the velocity feild $\vu$ in the Eulerian frame, i.e.,
\begin{flalign*}
    \quad \vu \equals \hat{\vu} \circ \inv{\La_t} \quad \iff \quad \vu(t, \vx) \equals \hat{\vu}(t, \inv{\La_t(\vx)}) &&
\end{flalign*}
Then if the velocity feild $\hat{\vu}$ and the reference configuration $\Omg(0)$ are known, we arrive at the
Cauchy problem
\begin{flalign*}
    \quad \begin{cases*}
        \pd_t{\vx}(t, \vxi) \equals \hat{\vu}(t, \vxi), \forallin{t}{I_T} \\
        \vx(0, \xi) \equals \vxi.
    \end{cases*} &&
\end{flalign*}


\medskip

\subsubsection*{Material Derivatives}
\begin{definition}
    \label{def:material-derivative}
    Let $\phi \in C^k(\OmgB)$ and $\hat{\phi} \equals \phi \circ \La_t$, then the \emph{material derivative}
    $D_t \phi$ is the derivative of $\phi$ w.r.t. $t$ in the Lagrangian frame expressed as a function in the Eulerian frame, i.e.,
    \begin{flalign*}
        \quad D_t \big( \phi(t, \vx) \big) &\defined \pd_t \hat{\phi}(t, \vxi), \quad \text{ s.t. } \vxi \equals \inv{\La_t}(\vx) &&\\
            & \equals \dfrac{d}{dt}\phi(t, \vx(t, \vxi)), \quad \forallin{\vxi}{\OmgB(0)}. &&
    \end{flalign*}
\end{definition}
By the multivariable chain rule, $D_t \phi$ in~\ref{def:material-derivative} may be written as
\begin{flalign*}
    \quad D_t \big( \phi(t, \vx) \big) \equals \partial_t \phi + \inp{\nabla \phi}{\vu}_{\R^N} \quad \text{ in } \OmgB &&
\end{flalign*}
where we may see that the material derivative $D_t \phi$ measures the rate of variation of $\phi$ along trajectory $T_{\vxi}$.
The literature also r.t. $D_t$ as the \emph{substantial derivative}, \emph{advective derivative}, \emph{lagrangian derivative}, or
\emph{convective derivative}.

\medskip

Now differentiating $\hat{\vu}$ w.r.t. $t$ yeilds the acceleration vector feild $\vahat$ in the Lagrangian frame
\begin{flalign*}
    \quad \vahat &\defined \pd_t \hat{\vu}(t, \vxi) = \pd_t^2 \vx(t, \vxi) \quad  \forallin{\vxi}{\OmgB(0)} &&\\
        &\iff \va \equals D_t \vu \equals \partial_t \vu + (\vu \cdot \nabla) \vu &&\\
        &\iff \va(t, \vx) \equals \pd_t \vu(t, \vx) + \sum_{i=1}^{N} u_i(t, \vx) \pd_{x_i} \vu(t, \vx) \quad \forallin{(t, \vx)}{\OmgB}. &&
\end{flalign*}
The acceleration $\va$ measures the rate of change of velocity $\vu$ along the trajectory $T_{\vxi}$, i.e.,
we applied the material derivative to the velocity feild $\vu$ component-wise.
In other words, $\va$ is the material derivative~\ref{def:material-derivative} applied component-wise to the velocity field $\vu$.

(see pg. 19 \cite{book2} for more details and examples)

\medskip

The defermation gradient $\vFhat(t)$ in the Lagrangian frame is defined as
\begin{flalign*}
    \quad \vFhat(t) &\defined \nabla_{\vxi} \La_t(\vxi) \quad \forallin{\vxi}{\OmgB(0)} &&\\
        &\iff \vFhat(t)(\vx) \equals \nabla_{\vxi} \La_t(\inv{\La_t}(\vx)) \equals \pd_{\vxi} \vx \quad \forallin{(t, \vx)}{\OmgB} &&
\end{flalign*}
The deformation gradient $\vFhat: \OmgB(0) \to \R^{N \times N}$ measures the local spatial deformation of fluid
particles in the reference configuration as they move to the current configuration under $\La_t$ along trajectory $T_{\vxi}$.
To ensure $\vFhat(t)$ is invertible, it is sufficient to note that $\La_t$ is a continuous bijection in $\OmgBBar$ with continuous inverse
$\inv{\La_t}$. Thus by the inverse function theorem $\vFhat(t)$ is invertible $\forallin{\vxi}{\OmgB(0)}$. So we may define the Jacobian determinant $\hat{J}(t)$ in the Lagrangian frame as
\begin{flalign*}
    \quad \hat{J}(t)(\vxi) &\defined \det\big(\vFhat(t)(\vxi)\big) \quad \forallin{\vxi}{\OmgB(0)} &&\\
        &\iff J(t)(\vx) \equals \det\big(\vFhat(t)(\inv{\La_t}(\vx))\big) \quad \forallin{(t, \vx)}{\OmgB} &&
\end{flalign*}
The Jacobian determinant $J(t)(\vx)$ measures the local change in volume of an infinitesimal fluid element
as it moves from the reference configuration to the current configuration at time $t$ along trajectory $T_{\vxi}$.
We'll later see that $\partial_t J$ relates to the fluids divergence $\Div(\vu)$.

\medskip

\subsection{NS-derivation}
In the following derivations we let $N \equiv 3$ so that $\OmgBt \subset \R^3$. We assume blood fluid is
a continuum that deforms continously, i.e., at every point $\vx \in \OmgBt$ and time $t \in I_T$, the
blood's kinematic quantities are described by sufficiently smooth fields. At microscopic scales this
continuum hypothesis breaks down, since matter is a discrete collection of molecules, but at macroscopic
scales empirical evidence suggests such models remain accurate.

\medskip

For each $t \in I_T$, herein assume $\OmgBt$ is simply connected and boundary $\pOmgB(t) \in C^1$,
so, any closed curve in $\OmgBt$ can be continuously contracted to a point within $\OmgBt$; such an
assumption is reasonable in a health vessel with a smooth wall. This regularity allows us to globally
define geometric quantities such as surface differential operators on the fluids boundary $\pOmgB$, then
we may apply stokes thoerem and Green's identities on the $C^1$ boundaries, useful for obtaining weak
integro-differential and variational formulations.

\medskip

\begin{remark}
    A weaker regularity condition is that $\pOmgB$ is Lipschitz. Here we must
    rely on standard trace theorems for Soblev spaces (e.g.\ $H^1(\OmgBt)$). In particular,
    the outward unit normal is then defined a.e. on $\pOmgB(t)$, so we can meaningfully speak of
    normal and tangential components of vector feilds on the boundary.
\end{remark}

\medskip

% Reynolds Transport
Let $V(t)$ denote a measurable material volume at time $t$ that moves with the fluid flow, i.e.,
\begin{flalign*}
    \quad V(t) &\subset \OmgBt, \quad \text{for each } t \in I_T &&
\end{flalign*}
Equivalently, $V(t)$ is the image of some reference volume $V(0) \subset \OmgB(0)$ under the
Lagrangian mapping $\La_t$, i.e.,
\begin{flalign*}
    \quad V(t) & = \La_t\big(V(0)\big). &&
\end{flalign*}
\begin{theorem}[Reynolds Transport Theorem]
    \label{thm:reynolds-transport}
    Let $V(0) \subset \OmgB(0)$ be a material volume in the reference configuration,
    and $V(t) \subset \OmgBt$ its image in the current configuration under $\La_t$. For any
    sufficiently smooth scalar field $\phi: \OmgB \to \R$,
    \begin{flalign*}
        \quad & \frac{d}{dt} \int_{V(t)} \phi(t, \vx) \, dV \;=\; \int_{V(t)} \partial_t \phi(t, \vx) \, dV
            + \int_{\p V(t)} \phi(t, \vx) \, \inp{\vu}{\nhat} \, dS, &&
    \end{flalign*}
    where $\vu$ is the fluid velocity field and $\nhat$ the outward unit normal on $\p V(t)$.
\end{theorem}
\begin{proof}
    Ref~\cite{book2}, theorem 2.2. pg. 21.
\end{proof}

\medskip
Let blood's velocity, thermodynamic pressure, and density fields be
\begin{flalign*}
    \quad & \vu: \OmgB \to \R^3,\;
        (t, x, y, z) \mapsto (u_1(t, x, y, z),\, u_2(t, x, y, z),\, u_3(t, x, y, z))^\top,
        \quad \bigg[\; \frac{m}{s} \;\bigg] && \\
    \quad & p: \OmgB \to \R^+, \;
        (t, x, y, z) \mapsto p(t, x, y, z),
        \quad \bigg[\; \mathrm{Pa} \equiv \frac{N}{m^2} \;\bigg] && \\
    \quad & \rho: \OmgB \to \R^+,\;
        (t, x, y, z) \mapsto \rho(t, x, y, z),
        \quad \bigg[\; \frac{kg}{m^3} \;\bigg]. &&
\end{flalign*}

\medskip
One may make a distinction between incompressible fluids and incompressible flows.
\begin{definition}[Incompressibile fluid]
    \label{def:incompressible-fluid}
    An element $V \in \OmgB$ with constant density $\rho(\vx, t) ~ \forallin{t, ~\vx}{I_T \times V(t)}$
    is an \emph{incompressible fluid}.
\end{definition}
\begin{definition}[Incompressible flow]
\label{def:incompressible-flow}
    An element $V \in \OmgB$ subject to $\vu$ with constant rate of material density change
    (in both space and time so that $D_t \rho = 0$ for all $t \in I_T$) undergoes an \emph{incompressible flow} in $\OmgB$.
\end{definition}
I.m.b.s. for all material elements $V \in \Omg$, \ref{def:incompressible-fluid} implies \ref{def:incompressible-flow}.
Note, the converse is not generally true: an incompressible flow ($D_t\rho=0$) only preserves density
along particle paths and allows $\rho=\rho(\vx)$ to vary spatially; but, particularly, in our case the
initial density is distributed uniformly in space, and here incompressible flow also implies incompressible fluid.
By the incompressibile assumption of blood fluid, the material density $\rho(\vx, t) \equiv \rho_0$
for all $\vx \in \Omgt$ and $t \in I_T$. Constant blood density implies $D_t \rho = 0$, i.e., incompressibile flow in $\Omgt$
\begin{flalign*}
    \quad & \partial_t \rho + \nabla \cdot (\rho \vu) = 0 \quad \impliedby \quad \rho \text{ constant in } \OmgBt &&
\end{flalign*}
By assumption blood is Newtonian, linear momentum balance follows from Newton's 2nd Law ($F = ma$) as
\begin{flalign*}
    \quad \rho(\partial_t \vu + (\vu\cdot\nabla)\vu ) \;=\; \Divbf(\vT) + \rho \vfb, \qquad \text{in} ~ \OmgBt &&
\end{flalign*}
where $\vT$ is the Cauchy stress tensor describing the fluids deformation, $\vfb$ may be some body force per unit mass $\big[\dfrac{m}{s^2}\big]$.
Let the rate-of-deformation tensor be defined as the symmetric
part of the velocity gradient, i.e.,
\begin{flalign*}
    \quad \vct{D}(\vu) \defined \tfrac{1}{2}\big(\nabla\vu+(\nabla\vu)^\top\big)
    \quad \text{s.t.} \quad \nabla\vu \defined \vect{
        \dfrac{\partial u_1}{\partial x} & \dfrac{\partial u_1}{\partial y} & \dfrac{\partial u_1}{\partial z} \\
        \dfrac{\partial u_2}{\partial x} & \dfrac{\partial u_2}{\partial y} & \dfrac{\partial u_2}{\partial z} \\
        \dfrac{\partial u_3}{\partial x} & \dfrac{\partial u_3}{\partial y} & \dfrac{\partial u_3}{\partial z}
    }. &&
\end{flalign*}
Note that $\vu \mapsto \vct{D}(\vu)$ captures spatial deformations of element $V$ in $\OmgB$ under flow $\vu$.
\begin{definition}
    \label{def:newtonian-fluid}
    A fluid is \emph{Newtonian} if its Cauchy stress tensor $\vT\;$ depends
    linearly on the rate-of-deformation tensor $\vct{D}(\vu)$.
\end{definition}
\begin{definition}
    \label{def:isotropic}
    A fluid is \emph{isotropic} if its constitutive response is independent of the coordinate system.
    Writing the Cauchy stress as $\vT\;=\vT\;(\vct{D})$,
    isotropy means that for every orthogonal rotator $\vct{Q}\in\mathrm{SO}(3)$,
    \begin{flalign*}
        \quad & \vct{Q}\, \vT(\vct{D})\,\vct{Q}^{\top} \;=\; \vT\;\!\big(\vct{Q}\,\vct{D}\,\vct{Q}^{\top}\big). &&
    \end{flalign*}
\end{definition}
\begin{definition}[Newtonian, isotropic constitutive law]
    \label{def:fluidsmodel}
    For a Newtonian, isotropic fluid the Cauchy stress is
    \begin{flalign*}
        \quad & \vT = -\,p\,\mathrm{I} + 2\,\eta\,\vct{D}(\vu) + \lambda\,\Div(\vu)\,\mathrm{I}, &&
    \end{flalign*}
    where $\eta>0$ is the dynamic (shear) viscosity and $\lambda$ is the bulk viscosity.
\end{definition}
So an isotropic fluid at rest (quiescent state $\vu\equiv\vct{0}$) sustains only hydrostatic stress:
\begin{flalign*}
    \quad \implies \vT \;=\; -\,p\,\mathrm{I} \quad \text{when} ~ \vu \equiv \vct{0}. &&
\end{flalign*}
\begin{remark}[Divergence-free condition]
    \label{rmk:incompressibility-divergence-free}
    By assuming blood is incompressible fluid~\ref{def:incompressible-fluid} and flow~\ref{def:incompressible-flow} we have
    \begin{flalign*}
        \quad & D_t \rho = 0 &&\\
        \quad & \iff \partial_t \rho + \Div(\rho \vu) = 0 &&\\
        \quad & \iff \Div(\rho_0 \vu) = 0 \quad (\because ~ \rho=\rho_0)&&\\
        \quad & \iff \nabla \cdot (\rho_0 \vu) = 0 &&\\
        \quad & \iff \inp{\rho_0 \vu}{\nabla} = 0 &&\\
        \quad & \iff \rho_0 \inp{\vu}{\nabla} = 0 &&\\
        \quad & \iff \inp{\vu}{\nabla} = 0 &&\\
        \quad & \iff \nabla \cdot \vu = 0, &&
    \end{flalign*}
    and we r.t. $\nabla \cdot \vu = \Div(\vu) = 0$ as the \emph{divergence-free condition} of $\vu$
    in $\Omgt$.
\end{remark}
Consequently, the constitutive law\ref{def:fluidsmodel} simplifies as follows.
\begin{definition}[Incompressible stress tensor]
    \label{def:improssible-stress-tensor}
    The Cauchy stress~\ref{def:fluidsmodel} simplifies to
    \begin{flalign*}
        \quad \vT ~ = ~ -\,p\,\mathrm{I} + 2\,\eta\,\vct{D}(\vu). &&
    \end{flalign*}
\end{definition}
\begin{remark}[Dynamic vs. Kinematic Viscosity]
    In our model assumptions, \emph{dynamic viscosity} $\eta \in \R^+$ and  \emph{kinematic viscosity} $\mu \in \R^+$
    relate as
    \begin{flalign*}
        \quad \mu\; \defined\; \frac{\eta}{\rho} ~ = ~  \frac{\eta}{\rho_0} \; \in \; \R^+. &&
    \end{flalign*}
    Here $\eta$ quantifies the internal resistance of blood to shear deformation, i.e., $\eta \defined \dfrac{\tau}{\dot{\gamma}}$,
    with units $[Pa \cdot s]$. Moreover $\mu$ adjusts $\eta$ by the density $\rho$, capturing the viscous diffusion of momentum per unit mass, with units $\big[\frac{m^2}{s}\big]$.
    Intuitively, $\eta$ measures how "thick" or "sticky" the fluid is, while $\mu$ measures how quickly momentum diffuses through the fluid due to viscosity.
\end{remark}
\begin{remark}[Newtonian Blood Justification]
    \label{rmk:newtonian-justification}
    When diameter $d$ and hematocrit effects are needed, one may use a Non-Newtonian
    model with relative viscosity $\eta_r(H,d)$ that scales an absolute baseline $\eta$:
    \begin{flalign*}
        \quad \eta_{\mathrm{eff}} \;=\; \eta_r(H,d)\,\eta \quad \text{(effective viscosity)}&&
    \end{flalign*}
    An empirical fit from \cite{bviscosity}
    \begin{flalign*}
        \quad \eta_r \;=\; 1 + (\eta_{0.45}-1)\,
        \frac{(1-H)^{C}-1}{(1-0.45)^{C}-1}
        \text{ s.t. }
        \begin{cases}
            \eta_{0.45} = 6\,e^{-0.085\,d} + 3.2 - 2.44\,e^{-0.06\,d^{0.645}},\\[3pt]
            C = \big(0.8 + e^{-0.075\,d}\big)\!\left(\dfrac{1}{1 + 10^{-11}\,d^{12}} - 1\right)
                + \dfrac{1}{1 + 10^{-11}\,d^{12}},
        \end{cases} &&
    \end{flalign*}
    where $d \defined 2R/(1.0 \mu m)$ is the (scaled) vessel diameter.
    In large vessels, $\eta_r$ is often constant, justifying the Newtonian assumption.
    [\cite{book1}, sec. 3.1]
\end{remark}

\red{A fluid that deforms independent of time is simple, such fluids deformation and rate of
deformation aren't subject to material memory effects (also r.t. as viscoelastic effects).
The contrary are complex fluids, their deformation is subject to both viscous and elastic
characteristics.} (Out of place)
