W.a.t. simulate bloods flow in a time-dependent fluid domain $\OmgBt \subset\R^{N-1}$, where
\begin{flalign*}
    \quad \OmgBt &:= \{\vx ~:~ \vx \text{ lies inside the vessel at time } t \} &&
\end{flalign*}
for all $t \in I_T$ s.t.
\begin{flalign*}
    \quad I_T \defined [0, T], \quad T > 0. &&
\end{flalign*}
We r.t. the fluids spatial-temporal domain as the fluids \emph{computational domain}
\begin{flalign*}
    \quad & \Omg \;:=\; \{ (x, t) \in \R^{N+1} ~:~ x \in \Omg(t), ~ t \in I_T \}. &&
\end{flalign*}

\medskip

The fluid region is modeled as a time-dependent family of open sets
$\{\OmgBt\}_{t \in I_T} \subset \R^3$ occupied by blood at time $t$.
For each $t \in I_T$, we assume that $\OmgBt$ is simply connected and has
Lipschitz continuous boundary. Thus any closed curve in $\OmgBt$ can be
continuously contracted to a point within $\OmgBt$, and standard results
from potential theory (e.g.\ Poincaré-type lemmas) apply when introducing
scalar potentials for irrotational vector fields. Under these regularity assumptions
on $\pOmgB$, standard trace theorems hold for Sobolev spaces (e.g.\ $H^1(\OmgBt)$).
In particular, the outward unit normal is defined a.e. on $\pOmgB$, so
we can meaningfully speak of normal and tangential components of vector fields on the boundary.

\medskip

\begin{remark}
    Often we may impose stronger conditions on $\pOmgB$, e.g., require that the
    boundary be a $C^1$ closed surface. This is reasonable since the vessel wall
    is typically smooth in healthy vessels. The additional regularity allows us
    to globally define geometric quantities such as curvature and surface differential
    operators on the fluids boundary $\pOmgB$.
\end{remark}

\medskip

Using the natural coordinate system, let $N \equiv 4$ so that $\OmgBt \in \R^3$.
We assume blood fluid is modeled as a continuum that deforms continously, i.e.,
at every point $\vx \in \OmgBt$ and time $t \in I_T$, the blood's kinematic quantities
are described by sufficiently smooth fields. At microscopic scales the continuum hypothesis
breaks down, since matter is a discrete collection of molecules, but at macroscopic scales
empirical evidence suggests such models remain accurate. Let blood's s.s. velocity,
thermodynamic pressure, and density fields be
\begin{flalign*}
    \quad & \vu: \OmgBt \to \R^3,\;
        (x, y, z) \mapsto (u_1(x, y, z),\, u_2(x, y, z),\, u_3(x, y, z))^\top,
        \quad \bigg[\; \frac{m}{s} \;\bigg] && \\
    \quad & p: \OmgBt \to \R,\;
        (x, y, z) \mapsto p(x, y, z),
        \quad \bigg[\; \mathrm{Pa} \equiv \frac{N}{m^2} \;\bigg] && \\
    \quad & \rho: \OmgBt \to \R^+,\;
        (x, y, z) \mapsto \rho(x, y, z),
        \quad \bigg[\; \frac{kg}{m^3} \;\bigg]. &&
\end{flalign*}

\medskip

A fluid that deforms indepenent of time is simple, such fluids deformation and rate of
deformation aren't subject to material memory effects (also r.t. as viscoelastic effects).
The contrary are complex fluids, their deformation is subject to both viscous and elastic
characteristics.

\begin{definition}
    \label{def:material_derivative}
    Let
    \begin{flalign*}
        \quad D_t\phi \;:=\; \partial_t \phi + \vu\!\cdot\!\Div(\phi), &&
    \end{flalign*}
    be the material derivative $D_t \phi$ measuring the instantanous rate of change of
    quantity $\phi$ moving with velocity field $\vu$. The literature also refers to $D_t$ as the
    \emph{substantial derivative}, \emph{advective derivative}, \emph{lagrangian derivative}, or \emph{convective derivative}.
\end{definition}
For $\phi$ subject to the flow $\vu$, note that
\begin{flalign*}
    \quad & D_t \phi(\vx, t) \;=\; \lim_{\Delta t \to 0} \frac{\phi(\vx + \vu(\vx, t)\, \Delta t, t + \Delta t) - \phi(\vx, t)}{\Delta t}, &&
\end{flalign*}
i.e., $D_t \phi$ is the instantanous rate of change of quantity $\phi$ if a material fluid element $V(t)_{\vx}$
positioned at $\vx \in \Omg(t)$ and moving with velocity $\vu(x, t)$.
By the incompressibile assumption of blood, the material density of $V(t)_{\vx}$ is $\rho(\vx, t) \equiv \rho_0$
for all $\vx \in \Omg(t)$ and $t \in I_T$. Implying $D_t \rho = 0$, so mass conservation in $\Omg(t)$ reads
\begin{flalign*}
    \quad & \partial_t \rho + \Div(\rho \vu) = 0 \quad \impliedby \text{incompressibility assumption} &&
\end{flalign*}
By assumption blood is Newtonian, linear momentum balance follows from Newton's 2nd Law ($F = ma$) as
\begin{flalign*}
    \quad \rho(\partial_t \vu + (\vu\cdot\nabla)\vu ) \;=\; \Divbf(\vT) + \rho \vfb, \qquad \text{in} ~ \Omg(t) &&
\end{flalign*}
where $\vT$ is the Cauchy stress tensor describing the fluids deformation, $\vfb$ may be some body force per unit mass $\big[\dfrac{m}{s^2}\big]$.
Note that $\Divbf$ acts rowise on $\vT$, implying $\exists$ matrix
$A ~ : ~ \vT \vu = \vT ~\because~ \text{blood is a simple fluid}$.
Let the rate-of-deformation tensor be defined as the symmetric
part of the velocity gradient, i.e.,
\begin{flalign*}
    \quad \vct{D}(\vu) := \tfrac{1}{2}\big(\nabla\vu+(\nabla\vu)^\top\big)
    \quad \text{s.t.} \quad \nabla\vu := \vect{
        \dfrac{\partial u_1}{\partial x} & \dfrac{\partial u_1}{\partial y} & \dfrac{\partial u_1}{\partial z} \\
        \dfrac{\partial u_2}{\partial x} & \dfrac{\partial u_2}{\partial y} & \dfrac{\partial u_2}{\partial z} \\
        \dfrac{\partial u_3}{\partial x} & \dfrac{\partial u_3}{\partial y} & \dfrac{\partial u_3}{\partial z}
    }. &&
\end{flalign*}
Note that $\vu \mapsto \vct{D}(\vu)$ captures spatial deformations of element $V_{\vx}$ in $\Omg$ under flow $\vu$.
\begin{definition}
    \label{def:newtonian-fluid}
    A fluid is \emph{Newtonian} if its Cauchy stress tensor $\vT\;$ depends
    linearly on the rate-of-deformation tensor $\vct{D}(\vu)$.
\end{definition}
\begin{definition}
    \label{def:isotropic}
    A fluid is \emph{isotropic} if its constitutive response is independent of the coordinate system.
    Writing the Cauchy stress as $\vT\;=\vT\;(\vct{D})$,
    isotropy means that for every orthogonal rotator $\vct{Q}\in\mathrm{SO}(3)$,
    \begin{flalign*}
        \quad & \vct{Q}\, \vT(\vct{D})\,\vct{Q}^{\top} \;=\; \vT\;\!\big(\vct{Q}\,\vct{D}\,\vct{Q}^{\top}\big). &&
    \end{flalign*}
\end{definition}
\begin{definition}[Newtonian, isotropic constitutive law]
    \label{def:fluidsmodel}
    For a Newtonian, isotropic fluid the Cauchy stress is
    \begin{flalign*}
        \quad & \vT = -\,p\,\mathrm{I} + 2\,\eta\,\vct{D}(\vu) + \lambda\,\Div(\vu)\,\mathrm{I}, &&
    \end{flalign*}
    where $\eta>0$ is the dynamic (shear) viscosity and $\lambda$ the bulk viscosity.
\end{definition}
So an isotropic fluid at rest (quiescent state $\vu\equiv\vct{0}$) sustains only hydrostatic stress:
\begin{flalign*}
    \quad \implies \vT \;=\; -\,p\,\mathrm{I} \quad \text{when} ~ \vu \equiv \vct{0}. &&
\end{flalign*}
One may make a distinction between incompressible fluids and incompressible flows.
feild $\vu$.
\begin{definition}[Incompressibile fluid]
    \label{def:incompressible-fluid}
    An element $V \in \Omg$ with constant density $\rho(\vx, t)$ for all $\vx \in V(t), t \in I_T$
    is an \emph{incompressible fluid}.
\end{definition}
\begin{definition}[Incompressible flow]
\label{def:incompressible-flow}
    An element $V \in \Omg$ subject to $\vu$ with constant rate of material density change
    (in both space and time so that $D_t \rho = 0$ for all $t \in I_T$) undergoes an \emph{incompressible flow} in $\Omg$.
\end{definition}
I.m.b.s. for all material elements $V \in \Omg$, \ref{def:incompressible-fluid} implies \ref{def:incompressible-flow}.
Note, the converse is not generally true: an incompressible flow ($D_t\rho=0$) only preserves density
along particle paths and allows $\rho=\rho(\vx)$ to vary spatially; but, particularly, in our case the
initial density is distributed uniformly in space, and here incompressible flow implies incompressible fluid.
\begin{remark}[Divergence-free condition]
    \label{rmk:incompressibility-divergence-free}
    By assuming blood is incompressible fluid and flow, we have
    \begin{flalign*}
        \quad & D_t \rho = 0 &&\\
        \quad & \iff \partial_t \rho + \Div(\rho \vu) = 0 &&\\
        \quad & \iff \Div(\rho_0 \vu) = 0 \quad (\because ~ \rho=\rho_0)&&\\
        \quad & \iff \nabla \cdot (\rho_0 \vu) = 0 &&\\
        \quad & \iff \langle\rho_0 \vu, \nabla \rangle = 0 &&\\
        \quad & \iff \rho_0 \langle \vu, \nabla \rangle = 0 &&\\
        \quad & \iff\langle \vu, \nabla \rangle = 0 &&\\
        \quad & \iff \nabla \cdot \vu = 0, &&
    \end{flalign*}
    and we r.t. $\nabla \cdot \vu = \Div(\vu) = 0$ as the \emph{divergence-free condition} of $\vu$
    in $\Omg(t)$.
\end{remark}
Consequently, the constitutive law simplifies as follows.
\begin{definition}[Incompressible stress tensor]
    \label{def:improssible-stress-tensor}
    The Cauchy stress~\ref{def:fluidsmodel} simplifies to
    \begin{flalign*}
        \quad \vT ~ = ~ -\,p\,\mathrm{I} + 2\,\eta\,\vct{D}(\vu). &&
    \end{flalign*}
\end{definition}
\begin{remark}[Dynamic vs. Kinematic Viscosity]
    In our model assumptions, \emph{dynamic viscosity} $\eta \in \R^+$ and  \emph{kinematic viscosity} $\mu \in \R^+$
    relate as
    \begin{flalign*}
        \quad \mu\; :=\; \frac{\eta}{\rho} ~ = ~  \frac{\eta}{\rho_0} \; \in \; \R^+. &&
    \end{flalign*}
    Here $\eta$ quantifies the internal resistance of blood to shear deformation, i.e., $\eta := \dfrac{\tau}{\dot{\gamma}}$,
    with units $[Pa \cdot s]$. Moreover $\mu$ adjusts $\eta$ by the density $\rho$, capturing the viscous diffusion of momentum per unit mass, with units $\big[\frac{m^2}{s}\big]$.
    Intuitively, $\eta$ measures how "thick" or "sticky" the fluid is, while $\mu$ measures how quickly momentum diffuses through the fluid due to viscosity.
\end{remark}
\begin{remark}[Newtonian Blood Justification]
    \label{rmk:newtonian-justification}
    When diameter $d$ and hematocrit effects are needed, one may use a Non-Newtonian
    model with relative viscosity $\eta_r(H,d)$ that scales an absolute baseline $\eta$:
    \begin{flalign*}
        \quad \eta_{\mathrm{eff}} \;=\; \eta_r(H,d)\,\eta \quad \text{(effective viscosity)}&&
    \end{flalign*}
    An empirical fit from \cite{bviscosity}
    \begin{flalign*}
        \quad \eta_r \;=\; 1 + (\eta_{0.45}-1)\,
        \frac{(1-H)^{C}-1}{(1-0.45)^{C}-1}
        \text{ s.t. }
        \begin{cases}
            \eta_{0.45} = 6\,e^{-0.085\,d} + 3.2 - 2.44\,e^{-0.06\,d^{0.645}},\\[3pt]
            C = \big(0.8 + e^{-0.075\,d}\big)\!\left(\dfrac{1}{1 + 10^{-11}\,d^{12}} - 1\right)
                + \dfrac{1}{1 + 10^{-11}\,d^{12}},
        \end{cases} &&
    \end{flalign*}
    where $d \defined 2R/(1.0 \mu m)$ is the (scaled) vessel diameter.
    In large vessels, $\eta_r$ is often constant, justifying the Newtonian assumption.
    [\cite{book1}, sec. 3.1]
\end{remark}
