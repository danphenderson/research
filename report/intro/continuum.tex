W.a.t. simulate blood flow in a time-dependent fluid domain $\OmgB \subset \R^{N+1}$;
we'll take $N \equivto 3$, the natural setting for the derivations that follow.
For each $t \in I_T \defined [0, T] \subset \R$, with $T > 0$, we define
\begin{flalign*}
    \quad \OmgBt &:= \{\vx \in \R^N : \vx \text{ lies inside the vessel at time } t\}. &&
\end{flalign*}
Formally, the fluid region is a time-dependent family of open sets
$\{\OmgBt\}_{\forallin{t}{I_T}} \subset \R^N$ occupied by blood at time $t$.
Let $\OmgB(0)$ r.t. the \emph{reference} configuration and $\Omg(t)$
r.t. the \emph{current} configuration.
The fluids computational domain r.t. all possible configurations of $\OmgBt$, i.e.,
the spatial-temporal lumen
\begin{flalign*}
    \quad & \Omg_B \defined I_T \times \OmgBt \equivto \{ (t, \vx) \in \R^{N+1} ~:~ x \in \Omgt, ~ t \in I_T \}. &&
\end{flalign*}

\medskip

For each material particle $\vxi \in \OmgB(0)$, let's assume $\vu(t, \vxi(t))$ is the flow
feild determined from a system of $PDEs$ governing $\vxi$'s motion in $\OmgB$.
Our aim here is to derive such PDEs -- the Navier-Stokes system.

\medskip

\paragraph{Blood Dynamics}
The spatial configuration $\OmgBt$ is determined by a Lagrangian mapping
\begin{flalign*}
    \quad & \La_t: \OmgB(0) \mapsto \OmgBt, \quad \vxi \mapsto \vx(t, \vxi) \equals \La_t(\vxi) &&
\end{flalign*}
Let's impose that $\La_t$ is a continuous bijection $\forall t \in I_t$ in $\OmgBBar$, then
$\exists ~ \inv{\La_t}: \ol{\OmgBt} \mapsto \ol{\OmgB(0)}$ s.t. $\vxi = \inv{\La_t}(\vx)$.
Thus, there is a one-to-one correspondence between material fluid particles in the reference
configuration and spatial points in the current configuration at each time $t$.
The variables $(t, \vx)$ and $(t, \vxi)$ are refered to as the Eulerian and Lagrangian coordinates, resp..
\begin{remark}
    \label{rmk:eulerian-lagrangian}
    Informally, the Eulerian approach focuses our attention to $\vx \in \OmgBt$, namely,
    some fluid particle located at $\vx$ at a particular time $t$.
    Whereas, the Lagrangian approach tracks individual fluid particles as they move
    through space and time, following their trajectories from their initial positions
    in the reference configuration.
\end{remark}

\medskip

Let $V(t)$ denote a measurable material volume at time $t$ that moves with the fluid flow, i.e.,
\begin{flalign*}
    \quad V(t) &\subset \OmgBt, \quad \text{for each } t \in I_T &&
\end{flalign*}
Equivalently, $V(t)$ is the image of some reference volume $V(0) \subset \OmgB(0)$ under the
Lagrangian mapping $\La_t$, i.e.,
\begin{flalign*}
    \quad V(t) & = \La_t\big(V(0)\big). &&
\end{flalign*}

\td{Define fields here, then follow with assumptions boxs for blood.}


\medskip
\begin{theorem}[Reynolds Transport Theorem]
    \label{thm:reynolds-transport}
    Let $V(0) \subset \OmgB(0)$ be a material volume in the reference configuration,
    and $V(t) \subset \OmgBt$ its image in the current configuration under $\La_t$. For any
    sufficiently smooth scalar field $\phi: \OmgB \to \R$,
    \begin{flalign*}
        \quad & \frac{d}{dt} \int_{V(t)} \phi(t, \vx) \, dV \;=\; \int_{V(t)} \partial_t \phi(t, \vx) \, dV
            + \int_{\p V(t)} \phi(t, \vx) \, \inp{\vu}{\nhat} \, dS, &&
    \end{flalign*}
    where $\vu$ is the fluid velocity field and $\nhat$ the outward unit normal on $\p V(t)$.
\end{theorem}
\begin{proof}
    Ref~\ref{book2}, theorem 2.2. pg. 21.
\end{proof}



\medskip


For each $t \in I_T$, herein assume $\OmgBt$ is simply connected and boundary $\pOmgB(t) \in C^1$,
so, any closed curve in $\OmgBt$ can be continuously contracted to a point within $\OmgBt$; such an
assumption is reasonable in a health vessel with a smooth wall. This regularity allows us to globally
define geometric quantities such as surface differential operators on the fluids boundary $\pOmgB$, then
we may apply stokes thoerem and Green's identities on the $C^1$ boundaries, useful for obtaining weak
integro-differential and variational formulations.

\begin{remark}
    A weaker regularity condition is that $\pOmgB$ is Lipschitz. Here we must
    rely on standard trace theorems for Soblev spaces (e.g.\ $H^1(\OmgBt)$). In particular,
    the outward unit normal is then defined a.e. on $\pOmgB(t)$, so we can meaningfully speak of
    normal and tangential components of vector feilds on the boundary.
\end{remark}

\medskip

\td{Move fluid definitions here, then follow with assumption boxs for blood.}

\medskip

In the following derivations we let $N \equiv 3$ so that $\OmgBt \subset \R^3$. We assume blood fluid is
a continuum that deforms continously, i.e., at every point $\vx \in \OmgBt$ and time $t \in I_T$, the
blood's kinematic quantities are described by sufficiently smooth fields. At microscopic scales this
continuum hypothesis breaks down, since matter is a discrete collection of molecules, but at macroscopic scales empirical evidence suggests such models remain accurate.

\medskip

Let blood's velocity, thermodynamic pressure, and density fields be
\begin{flalign*}
    \quad & \vu: \OmgB \to \R^3,\;
        (t, x, y, z) \mapsto (u_1(t, x, y, z),\, u_2(t, x, y, z),\, u_3(t, x, y, z))^\top,
        \quad \bigg[\; \frac{m}{s} \;\bigg] && \\
    \quad & p: \OmgB \to \R^+, \;
        (t, x, y, z) \mapsto p(t, x, y, z),
        \quad \bigg[\; \mathrm{Pa} \equiv \frac{N}{m^2} \;\bigg] && \\
    \quad & \rho: \OmgB \to \R^+,\;
        (t, x, y, z) \mapsto \rho(t, x, y, z),
        \quad \bigg[\; \frac{kg}{m^3} \;\bigg]. &&
\end{flalign*}

\medskip

\red{A fluid that deforms independent of time is simple, such fluids deformation and rate of
deformation aren't subject to material memory effects (also r.t. as viscoelastic effects).
The contrary are complex fluids, their deformation is subject to both viscous and elastic
characteristics.} (Out of place)

\begin{definition}
    \label{def:material_derivative}
    Let $V_{\vx}(t)$ be a material fluid element at position $\vx$ in the reference configuration
    $\OmgB(0)$ whos motion is subject to velocity $\vu(t, \vx )$, then
    \begin{flalign*}
        \quad D_t\phi \;:=\; \partial_t \phi + \vu\!\cdot\!\nabla \phi &&
    \end{flalign*}
    is the material derivative $D_t \phi$ measuring the instantanous rate of change of
    a physical quantity $\phi$ moving with velocity field $\vu$.
    The literature also r.t. $D_t$ as the \emph{substantial derivative}, \emph{advective derivative}, \emph{lagrangian derivative}, or \emph{convective derivative}.
\end{definition}
\todo{Incompressible definition hasn't yet been given}
\red{By the incompressibile assumption of blood}, the material density of $V(t)_{\vx}$ is $\rho(\vx, t) \equiv \rho_0$
for all $\vx \in \Omgt$ and $t \in I_T$. \green{Implying $D_t \rho = 0$, so mass conservation in $\Omgt$ reads}
\begin{flalign*}
    \quad & \partial_t \rho + \nabla \cdot (\rho \vu) = 0 \quad \impliedby \text{\red{incompressibility assumption of fluid or flow?}} \quad \text{in } \Omgt &&
\end{flalign*}
By assumption blood is Newtonian, linear momentum balance follows from Newton's 2nd Law ($F = ma$) as
\begin{flalign*}
    \quad \rho(\partial_t \vu + (\vu\cdot\nabla)\vu ) \;=\; \Divbf(\vT) + \rho \vfb, \qquad \text{in} ~ \Omgt &&
\end{flalign*}
where $\vT$ is the Cauchy stress tensor describing the fluids deformation, $\vfb$ may be some body force per unit mass $\big[\dfrac{m}{s^2}\big]$.
Let the rate-of-deformation tensor be defined as the symmetric
part of the velocity gradient, i.e.,
\begin{flalign*}
    \quad \vct{D}(\vu) \defined \tfrac{1}{2}\big(\nabla\vu+(\nabla\vu)^\top\big)
    \quad \text{s.t.} \quad \nabla\vu \defined \vect{
        \dfrac{\partial u_1}{\partial x} & \dfrac{\partial u_1}{\partial y} & \dfrac{\partial u_1}{\partial z} \\
        \dfrac{\partial u_2}{\partial x} & \dfrac{\partial u_2}{\partial y} & \dfrac{\partial u_2}{\partial z} \\
        \dfrac{\partial u_3}{\partial x} & \dfrac{\partial u_3}{\partial y} & \dfrac{\partial u_3}{\partial z}
    }. &&
\end{flalign*}
Note that $\vu \mapsto \vct{D}(\vu)$ captures spatial deformations of element $V_{\vx}$ in $\Omg$ under flow $\vu$.
\begin{definition}
    \label{def:newtonian-fluid}
    A fluid is \emph{Newtonian} if its Cauchy stress tensor $\vT\;$ depends
    linearly on the rate-of-deformation tensor $\vct{D}(\vu)$.
\end{definition}
\begin{definition}
    \label{def:isotropic}
    A fluid is \emph{isotropic} if its constitutive response is independent of the coordinate system.
    Writing the Cauchy stress as $\vT\;=\vT\;(\vct{D})$,
    isotropy means that for every orthogonal rotator $\vct{Q}\in\mathrm{SO}(3)$,
    \begin{flalign*}
        \quad & \vct{Q}\, \vT(\vct{D})\,\vct{Q}^{\top} \;=\; \vT\;\!\big(\vct{Q}\,\vct{D}\,\vct{Q}^{\top}\big). &&
    \end{flalign*}
\end{definition}
\begin{definition}[Newtonian, isotropic constitutive law]
    \label{def:fluidsmodel}
    For a Newtonian, isotropic fluid the Cauchy stress is
    \begin{flalign*}
        \quad & \vT = -\,p\,\mathrm{I} + 2\,\eta\,\vct{D}(\vu) + \lambda\,\Div(\vu)\,\mathrm{I}, &&
    \end{flalign*}
    where $\eta>0$ is the dynamic (shear) viscosity and $\lambda$ the bulk viscosity.
\end{definition}
So an isotropic fluid at rest (quiescent state $\vu\equiv\vct{0}$) sustains only hydrostatic stress:
\begin{flalign*}
    \quad \implies \vT \;=\; -\,p\,\mathrm{I} \quad \text{when} ~ \vu \equiv \vct{0}. &&
\end{flalign*}
One may make a distinction between incompressible fluids and incompressible flows.
feild $\vu$.
\begin{definition}[Incompressibile fluid]
    \label{def:incompressible-fluid}
    An element $V \in \OmgB$ with constant density $\rho(\vx, t) ~ \forallin{t, ~\vx}{I_T \times V(t)}$
    is an \emph{incompressible fluid}.
\end{definition}
\begin{definition}[Incompressible flow]
\label{def:incompressible-flow}
    An element $V \in \OmgB$ subject to $\vu$ with constant rate of material density change
    (in both space and time so that $D_t \rho = 0$ for all $t \in I_T$) undergoes an \emph{incompressible flow} in $\OmgB$.
\end{definition}
I.m.b.s. for all material elements $V \in \Omg$, \ref{def:incompressible-fluid} implies \ref{def:incompressible-flow}.
Note, the converse is not generally true: an incompressible flow ($D_t\rho=0$) only preserves density
along particle paths and allows $\rho=\rho(\vx)$ to vary spatially; but, particularly, in our case the
initial density is distributed uniformly in space, and here incompressible flow implies incompressible fluid.
\begin{remark}[Divergence-free condition]
    \label{rmk:incompressibility-divergence-free}
    By assuming blood is incompressible fluid and flow, we have
    \begin{flalign*}
        \quad & D_t \rho = 0 &&\\
        \quad & \iff \partial_t \rho + \Div(\rho \vu) = 0 &&\\
        \quad & \iff \Div(\rho_0 \vu) = 0 \quad (\because ~ \rho=\rho_0)&&\\
        \quad & \iff \nabla \cdot (\rho_0 \vu) = 0 &&\\
        \quad & \iff \inp{\rho_0 \vu}{\nabla} = 0 &&\\
        \quad & \iff \rho_0 \inp{\vu}{\nabla} = 0 &&\\
        \quad & \iff \inp{\vu}{\nabla} = 0 &&\\
        \quad & \iff \nabla \cdot \vu = 0, &&
    \end{flalign*}
    and we r.t. $\nabla \cdot \vu = \Div(\vu) = 0$ as the \emph{divergence-free condition} of $\vu$
    in $\Omgt$.
\end{remark}
Consequently, the constitutive law simplifies as follows.
\begin{definition}[Incompressible stress tensor]
    \label{def:improssible-stress-tensor}
    The Cauchy stress~\ref{def:fluidsmodel} simplifies to
    \begin{flalign*}
        \quad \vT ~ = ~ -\,p\,\mathrm{I} + 2\,\eta\,\vct{D}(\vu). &&
    \end{flalign*}
\end{definition}
\begin{remark}[Dynamic vs. Kinematic Viscosity]
    In our model assumptions, \emph{dynamic viscosity} $\eta \in \R^+$ and  \emph{kinematic viscosity} $\mu \in \R^+$
    relate as
    \begin{flalign*}
        \quad \mu\; \defined\; \frac{\eta}{\rho} ~ = ~  \frac{\eta}{\rho_0} \; \in \; \R^+. &&
    \end{flalign*}
    Here $\eta$ quantifies the internal resistance of blood to shear deformation, i.e., $\eta \defined \dfrac{\tau}{\dot{\gamma}}$,
    with units $[Pa \cdot s]$. Moreover $\mu$ adjusts $\eta$ by the density $\rho$, capturing the viscous diffusion of momentum per unit mass, with units $\big[\frac{m^2}{s}\big]$.
    Intuitively, $\eta$ measures how "thick" or "sticky" the fluid is, while $\mu$ measures how quickly momentum diffuses through the fluid due to viscosity.
\end{remark}
\begin{remark}[Newtonian Blood Justification]
    \label{rmk:newtonian-justification}
    When diameter $d$ and hematocrit effects are needed, one may use a Non-Newtonian
    model with relative viscosity $\eta_r(H,d)$ that scales an absolute baseline $\eta$:
    \begin{flalign*}
        \quad \eta_{\mathrm{eff}} \;=\; \eta_r(H,d)\,\eta \quad \text{(effective viscosity)}&&
    \end{flalign*}
    An empirical fit from \cite{bviscosity}
    \begin{flalign*}
        \quad \eta_r \;=\; 1 + (\eta_{0.45}-1)\,
        \frac{(1-H)^{C}-1}{(1-0.45)^{C}-1}
        \text{ s.t. }
        \begin{cases}
            \eta_{0.45} = 6\,e^{-0.085\,d} + 3.2 - 2.44\,e^{-0.06\,d^{0.645}},\\[3pt]
            C = \big(0.8 + e^{-0.075\,d}\big)\!\left(\dfrac{1}{1 + 10^{-11}\,d^{12}} - 1\right)
                + \dfrac{1}{1 + 10^{-11}\,d^{12}},
        \end{cases} &&
    \end{flalign*}
    where $d \defined 2R/(1.0 \mu m)$ is the (scaled) vessel diameter.
    In large vessels, $\eta_r$ is often constant, justifying the Newtonian assumption.
    [\cite{book1}, sec. 3.1]
\end{remark}
