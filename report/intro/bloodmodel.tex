One chooses a model based upon the specific application, computational resources,
and desired accuracy. We construct models of blood flow in
various geometries, starting from a single vessel, then extending our approach to
bifurcations and arterial networks. Our strategy involves a \emph{domain decomposition approach}.

\medskip

We seek solutions to inital and boundary value problems of Eq.~\ref{def:NSEquations}.
\begin{definition}
    \label{def:init_condition}
    Let $\vct{u_0} : \Omg(0) \to R^3, ~ \vct{x} \mapsto \vct{u_0}(x) ~ : ~ \vu(0, \vct{x}) = \vct{u_0}$.
    We refer to $\vct{u_0}$ as the initial condition of velocity feild $\vu$
\end{definition}
\begin{definition}
    Let $\vct{u^0}(t)$ and $\vct{u^1}(t)$ be the velocity feilds at $S_0$ and $S_1$.
\end{definition}
In practice, $\vct{u_0}(z)$ may be prescribed or determined from sensor data.

\subsection{Dimension-Reduced Models of Blood Flow}
    \label{subsec:dim-reduced-models}

    We start our disscusion with 1D and 0D models, reducing the
    d.o.f. in the NS system~\ref{def:NSEquations} by imposing further
    simplifying assumptions. Namely, w.a.t. compute average pressures
    and velocities after a relatively short simulation time by solving
    the 1D NS system in a compliant vessel with suitable side conditions.
    Because our model averages pressure and velocity over a surface-area, we obtain
    a uniform distribution of WSS on the vessel.

    \medskip

    One may start by introducing a rigid-vessel assumption, which leads to a no slip condition
    that $\vu \big|_{\pOmgt} \; = \; \vct{0}$. Instead we seek
    to model the link between blood flow and the deformation of the vessel wall.
    We begin our derivation of Dimension-Reduced models by assuming the following
    transformation exists of our fluid domain boundary $\Omgt$ to a simplified geometry.

    \medskip

    \begin{figure}[!htb]
        \label{fig:compliant-vessel}
        \centering
        \captionsetup{aboveskip=2pt,belowskip=2pt}
        \includegraphics[width=0.95\linewidth,height=0.45\textheight,keepaspectratio]{compliant-vessel.png}
        \caption{
            From \cite{book1} [Fig. 3.2, pg. 37]
        }
    \end{figure}

    \newpage

    We consider the fluid dynamics of the following fluid element contained in a portion of the lumen $\Omgt$
    \begin{figure}[!htb]`'
        \label{fig:fluid-element}
        \centering
        \captionsetup{aboveskip=2pt,belowskip=2pt}
        \includegraphics[width=0.8\linewidth,height=0.25\textheight,keepaspectratio]{fig33.png}
        \caption{
            From \cite{book1} [Fig. 3.3, pg. XX]
        }
    \end{figure}
    Let $S_1(t), ~ S_2(t)$ be the time-dependent shaded boundaries at $z = z_1$ and $z = z_2$ s.t.
    $0 < z_1 < z_2 < \ell$.
    Let $V_t$ be the fluid element of blood. The boundary of the fluid element is
    $\partial V_t \; = \; S_1(t) \cup S_2(t) \cup \partial V_{t,w}$ such that $\partial V_{t,w}$
    is the vessel wall in contact with the fluid element.
    According to the Reynold's transport theorem for scalar feild $\phi \in L^1(\Omgt)$.
    \begin{flalign*}
        \quad \dfrac{d}{dt} \int_{V_t} \phi dV \;
            = \; \int_{V_t} \dfrac{\partial \phi }{\partial t} dV \;
            + \; \int_{\partial V_t}  (\vct{u_b} \cdot \nhat) \phi dS &&
    \end{flalign*}
    where $\vct{u_b}$ is the velocity feild deforming the boundary $\partial V_t$
    (Pf. see \href{https://en.wikipedia.org/wiki/Reynolds_transport_theorem}{wiki}).
    If we assume the normal component of $\vct{u_b} = \vct{0}$ near the inlet
    and outlet boundaries $S_1$ and $S_2$ (resp.) of $\Omg$, then the motion of the vessel wall
    is coupled to the blood flow through the fluid element $V_t$. The velocity $\vct{u_b}$
    is equivalent to the velocity of the vessel wall $\partial \Omgt$ in contact with the boundary element $\partial V_t$.
    I.e., the vessel wall velocity $\vct{u_w} = \vct{u_b}$.
    Now let $\vct{w} = \vct{u_w} - \vu$ be the relative velocity of the vessel wall
    w.r.t. the velocity $\vu = (u_1, u_2, u_3)^\top$ of the blood element $V_t$. Then it follows that
    \begin{flalign*}
        \quad \; \int_{\partial V_t} \big(\vct{u_b} \cdot \nhat \big)\phi \; dS ~
            & = ~ \; \int_{\partial V_t} \big(\vct{u_w} \cdot \nhat \big)\phi \; dS &&\\
        \quad \; & = \; \int_{\partial V_t} \big(\vct{w} \cdot \nhat \big)\phi \; dS
            ~ + ~ \; \int_{\partial V_t} \big(\vu \cdot \nhat \big)\phi \; dS &&
    \end{flalign*}
    Let $\bar{\phi}$ denote the average value of $\phi$ defined over a surface $S$
    \begin{flalign*}
        \quad \bar{\phi} \defined \frac{1}{A} ~ \int_{S(z, t)} \phi \;dS\quad :  \quad  A(z, t) \defined \int_{S(z, t)} dS &&
    \end{flalign*}
    Now we may rewrite the volume integral in the LHS of RT theorem
    \begin{flalign*}
        \quad  \int_{V_t} \phi dV \; = \; \int_{z_1}^{z_2} \int_{S(z, t)} \phi \;dS \; dz
            \;=\; \int_{z_1}^{z_2} A \cdot \bar{\phi} \; dz &&
    \end{flalign*}
    where $z_1 < z_2$ are fixed $z$-coordinates for $S_1$ and $S_2$.
    Then we differentiate the integrands in the above equation w.r.t. $t$
    \begin{flalign*}
        \quad \int_{V_t} \dfrac{\partial \phi}{\partial t} dV \;=\; \int_{z_1}^{z_2} \dfrac{\partial}{\partial t} \bigg[A \cdot \bar{\phi} \bigg]\; dz, &&
    \end{flalign*}
    and we've rewritten the first term in the RHS of the reynolds system.
    The surface integral in the RHS may be written as
    \begin{flalign*}
        \quad \int_{\partial V_t}  (\vct{u_b} \cdot \nhat) \phi dS = \int_{\partial V_t}  (\vct{u_b} \cdot \nhat) \phi dS.... &&
    \end{flalign*}\todo{cleanup, ref. pg. 43}
    With a little more work, one may obtain:
    \begin{definition}
        \label{def:1dReynolds}
        The 1D Reynolds Transport theorem for both compressible and incompressible fluids:
        \begin{flalign*}
            \quad \frac{\partial }{\partial t} \bigg( A \bar{\phi} \bigg) \; + \; \frac{\partial}{\partial z}\big( A (\overline{\phi \cdot u_3}) \big)
            \; = \; \int_S \bigg( \dfrac{\partial \phi}{\partial t} + \nabla \cdot (\phi \vu) \bigg) dS
            \; + \; \int_{\partial S} \phi \vct{w} \cdot \nhat \; d \gamma &&
        \end{flalign*}
    \end{definition}

    \begin{remark}
        \label{1DMassConservation}
        By taking $f = \rho$ in \ref{def:1dReynolds}, mass conservation follows directly. Also, by our assumption
        that blood is incompressible, we have $\begin{cases*}\Div(\vu) = 0\\\rho =\text{const.}\\
        \end{cases*}$ and we simplify \ref{def:1dReynolds} as
        \begin{flalign*}
            \quad \frac{\partial A }{\partial t} \; + \; \frac{\partial}{\partial z}\big( A (\overline{u_3}) \big)
            \; = \; \int_{\partial S} \vct{w} \cdot \nhat \; d \gamma &&
        \end{flalign*}
        The RHS term above describing the transport process across the vessel wall. \todo{complete, pg. 45}
    \end{remark}

    \begin{remark}
        \label{1DMomentumConservation}
        By taking $f = u_3$ in \ref{def:1dReynolds}, momentum conservation follows directly. Also,
        by our assumption that blood is incompressible, we simplify \ref{def:1dReynolds} as
        \begin{flalign*}
            \quad \frac{\partial }{\partial t} \bigg( A u_3 \bigg) \; + \; \frac{\partial}{\partial z}\big( A (\overline{u_3^2}) \big)
            \; = \; \int_S \bigg( \dfrac{\partial u_3}{\partial t} + \nabla u_3 \cdot \vu \bigg) dS
            \; + \; \int_{\partial S} u_3 \vct{w} \cdot \nhat \; d \gamma &&
        \end{flalign*}
        The RHS term above describing the transport process across the vessel wall.
    \end{remark}


    \begin{remark}[Tube law from a thin elastic cylindrical wall]
        \label{rem:tube-law-derivation}
        We briefly justify the pressure--area relation used in \textsf{aq\_1d\_compliant.jl}
        \begin{definition}
            \label{def:tube-law}
            \begin{flalign*}
                \quad & p(A) - p_{ext} \;=\; \beta\big(\sqrt{A} - \sqrt{A_0}\big) &&
            \end{flalign*}
        \end{definition}
        used in the 1D $(A,Q)$ model.  Consider a straight cylindrical vessel with
        (local) lumen radius $R(x,t)$, reference radius $R_0$, wall thickness
        $h \ll R$, and internal pressure $p(x,t)$ relative to an external pressure
        $p_{ext}$ (assumed constant in space and time for simplicity).  The
        corresponding lumen area is
        \begin{flalign*}
            \quad & A(x,t) = \pi R(x,t)^2, &&
            \qquad
            A_0 = \pi R_0^2. &&
        \end{flalign*}

        Under the thin--wall assumption, balance of forces in the circumferential
        direction (Young--Laplace law) yields
        \begin{definition}
            \label{def:laplace}
            \begin{flalign*}
                \quad & (p - p_{ext})\,2\pi R \;=\; \sigma_\theta\,2h\pi &&
            \end{flalign*}
        \end{definition}
        where $\sigma_\theta$ is the circumferential (hoop) Cauchy stress in the
        vessel wall.  We model the wall as linearly elastic in the hoop direction,
        so that
        \begin{definition}
            \label{def:linear-elastic}
            \begin{flalign*}
                \quad & \sigma_\theta \;=\; E_{eff}\,\varepsilon_\theta &&
            \end{flalign*}
        \end{definition}
        with an effective circumferential modulus $E_{eff} > 0$ and circumferential
        strain
        \begin{flalign*}
            \quad & \varepsilon_\theta \;=\; \frac{\text{change in circumference} - {\text{reference circumference}}}{\text{reference circumference}} &&
            \quad & = \; \frac{2\pi R - 2\pi R_0}{2\pi R_0} &&
            \quad & = \; \frac{R - R_0}{R_0}. &&
        \end{flalign*}

        Equating \eqref{def:laplace} and \eqref{def:linear-elastic} gives
        \begin{flalign*}
            \quad & (p - p_{ext})\,2\pi R \;=\; E_{eff}\,\frac{R - R_0}{R_0}\,2h\pi &&\\
            \quad & \iff (p - p_{ext})\,R \;=\; E_{eff}\,\frac{R - R_0}{R_0}\,h &&
        \end{flalign*}
        Express $R$ and $R_0$ in terms of the areas $A$ and $A_0$:
        \begin{flalign*}
            \quad  R & = \sqrt{\frac{A}{\pi}} &&\\
            \quad & = \; \frac{\sqrt{A}}{\sqrt{\pi}}, &&\\
            R_0 = \sqrt{\frac{A_0}{\pi}}
            \quad & = \;
            \frac{\sqrt{A_0}}{\sqrt{\pi}},
        \end{flalign*}
        so that
        \begin{flalign*}
            \quad R - R_0
            \;=\;
            \frac{\sqrt{A}}{\sqrt{\pi}} - \frac{\sqrt{A_0}}{\sqrt{\pi}} &&\\
            \quad & = \; \frac{1}{\sqrt{\pi}}\big(\sqrt{A} - \sqrt{A_0}\big). &&
        \end{flalign*}
        Substituting into the expression for $p - p_{ext}$, we obtain
        \begin{flalign*}
            \quad & (p - p_{ext})\,\frac{\sqrt{A}}{\sqrt{\pi}}
            \;=\;
            E_{eff} h\,
            \frac{\dfrac{1}{\sqrt{\pi}}\big(\sqrt{A} - \sqrt{A_0}\big)}{R_0} &&\\
            \quad & \iff (p - p_{ext})\,\frac{\sqrt{A}}{\sqrt{\pi}}
            \;=\;
            E_{eff} h\,
            \frac{\dfrac{1}{\sqrt{\pi}}\big(\sqrt{A} - \sqrt{A_0}\big)}{R_0}. &&\\
        \end{flalign*}
        For moderate deformations where $A$ remains close to $A_0$, we approximate
        the factor $1/\sqrt{A}$ by its reference value $1/\sqrt{A_0}$, which yields
        \begin{flalign*}
            \quad & (p - p_{ext})\,\frac{\sqrt{A_0}}{\sqrt{\pi}}
            \;=\;
            E_{eff} h\,
            \frac{\dfrac{1}{\sqrt{\pi}}\big(\sqrt{A} - \sqrt{A_0}\big)}{R_0} &&\\
            \quad & \iff (p - p_{ext})
            \;=\;
            \frac{E_{eff} h}{R_0\sqrt{A_0}}\,
            \big(\sqrt{A} - \sqrt{A_0}\big). &&
        \end{flalign*}
        Defining the lumped stiffness parameter
        \begin{flalign*}
            \quad & \beta \;:=\; \frac{E_{eff} h}{R_0\sqrt{A_0}}. &&
        \end{flalign*}
        we arrive at the tube law \eqref{eq:tube-law} used in the 1D model:
        \begin{flalign*}
            \quad &
            p(A) - p_{ext}
            \;=\;
            \beta\big(\sqrt{A} - \sqrt{A_0}\big). &&
        \end{flalign*}
        In the numerical experiments below, we take $\beta$ and $A_0$ to be
        constant along the vessel, so that $p$ can be written as a function of
        $A$ alone.
    \end{remark}

    \medskip

    \subsubsection{0D Models}
        \label{subsubsec:0D}
        The 0D model, on the other hand, treats the vessel as a lumped parameter system,
        focusing on overall pressure and flow relationships without spatial resolution.
