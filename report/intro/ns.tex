When coupling our momentum balance equation with the divergence-free condition of $\vu$, we obtain the
Navier-Stokes (NS) equations. We present several equivalent forms of the NS equations below.
First, we write the NS system in conservative form.
\medskip
\begin{definition}[Conservative-Momentum Balance Form]
    \label{def:nsconservative}
    \begin{flalign*}
        \quad & \begin{cases*}
            \quad \pd_t(\rho \vu) \; +\;  (\rho\vu\cdot\nabla)\vu
                \equals -\,\nabla p \;\+\; \Divbf\!\big( 2\,\eta\,\vct{D}(\vu) \big) \;\+\; \rho\,\vf, \\
            \quad \Div(\vu) \equals 0,\qquad \rho \equivto \rho_0>0 \text{ (constant).}
        \end{cases*} &&
    \end{flalign*}
\end{definition}
Since $\rho \equivto \rho_0$ and $\eta \equivto \mu |\vct{D}|^2$, the advective form follows
from~\ref{def:nsconservative}.
\begin{definition}[Generalized-Newtonian Navier-Stokes (NS)]
    \label{def:NSEquations}
    \begin{flalign*}
        \quad & \begin{cases*}
            \quad \rho\big(\pd_t \vu + (\vu\cdot\nabla)\vu\big) \equals -\,\nabla p \;+\; \Divbf\!\Big( 2\,\mu(|\vct{D}|^2)\,\vct{D}\Big) \;+\; \rho\,\vf \\
        \quad \Div(\vu) = 0,
        \end{cases*} &&
    \end{flalign*}
\end{definition}
We divide $\rho$ and obtain the kinematic-viscosity form from~\ref{def:NSEquations}
as presented in \cite{book1}, \cite{book2}, and \red{\cite{hemodynamics} (confirm)}.
\begin{definition}[Kinematic-Viscosity Navier-Stokes (NS)]
    \label{def:kinematic-viscosity-ns}
    \begin{flalign*}
        \quad & \begin{cases*}
        \quad \pd_t \vu + (\vu\cdot\nabla)\vu \equals -\,\frac{\nabla p}{\rho}\, \;+\; \Divbf\!\Big( \frac{2}{\rho}\,\mu(|\vct{D}|^2)\,\vct{D}\Big) \;+\; \vct{f}, \\
        \quad \Div(\vu) = 0,
        \end{cases*} &&
    \end{flalign*}
\end{definition}
\medskip
Finally, we write the NS system in operator form. \red{(check arguments of F.
Rewrite with Laplacian instead of divergence of stress tensor?)}
{\small
\begin{definition}
    \label{def:standard-form-ns}
    We write Eq. \ref{def:NSEquations} in standard form
    \begin{flalign*}
        \quad & \begin{cases*}
            \quad F\big(\pd_t\vu,\,\nabla\vu,\,\nabla p,\, \vu,\, p;\, \rho, \mu\big) \equals \vct{f}, \\
            \quad \Div(\vu) \equals 0,
        \end{cases*} && \\
        \text{where} \quad & F\big(\pd_t\vu,\,\nabla\vu,\,\nabla p,\, \vu,\, p;\, \rho, \mu\big) \defined \pd_t \vu \;+\;  (\vu\cdot\nabla)\vu \;+\; \frac{\nabla p}{\rho}  \;-\;  \Divbf\!\Big( \frac{2}{\rho}\;\mu\;(|\vct{D}|^2)\,\vct{D}\Big) &&
    \end{flalign*}
\end{definition}}
\begin{remark}
    \label{classifying NS}
    The NS equations are a non-linear coupled system of PDEs.
    \medskip
    The first equation follows from the balance of linear momentum and it's
    terms are characterized as:
    \begin{itemize}
        \item  The convective term $(\vu\cdot\nabla)\vu = \vect{
            \inp{\vu}{\nabla \vu_1}\\
            \inp{\vu}{\nabla \vu_2}\\
            \inp{\vu}{\nabla \vu_3}
        }$ governs acceleration of fluid (non-linear).
        \item The diffusive term $\Divbf\!\Big( 2\,\mu(|\vct{D}|^2)\,\vct{D}\Big)$ describes
        the viscouselastic behavior (linear since $\mu$ is constant).
    \end{itemize}
    The second equation is the continuity equation, a consequence of the assumed fluid properties of blood
    that lead to the divergence-free condition on $\vu$.
    \medskip
    The total system comprises of four equations in four unknowns: the three components of the velocity field
    $\vu$ and the pressure field $p$.
\end{remark}
If pressure $p$ and the velocity $\vu$ are given, the Cauchy stress $\vT$ is computed
from Eq. \ref{def:improssible-stress-tensor}. It follows that the wall shear stress (WSS) at the vessel wall is:
\begin{flalign*}
    \quad \text{WSS} \defined \inp{\vct{t_{blood}}}{\vT\;\hat{n}} \quad : \quad \begin{cases*}
        t_{blood} \text{ is tangent of a flow line through a cross-sectional area} \\
        \hat{n} \text{ is outer normal of the cross-sectional area}
    \end{cases*}&&
\end{flalign*}
Forgoing the rigid-wall assumption allows us to model the relationship
between the vessel wall and blood flow. Applicable models are referred
to as fluid-structure interaction (FSI) models.\todo{discuss in a later section}

\medskip

\subsection{Existence and Uniqueness of NS}
    \label{subsec:existenceUniqueness}

    \begin{remark}[Global Regularity Problem for (NS)]\
        \label{rem:global-regularity}
        \textit{For any smooth, spatially localized initial data $ \mathbf{u}_0 $,
        does there exist a global smooth solution $ (\mathbf{u}, p) $ to NS~\ref{def:NSEquations}, when $N\equals 3$.
        Such question is one of the Millennium Prize Problems posed
        by the Clay Mathematics Institute in 2000, with a prize of one million
        dollars for a correct solution.}
    \end{remark}

    \begin{theorem}[Local Existence and Uniqueness]
        \label{thm:local-existenceUniqueness}
        Given smooth, localized initial data $\vct{u}_0$, there exists a maximal time $0 < T_* \leq \infty$
        for which a unique solution exists.
    \end{theorem}

    \medskip

    If $T_* < \infty$, a \textbf{blow-up} occurs:
    \begin{flalign*}
        \quad & \sup_{x \in \mathbb{R}^3} |\mathbf{u}(t, x)| \to +\infty \quad \text{as} \quad t \to T_*. &&
    \end{flalign*}
    Otherwise, if $T_* = \infty$, then $|\vct{u}| \to 0$ as $t \to \infty$. Numerical evidence suggests global regularity
    holds in most practical cases, but turbulent behavior can emerge for large initial data.

    \subsubsection*{Heuristic Considerations and Energy Balance}
        Starting from the incompressibility condition:
        \begin{flalign*}
            \quad & \Div(\vct{u}) = 0 && \\
            \quad & \iff ~ \rho \text{ is constant in } \Omega(t) && \\
            \quad & \iff ~ \text{ chain rule applies to } \Divbf\big(\frac{2}{\rho} \mu (|\vct{D}|^2) \vct{D}\big) &&\\
            \quad \implies &  \Divbf\big(\frac{2}{\rho} \mu (|\vct{D}|^2) \vct{D}\big) = \nabla \cdot \big(\frac{2}{\rho} \mu (|\vct{D}|^2) \vct{D}\big)&&\\
            \quad & = \langle\frac{2}{\rho} \mu (|\vct{D}|^2) \vct{D}, \nabla \rangle&&\\
            \quad  & = \frac{2}{\rho} \langle \mu (|\vct{D}|^2) \vct{D}, \nabla \rangle&&\\
            \quad & = \frac{2}{\rho} \nabla \cdot \big( \mu (|\vct{D}|^2) \vct{D}\big)&&\\
            \quad & \therefore \quad \frac{2}{\rho}~ \Divbf\big( \mu (|\vct{D}|^2) \vct{D}\big) = 0.&&
        \end{flalign*}
        This vanishes if $\mu$ is constant (Newtonian fluid) and $\rho$ is constant (incompressibility).
        So the diffusive term becomes:
        \begin{flalign*}
            \quad \frac{2\eta}{\rho} \Delta \vct{u}, \quad \text{with } \eta = \mu \rho.&&
        \end{flalign*}\todo{show simplification of diffusive term to laplacian}
        We heuristically compare dominant terms:
        \begin{enumerate}
            \item If $\eta \Delta \vct{u} \gg (\vct{u} \cdot \nabla) \vct{u}$, viscous dissipation dominates $\Rightarrow$ smooth, regular behavior.\
            \item If $(\vct{u} \cdot \nabla) \vct{u} \gg \eta \Delta \vct{u}$, nonlinearity dominates $\Rightarrow$ turbulence, potential blow-up.
        \end{enumerate}
        We construct rigorous energy estimates in Section~\ref{sec:math-models}.
    \medskip

\subsubsection{NS in Cylindrical Coordinates}
    \label{subsubsec:cylindrical-coords}
    The relationship between cartesian and cylidrical coordinates is
    \begin{flalign*}
        \quad (x, ~y, ~z) ~ \mapsto (r \sin(\theta), ~ r\cos(\theta), ~z), \quad r = \sqrt{x^2 + y^2}.&&
    \end{flalign*}

    \medskip

    Let $(r,\theta,z)$ denote cylindrical coordinates on $\R^3$ with
    orthonormal basis $(\e_r,\e_\theta,\e_z)$, and write the velocity as
    \begin{flalign*}
        \quad \vu(t, r,\theta,z) &\equivto u_r(t, r, \theta,z)\,\e_r \;+\; u_\theta(t, r, \theta,z)\,\e_\theta \;+\; u_z(t, r, \theta,z)\,\e_z. &&\\
        &\equals u_r\,\e_r \;+\; u_\theta\,\e_\theta \;+\; u_z\,\e_z
    \end{flalign*}
    where $u_r, u_\theta, u_z: \OmgB \to \R$ are the cylindrical
    velocity components in the radial, azimuthal, and axial directions respectively. Then the kinematic-viscosity form of NS~\ref{def:kinematic-viscosity-ns} in cylindrical coordinates
    involves the following differential operators:
    \begin{flalign*}
        \quad (\vu\cdot\nabla) &\equals u_r \,\pd_r \;+\; \frac{u_\theta}{r}\,\pd_\theta \;+\; u_z \,\pd_z, &&\\
        \nabla p &\equals \pd_r p\,\e_r \;+\; \frac{1}{r}\,\pd_\theta p\,\e_\theta \;+\; \pd_z p\,\e_z, &&\\
        \Div(\vu) &\equals \frac{1}{r}\,\pd_r(r u_r) \;+\; \frac{1}{r}\,\pd_\theta u_\theta \;+\; \pd_z u_z. &&
    \end{flalign*}
    The components of the rate-of-strain tensor $\vct{D} = \tfrac{1}{2}\big(\nabla\vu + (\nabla\vu)^\top\big)$
    with respect to $(\e_r,\e_\theta,\e_z)$:
    \begin{flalign*}
        \quad D_{rr} &= \pd_r u_r, & D_{\theta\theta} &= \frac{1}{r}\,\pd_\theta u_\theta + \frac{u_r}{r}, & D_{zz} &= \pd_z u_z, && \\
        \quad D_{r\theta} &= \frac{1}{2}\Big(\pd_r u_\theta - \frac{u_\theta}{r} + \frac{1}{r}\,\pd_\theta u_r\Big), &
        D_{rz} &= \frac{1}{2}\big(\pd_r u_z + \pd_z u_r\big), & D_{\theta z}&= \frac{1}{2}\Big(\frac{1}{r}\,\pd_\theta u_z + \pd_z u_\theta\Big) &&.
    \end{flalign*}
    The viscous (extra) stress is $\vct{\tau} = 2\,\mu(|\vct{D}|^2)\,\vct{D}$, so
    \begin{flalign*}
        \quad \tau_{ij} = 2\,\mu(|\vct{D}|^2)\,D_{ij}, \qquad i,j\in\{r,\theta,z\}. &&
    \end{flalign*}
    Writing $\vct{f} = f_r\,\e_r + f_\theta\,\e_\theta + f_z\,\e_z$, the three components of the momentum equation are
    \begin{flalign*}
        \text{(radial)}\quad
        &\pd_t u_r
        + u_r \pd_r u_r
        + \frac{u_\theta}{r}\pd_\theta u_r
        + u_z \pd_z u_r
        - \frac{u_\theta^2}{r} &&\\
        &\quad \equals -\,\frac{1}{\rho}\,\pd_r p
        + \frac{1}{\rho}\Big[
              \frac{1}{r}\pd_r(r \tau_{rr})
            + \frac{1}{r}\pd_\theta \tau_{r\theta}
            + \pd_z \tau_{rz}
            - \frac{1}{r}\tau_{\theta\theta}
          \Big]
        + f_r, && \\
        \text{(azimuthal)}\quad
        &\pd_t u_\theta
        + u_r \pd_r u_\theta
        + \frac{u_\theta}{r}\pd_\theta u_\theta
        + u_z \pd_z u_\theta
        + \frac{u_r u_\theta}{r} \\
        &\quad \equals -\,\frac{1}{\rho r}\,\pd_\theta p
        + \frac{1}{\rho}\Big[
              \frac{1}{r}\pd_r(r \tau_{r\theta})
            + \frac{1}{r}\pd_\theta \tau_{\theta\theta}
            + \pd_z \tau_{\theta z}
            + \frac{1}{r}\tau_{r\theta}
          \Big]
        + f_\theta, &&\\
        \text{(axial)}\quad
        &\pd_t u_z
        + u_r \pd_r u_z
        + \frac{u_\theta}{r}\pd_\theta u_z
        + u_z \pd_z u_z \\
        &\quad \equals -\,\frac{1}{\rho}\,\pd_z p
        + \frac{1}{\rho}\Big[
              \frac{1}{r}\pd_r(r \tau_{rz})
            + \frac{1}{r}\pd_\theta \tau_{\theta z}
            + \pd_z \tau_{zz}
          \Big]
        + f_z. &&
    \end{flalign*}
    The incompressibility condition in cylindrical coordinates is
    \begin{flalign*}
        \quad \frac{1}{r}\,\pd_r(r u_r) \;+\; \frac{1}{r}\,\pd_\theta u_\theta \;+\; \pd_z u_z \equals 0. &&
    \end{flalign*}



    \begin{figure}[!htb]
    \label{fig:compliant-vessel}
        \centering
        \captionsetup{aboveskip=2pt,belowskip=2pt}
        \includegraphics[width=0.45\linewidth,height=0.45\textheight,keepaspectratio]{compliant-vessel.png}
        \caption{
            From \cite{book1} [Fig. 3.2, pg. 37]: A compliant vessel with
            time-dependent radius $R(z,t)$ along the axial position $z$.
            Since $\pOmgB \in C^1$, the normal vector $\hat{n}$ is well-defined a.e. on the boundary.
            We assume such transformation is possible for blood vessels in our model.
        }
    \end{figure}
    Assume a vessel of length $\ell$ is aligned with the z-axis
    whose cross-section is circular with radius $R(z,t)$ at axial position $z$ and time $t$.
    Our fluid domain becomes
    \begin{equation*}
        \label{fluid-domain-cylidrical}
        \Omgt \; = \; \{(r, ~\theta ~, z) \in \R^3 ~ : ~r \in [0, R(z, t)], ~ \theta \in [0, 2\pi), ~ z \in [0, \ell) \}
    \end{equation*}
    where $R(z,t)$ is the vessel radius at axial position $z$ and time $t$.

    \medskip
    \td{WIP: continue from here...}
