\documentclass[10pt]{article}



% ===================== PREAMBLE ====================================
% Contains packages, custom commands, and formatting
% ===================================================================
% Packages
% ---------- Basic Packages ----------
\usepackage[utf8]{inputenc}
\usepackage{array}
\usepackage{geometry}
\usepackage{setspace}
\usepackage{multicol}
\usepackage{xcolor}
\usepackage{graphicx}
\usepackage{float}
\usepackage{caption}
\usepackage{booktabs}
\usepackage{url}
\usepackage{hyperref}
\usepackage{lipsum}
\usepackage{fancybox}
\usepackage{todonotes}
\usepackage[dvipsnames]{xcolor}
\usepackage{mdframed} % boxed theorems
\usepackage{thmtools} % couples well with mdframed
\usepackage{tikz} \usetikzlibrary{arrows.meta, calc, decorations.pathmorphing, positioning}
\usepackage{listings}

% --------- Code Listings ----------
\captionsetup[lstlisting]{font={small}, labelfont={bf}, labelsep=colon}
\renewcommand{\lstlistingname}{Code}

% ---------- Color Commands ----------
\DeclareRobustCommand{\yellow}[1]{\begingroup\color{Yellow}\ignorespaces#1\endgroup\ignorespacesafterend}
\DeclareRobustCommand{\red}[1]{\begingroup\color{Red}\ignorespaces#1\endgroup\ignorespacesafterend}
\DeclareRobustCommand{\blue}[1]{\begingroup\color{Blue}\ignorespaces#1\endgroup\ignorespacesafterend}
\DeclareRobustCommand{\green}[1]{\begingroup\color{Green}\ignorespaces#1\endgroup\ignorespacesafterend}
\DeclareRobustCommand{\gray}[1]{\begingroup\color{Gray}\ignorespaces#1\endgroup\ignorespacesafterend}
\newenvironment{redblock}{\begingroup\color{Red}\ignorespaces}{\endgroup\ignorespacesafterend}


% ---------- Math and Theorem Environments ----------
\usepackage{algorithm}
\usepackage{algpseudocode}
\usepackage{amsmath, amssymb, amsthm, amsfonts, mathtools, bm, amsopn}
\usepackage{witharrows}
\usepackage[dvipsnames]{xcolor}
\usepackage{mdframed}       % boxed theorems
\usepackage{thmtools}       % couples well with mdframed
\usepackage{capt-of}        % if you prefer \captionof

% ================== THEOREM ENVIRONMENTS ==================
\numberwithin{equation}{subsection} % optional

% Main shared counter “theorem” per section:
\newtheorem{theorem}{Theorem}[subsection]

% All others share the same counter as “theorem”:
\newtheorem{lemma}[theorem]{Lemma}
\newtheorem{proposition}[theorem]{Proposition}
\newtheorem{corollary}[theorem]{Corollary}

\theoremstyle{definition}
\newtheorem{definition}[theorem]{Definition}
\newtheorem{example}[theorem]{Example}
\newtheorem{prop}[theorem]{Proposition}

\theoremstyle{remark}
\newtheorem{remark}[theorem]{Remark}

% Boxed Assumption using the custom style, sharing the same counter:
\declaretheorem[
  name=Assumption,
  style=boxed,
  sibling=theorem
]{assumption}
% % Define a theorem style that uses mdframed for the box
% \declaretheoremstyle[
%   headfont=\normalfont\bfseries,
%   mdframed={
%     linewidth=0.5pt,
%     roundcorner=4pt,
%     backgroundcolor={gray!5},
%     innertopmargin=6pt,
%     innerbottommargin=6pt,
%     innerleftmargin=8pt,
%     innerrightmargin=8pt
%   }
% ]{boxed}
% % Create the “assumption” environment, numbered within each section
% \declaretheorem[name=Assumption, style=boxed, numberwithin=section]{assumption}
% % Other logical environments
% \newenvironment{solution}
%   {\renewcommand\qedsymbol{$\blacksquare$}\begin{proof}[\textbf{Solution}]}
%   {\end{proof}}
% \newtheorem{lemma}{Lemma}
%   \renewcommand{\thelemma}{\Alph{lemma}}
% \newtheorem*{lem}{Lemma}
% \newtheorem{corollary}[lemma]{Corollary}
% \newtheorem*{theorem}{theorem}
% \theoremstyle{definition}
% \newtheorem{definition}{Definition}
% \theoremstyle{remark}
% \newtheorem*{remark}{Remark}
% \theoremstyle{claim}
% \newtheorem*{claim}{Claim}

% ---------- Page setup ----------
\geometry{letterpaper, margin=1in}
\setlength{\parindent}{0pt}
\linespread{1.5}
\renewcommand{\arraystretch}{1.25} % Adjust as needed

% ---------- Shaded box's to tag ideas ----------
\newcommand{\td}[1]{\todo[inline, color=red!20, size=\small]{#1}}
\newcommand{\note}[1]{\todo[inline, color=blue!20, size=\small]{#1}}
\newcommand{\remeber}[1]{\todo[inline, color=green!20, size=\small]{#1}}

% ---------- Sets and number systems ----------
\newcommand{\Z}{\mathbb{Z}}
\newcommand{\Q}{\mathbb{Q}}
\newcommand{\C}{\mathbb{C}}
\newcommand{\R}{\mathbb{R}}
\newcommand{\N}{\mathbb{N}}
\newcommand{\La}{\mathcal{L}}

% ---------- Summations ----------
\newcommand{\sumforall}{\sum_{\forall i}}
\newcommand{\sumfor}[1]{\sum_{\forall #1}}

% ---------- Generic Domains ----------
\newcommand{\Omg}{\Omega}
\newcommand{\OmgBar}{\overline{\Omega}}
\newcommand{\Omgt}{\Omega(t)}
\newcommand{\pOmg}{\partial \Omega}
\newcommand{\pOmgt}{\partial \Omega(t)}


%---------- Blood Domain ----------
\newcommand{\OmgB}{\Omega_B}
\newcommand{\OmgBt}{\Omega_B(t)}
\newcommand{\OmgBtBar}{\overline{\Omega}_B(t)}
\newcommand{\OmgBBar}{\overline{\Omega}_B}
\newcommand{\pOmgB}{\partial \Omega_B}
\newcommand{\pOmgBt}{\partial \Omega_B(t)}

%---------- Vessel Wall Domain ----------
\newcommand{\OmgW}{\Omega_W}
\newcommand{\OmgWt}{\Omega_W(t)}
\newcommand{\pOmgW}{\partial \Omega_W}

% ---------- Symbols ----------
\newcommand{\e}{\epsilon}
\newcommand{\p}{\rho}
\newcommand{\eps}{\epsilon}
\newcommand{\la}{\lambda}

% ---------- Linear Algebra ----------
\newcommand{\matr}[1]{\begin{pmatrix}#1\end{pmatrix}} % For matrices
\newcommand{\vect}[1]{\begin{bmatrix}#1\end{bmatrix}} % For column vectors
\newcommand{\vct}[1]{\bm{#1}} % For bold vectors
\newcommand{\transp}{^{\mathsf{T}}} % Transpose symbol
\newcommand{\sym}{\text{sym}} % Symmetric part of a matrix
\DeclarePairedDelimiter\ip{\langle}{\rangle} % Inner product
\DeclarePairedDelimiter\norm{\lVert}{\rVert} % Norm
\newcommand{\zero}{\mathbf{0}} % Zero vector/matrix
\newcommand{\ident}{\mathbf{I}} % Identity matrix
\newcommand{\id}[1]{\operatorname{id}_{#1}} % Identity map on set #1
\DeclareMathOperator{\tr}{tr} % Trace
\DeclareMathOperator{\detm}{det} % Determinant
\DeclareMathOperator{\rank}{rank} % Rank
\DeclareMathOperator{\nullity}{nullity} % Nullity

% Block Matrices and Vectors ----------
\newcommand{\blockmat}[2]{\begin{pmatrix} #1 & #2 \end{pmatrix}} % Block matrix with 2 columns
\newcommand{\augmatr}[2]{\left(\begin{array}{#1|#1}#2\end{array}\right)}
\newcommand{\bmfour}[4]{\left( \begin{array}{c|c} #1 & #2 \\ \hline #3 & #4 \end{array} \right)} % 2x2 block matrix with lines
\newcommand{\blockvect}[2]{\begin{bmatrix} #1 \\ #2 \end{bmatrix}} % Block vector with 2 rows

% Unit vectors ----------
\newcommand{\nhat}{\hat{\vct{n}}}
\newcommand{\that}{\hat{\vct{t}}}
\newcommand{\ihat}{\hat{\vct{i}}}
\newcommand{\jhat}{\hat{\vct{j}}}
\newcommand{\khat}{\hat{\vct{k}}}

% Generic vectors and tensors. (bold notation) ---------
\newcommand{\vu}{\vct{u}}
\newcommand{\vv}{\vct{v}}
\newcommand{\vT}{\vct{T}}
\newcommand{\vt}{\vct{t}}
\newcommand{\vTx}{\vct{T}(\vct{x})}
\newcommand{\vx}{\vct{x}}
\newcommand{\vxi}{\vct{\xi}}
\newcommand{\vfb}{\vct{f_b}}
\newcommand{\vf}{\vct{f}}
\newcommand{\vF}{\vct{F}}
\newcommand{\vA}{\vct{A}}
\newcommand{\vFhat}{\widehat{\vct{F}}}
\newcommand{\vJ}{\vct{J}}
\newcommand{\vJhat}{\widehat{\vct{J}}}
\newcommand{\Jdet}{J}
\newcommand{\Jhatdet}{\widehat{J}}
\newcommand{\phihat}{\widehat{\phi}}
\newcommand{\vw}{\vct{w}}
\newcommand{\va}{\vct{a}}
\newcommand{\vahat}{\widehat{\vct{a}}}
\newcommand{\vuhat}{\widehat{\vct{u}}}
\newcommand{\vub}{\widehat{\vct{u_b}}}
\newcommand{\vepsilon}{\vct{\epsilon}}
\newcommand{\vtheta}{\vct{\theta}}
\newcommand{\ej}{\vct{e_j}}
\newcommand{\eo}{\vct{e_1}}
\newcommand{\ef}{\vct{e_1}} % first basis vector
\newcommand{\el}{\vct{e_n}} % nth basis vector

% Learning Stuff
\newcommand{\vW}[1]{\vct{W_{#1}}} % Weight matrix
\newcommand{\vN}[1]{\vct{N_{#1}}} % Neural Network Layer output}
\newcommand{\vb}[1]{\vct{b_{#1}}} % Bias vector
\newcommand{\vsigma}{\vct{\sigma}} % Activation function
\newcommand{\vmL}{\mathcal{L}_{\text{obj}}} % Loss function
\newcommand{\vs}{\vct{s}} % Step direction
\newcommand{\vH}{\vct{H}} % Hessian matrix
\newcommand{\vd}{\vct{d}} % Direction vector 1

% Basis vectors
\newcommand{\generalbasis}{\{\vct{v_1}, \ldots, \, \vct{v_n}\}}
\newcommand{\stdbasis}{\{\vct{e_1}, \ldots, \, \vct{e_n}\}}

% ---------- Operators ----------
\newcommand{\equivto}{\; \equiv \;}
\newcommand{\defined}{\; := \;}
\newcommand{\equals}{\; = \;}
\newcommand{\ol}{\overline}
\newcommand{\pd}{\partial}
\newcommand{\dt}{\frac{d}{dt}}
\newcommand{\pdt}{\frac{\partial}{\partial t}}
\newcommand{\dx}{\, d\vct{x}}
\newcommand{\dxi}{\, d\vct{\xi}}
\newcommand{\dSx}{\, dS_{\vct{x}}}
\newcommand{\argmin}{\operatorname{argmin}}
\DeclareMathOperator{\col}{col}
\DeclareMathOperator{\spn}{span}
\DeclareMathOperator{\range}{range}
\DeclareMathOperator{\kernel}{ker}
\newcommand{\inv}[1]{#1^{-1}}
\DeclareMathOperator{\dia}{diag}

\newcommand{\del}{\Delta}
\newcommand{\Div}{\text{div}}
\newcommand{\Divbf}{\textbf{div}}
\newcommand{\inp}[2]{\langle #1, \; #2 \rangle}
\newcommand{\forallin}[2]{\forall #1 \in #2}
\newcommand{\forallsubsets}[2]{\forall \; #1 \subset #2}
\newcommand{\units}[1]{\; \big[ \; #1 \; \big]}
\newcommand{\blinform}[2]{a\big( #1, \; #2 \big)}

% ---------- Define custom colors ----------
\definecolor{bg}{rgb}{0.95,0.95,0.92}
\definecolor{codeblue}{rgb}{0.25,0.5,0.75}
\definecolor{keyword}{rgb}{0.5,0,0.5}
\definecolor{string}{rgb}{0,0.5,0}

% ---------- Define the style for IPython ----------
\lstdefinelanguage{ipython}{
  basicstyle=\ttfamily\footnotesize\color{black},
  keywordstyle=\color{keyword}\bfseries,
  stringstyle=\color{string},
  backgroundcolor=\color{bg},
  morekeywords={In,Out},
  frame=lines,
  numbers=left,
  numbersep=5pt
}

% ---------- Define the style for Julia ----------
\lstdefinelanguage{julia}{
  keywords={struct, abstract, primitive, typealias, bitstype, mutable, module, baremodule, import, importall, using, export, if, else, elseif, for, while, break, continue, function, return, type, global, local, const, let, do, try, catch, finally, end, quote, macro, where},
  basicstyle=\ttfamily\footnotesize\color{black},
  keywordstyle=\color{keyword}\bfseries,
  stringstyle=\color{string},
  backgroundcolor=\color{bg},
  frame=lines,
  numbers=left,
  numbersep=8pt,
  inputencoding=utf8,
  extendedchars=true,
  escapeinside={(*}{*)},
  literate={~}{{$\sim$}}1
           {₁}{{\textsubscript{1}}}1
           {₂}{{\textsubscript{2}}}1
           {₃}{{\textsubscript{3}}}1
           {ᵀ}{{$^{\mathrm{T}}$}}1
           {⋅}{{$\cdot$}}1
           {∈}{{$\in$}}1
           {ℝ}{{$\mathbb{R}$}}1
           {→}{{$\to$}}1
           {π}{{$\pi$}}1
           {≤}{{$\leq$}}1
           {≥}{{$\geq$}}1
           {≈}{{$\approx$}}1
           {ϵ}{{$\epsilon$}}1
           {κ}{{$\kappa$}}1
           {Δ}{{$\Delta$}}1
           {λ}{{$\lambda$}}1
}

% ---------- Define the style for Mathematica ----------
\lstdefinelanguage{Mathematica}{
  keywords={Module,With,Block,If,Then,Else,Which,Switch,For,While,Return,Do,Table,Plot,Map,Apply,Function,Set,SetDelayed,Clear,Quit,NonlinearModelFit,FindMinimum,FindMaximum,FindRoot,NIntegrate,Integrate,Simplify,FullSimplify,DSolve,RSolve,NSolve,NDSolve,Limit,Series,Assuming,Expand,Factor,TableForm,MatrixForm,Part,Length,Dimensions,Transpose,MapThread,MapAt,Flatten,Thread,Join,Outer,ConstantArray,Riffle,ArrayPlot,Plot3D,ContourPlot,ParametricPlot,MatrixPlot,Graphics,Graphics3D,Manipulate,Evaluate,Sin,Cos,Exp,Log,Sqrt},
  keywordstyle=\color{keyword}\bfseries,
  sensitive=true,
  morecomment=[l]{(*}, morecomment=[r]{*)},
  commentstyle=\itshape\color{gray},
  morestring=[b]",
  stringstyle=\color{string},
  basicstyle=\ttfamily\footnotesize\color{black},
  backgroundcolor=\color{bg},
  frame=lines,
  numbers=left,
  numbersep=5pt,
  showstringspaces=false,
  escapeinside={(*}{*)}
}

% ---------- Optionally define a style for Mathematica code cells ----------
\lstdefinestyle{mathematicaStyle}{
  language=Mathematica,
  frame=lines,
  backgroundcolor=\color{bg},
  basicstyle=\ttfamily\footnotesize,
  keywordstyle=\color{keyword}\bfseries,
  commentstyle=\itshape\color{gray},
  stringstyle=\color{string},
  numbers=left,
  numbersep=5pt,
  showstringspaces=false
}

\usepackage{csquotes}                                   % TODO: why isn't this in preamble.tex?
\usepackage[backend=biber,style=numeric]{biblatex}
\addbibresource{citations.bib}                          % BibTeX file



% ===================== TITLE PAGE ==================================
% ===================================================================
\title{Blood Flow in the Human Circulatory System}
\author{Daniel Henderson, Michigan Technological University \\ \texttt{dahender@mtu.edu}}
\date{\today}



% ===================== BEGIN DOCUMENT ==============================
% ===================================================================
\begin{document}
    \maketitle


    % ------------------ COVER PAGE
    \noindent Report of modern techniques for modeling the motion of
    blood within a Human's Macrocirculatory System. \medskip
    \textbf{Keywords:} \textit{
        computational hemodynamics, 0D blood-flow, 1D blood-flow, 2D-blood-flow,
        PINN's, finite element methods, discontinous galerkin, Lax-Wendroff, fluid-structure ineraction (FSI)
    }
    \medskip \tableofcontents \newpage



    % ------------------ PRELIMINARIES
    \section{Preliminaries}
        \label{sec:prelims}

        \subsection{Notation}
            \label{subsec:notation} \begin{minipage}[t]{0.48\textwidth}
    \begin{tabular}{@{}p{0.4\linewidth} p{1.8\linewidth}@{}}
        $\therefore$ & consequently\\
        $\because$ & because\\
        $\implies$ & implies \\
        $\iff$ & if and only if \\
        $:=$ & defines\\
        $\equiv$ & equivalence \\
        $ \R$                             & set of real numbers \\
        $ \R^{+}$                         & set of positive real numbers \\
        $ \R^{-}$                         & set of negative real numbers \\
        $ \R^{n}$                         & n-dimensional real vector space \\
        $ \Omg \subset \R^{n}$          & a connected open subset of $\R^{n}$ \\
        $ \OmgBar$                      & the closure of $\Omg$ \\
        $ \partial \Omg$                & the boundary of $\Omg$ \\
        $C^{k}(\Omg)$                   & space of $k$ times continuously differentiable functions on $\Omg$ \\
        $C^{k}_{0}(\Omg)$               & space of $k$ times continuously differentiable functions with compact support in $\Omg$ \\
        $C^{k}_{0}(\overline{\Omg})$    & space of $k$ times continuously differentiable functions which have bounded and uniformly
                                            continuous derivatives up to order $k$ with compact support in $\Omg$ \\
        $C^{\infty}_{0}(\Omg)$          & space of smooth functions with compact support in $\Omg$ \\
        $L^{p}(\Omg)$                   & Lebesgue space of $p$-integrable functions on $\Omg$ \\
    \end{tabular}
\end{minipage}\hfill
\newpage
\begin{minipage}[t]{0.48\textwidth}
    \begin{tabular}{@{}p{0.4\linewidth} p{1.8\linewidth}@{}}
        $ dx $                            & Lebesgue measure on $\R^{n}$ \\
        $ dS $                            & surface measure on $\partial \Omg$ \\
        $ dV $                            & volume measure on $\Omg$ \\
        $ \nabla $                        & gradient operator \\
        $ \del = \nabla^2 = \nabla \cdot \nabla(\cdot)$ & Laplace operator \\
        $ \Div $                          & divergence of a vector field \\
        $ \Divbf $                        & divergence of a tensor \\
        $ \vct{v}_i$                      & $i$-th component of vector $\vct{v}$ \\
        $ \langle \cdot, \cdot \rangle_X$ & inner product on vector space $X$ \\
        $ \langle \vu, \vct{v} \rangle$ & inner product of vectors $\vu, \vct{v} \in \R^n$ \\
        $ \dfrac{\partial}{\partial \nhat} = \langle \nabla, \nhat \rangle$ & normal derivative on $\partial \Omg$ \\
        $\| \cdot \|$                     & $L^2$-norm \\
    \end{tabular}
\end{minipage}

        \newpage

        \subsection{Symbols and Abbreviations}
            \label{subsec:syms-abbrevs} \begin{minipage}[t]{0.48\textwidth}
    \begin{tabular}{@{}p{0.4\linewidth} p{1.8\linewidth}@{}}
        a.e. & almost everywhere\\
        e.g. & "exempli gratia" (for example)\\
        i.e. & "id est" (that means) \\
        s.s. & sufficiently smooth \\
        s.t. & such that \\
        r.t. & refers to \\
        w.r.t. & with respect to\\
        m.b.s. & must be shown \\
        i.m.b.s. & it must be shown \\
        i.r.t.s. & it remains to show \\
        w.a.t.s. & we aim to show \\
        bpm & beats per minute \\
        wlog & without loss of generality \\
        ODE & Ordinary Differential Equation \\
        PDE & Partial Differential Equation \\
        PDES & System of Partial Differential Equations \\
        IC & Initial Condition \\
        BC & Boundary Condition \\
        0D & Zero dimensional \\
        1D & One dimensional \\
        2D & Two dimensional \\
        3D & Three dimensional \\
        FSI & Fluid-Structure Interaction \\
        SB & Stenotic Blockage \\
        RBC & Red Blood Cell \\
        CVD & Cardiovascular disease \& CVDs r.t. such diseases\\
    \end{tabular}
\end{minipage}

        \newpage

        \subsection{Parameters and Units}
            \label{subsec:params} \begin{minipage}[t]{0.48\textwidth}
    \begin{tabular}{@{}p{0.4\linewidth} p{1.8\linewidth}@{}}
        $R$ & radius of vessel with diameter $2R$ \\
        $\eta$ & dynamic viscosity $\quad \units{Pa \cdot s}$ \\
        $\mu$ & kinematic viscosity $\quad \units{\frac{cm^2}{s}}$\\
        $\tau$ & shear stress \\
        $\dot{\gamma}$ & shear rate \\
        $\rho$ & density field $\quad \units{\frac{kg}{cm^3}}$\\
        $p$ & pressure field \\
        $\vu$ & velocity field $\quad \units{\frac{cm}{s}}$\\
        $W_0$ & Womersley number $\units{-}$ \\
        $Re$ & Reynolds number $\units{-}$ \\
        $Pe$ & Péclet number $\units{-}$ \\
        $c$ & concentration of a material element \\
        $D$ & diffusion coefficient $\quad \units{\frac{cm^2}{s}}$\\
        $t$ & time $\quad \units{s}$ \\
        $T$ & terminal time $\quad \units{s}, t > 0$ \\
        $\omega$ & angular frequency $\quad \units{\frac{rad}{s}}$ \\
        $\vct{f_b}$ & body force per unit volume $\quad \units{\frac{N}{cm^3}}$ \\
    \end{tabular}
\end{minipage}

        \newpage

        \subsection{Mathematical Foundations}
            \label{subsec:maths} % domain disscussion
We assume Zermelo-Fraenkel set theory with the axiom of choice (ZFC), and according to Cohen\href{https://en.wikipedia.org/wiki/Paul_Cohen},
it's necessary to also posulate the existence of the continuum hypothesis (CH).
A \emph{domain} will r.t. an open and bounded subset of $\R^N$ with nonempty interior $\Omg^\circ$,
for all $N\in\{1,2,3,4\}$. By the Heine-Borel theorem, every domain $\Omg$ has a well-defined boundary
$\partial \Omg$ with compact closure $\OmgBar = \Omg \cup \pOmg$. The compact domains of $\R^N$ are
precisely the closed and bounded subsets of $\R^N$.

% % simply connected, lipschitz domain, and convex domain
% When every path between two points in $\Omg$ may be continously contrated to a point without leaving
% $\Omg$, we say that $\Omg$ is \emph{simply connected}. A \emph{Lipschitz domain} is a domain $\Omg$
% s.t for every point $x\in\pOmg$, there exists a neighborhood $U_x$ of $x$ s.t $\Omg \cap U_x$ is the
% region above the graph of a Lipschitz continuous function (after a suitable rotation and translation
% of coordinates). A \emph{convex domain} is a domain $\Omg$ s.t for every $x,y\in\Omg$, the line segment
% connecting $x$ and $y$ is contained in $\Omg$.

\medskip

% function space discussion
Given domain $\Omega$, the function space $\mathcal{F}(\Omg)$ is a linear
space (sp.) $(F(\Omg), \, K)$ of scalar valued functions $f:\Omg \to K$; e.g. $F = C$ and $K = R$
is $C^0(\Omg)$, the sp. of continuous real valued functions on $\Omg$. Let $C^{k}(\Omg)$ be the
sp. of $k$-tides continuously differentiable functions on $\Omg$. Then $C^k(\OmgBar)$ is the sp. of functions
$f \in C^k(\Omg)$ s.t $f$ and it's derivatives up to order $k$ may be continuously extended to the boundary $\pOmg$.
The space $L^p(\Omg)$ denotes the Lebesgue sp. of $p$-integrable functions on $\Omg$. If We denote
the space of Lipschitz continous functions on $\Omega$ as $Lip(\Omg)$.\todo{relate to $C(\Omg)$, cite brenners book}.


\medskip

% more on sp.s: topological to metric to normed to inner-product sp.,
%   then completeness and banach sp. discussion; Hilbert sp.s are banach.

\medskip

% definition for scalar feilds
For real-valued $f \in C^1(\Omg)$, the partial derivative w.r.t. the $i$-th coordinate of $\Omg$ is
\begin{flalign*}
    \quad \pd_i f = \frac{\pd f}{\pd x_i}, &&
\end{flalign*}
the gradient of $f$ is $\nabla f = \left(\pd_1 f, \ldots  \pd_N f\right)$
and the Laplacian of $f$ is $\Delta f = \sum_{i=1}^N \pd_i^2 f.$
\todo{introduce $C_0$}
\medskip

The integral of $f \in L^1(\Omg)$ is
\begin{flalign*}
    \quad \int_{\Omg} f(\vx) \, \dx \; \equiv \; \int_\Omg f ~ .&&
\end{flalign*}
\todo{introduce $L_{loc}$}
For a vector field $\vct{F} = (F_1, \ldots, F_N) \in C^1(\Omg; \R^N)$,
the divergence is defined as $\Div \vct{F} = \sum_{i=1}^N \pd_i F_i$.

\medskip

\td{Work In Progress (WIP) -- Complete Last. Incoporate following

\begin{flalign*}
    \quad & \phi:\Omg(t)\to\R \;\text{ r.t. a \emph{scalar field}}, & \\
    \quad & \vct f:\Omg(t)\to\R^N \;\text{ r.t. a \emph{vector field}}, &&\\
    \quad & \vct T:\Omg(t)\to\R^{N\times N} \;\text{ r.t. a \emph{(second-order) tensor field}}. &&
\end{flalign*}
The tensor $\vct T$ defines a linear map from $\R^N \to \R^N$, and once a basis is fixed,
$\exists ~ A \in \R^{N\times N}$ s.t. $\vct T(\vct x) = A \,\vct x$ for all $\vct x \in \Omg(t)$.}

        \newpage

    \section{Introduction}
        \label{sec:intro}
        Hemodynamics studies the kinematics of blood. Our interest is the kinematic motion
        of blood within the Human macrocirculatory system, i.e. the flow of blood in large
        vessels such as arteries and veins. Blood is observed as a complex fluid of formed elements
        suspended in plasma, thus, the rheological behavior of blood is non-trivial. We report techniques
        and methodologies for modeling bloods' motion in the Human macrocirculatory system.

        \medskip

        After stating the motivation of our study, Sec.~\ref{subsec:pysiology} provides a physiological review of a Human's circulatory system in
        Sec.~\ref{subsec:continuum} states the continuum hypothesis, a necessary posulate of
        fluid mechanics. We adopt the continuum hypothesis and treat blood as a continuous medium, then our hemodynamics
        problem simplifies describing the motion of a continous media. In Sec.~\ref{subsec:bloodmodel}, we discuss assumptions
        one may impose on the material-rheological properties of blood. By adopting a particular rheological model, we obtain the necessary constitutive
        relations for a mathematical description of bloods motion. Treating blood as a simple fluid (single-contitute, homogenous, and isotropic
        mixture) yeilds a Newtonian rheological model of blood (i.m.b. justified such models are valid in large vessels, e.g.
        arteries and veins, where the shear rates are sufficiently high) %[\cite{fung1997biomechanics}, Ch.~2]).
        Assuming the density of a material element of blood remains constant as it flows within the vessel, is
        r.t. the incompressibility condition. Together, these assumptions result in an incompressible-newtonian rheological model of blood.
        Finally, in Sec.~\ref{subsec:ns} the Navier-Stokes (NS) system of Partial Differential Equations (PDEs) governing
        the motion of an incompressible-newtonian continuum of blood are derived. The NS equations serve
        as the foundation for all subsequent modeling techniques reported herein.

        \medskip

        \paragraph{Motivation}
        Coronary artery stenosis (CAS) is the narrowing of the coronary arteries due
        to the buildup of plaque. Such narrowing can restrict blood flow to the heart muscle,
        which may lead to various cardiovascular problems. Current methods for predicting a stenotic
        blockage (SB) in a coronary artery are rudimentary, and often SB predicition doesn't mean obstruction [\cite{ehab332}].
        Additionally, current clinical methods for assessing the severity of a SB rely on imaging techniques
        such as angiography, intravascular ultrasound (IVUS), and optical coherence tomography (OCT) to visualize the 
        arteries and identify areas of narrowing. Such methods provide valuable information about the anatomy of the arteries, but they do not
        provide direct information about the functional significance of the CAS. Functional assessment of CAS typically involves measuring the fractional flow reserve (FFR),
        which is the ratio of the blood pressure downstream of the stenosis to the blood pressure upstream of the stenosis during maximum blood flow.
        However, measuring FFR requires the use of a pressure wire, which can be invasive and carries some risks.
        Therefore, there is a need for non-invasive methods to assess the functional significance of CAS.

        
        \newpage

        \subsection{Pysiology}
            \label{subsec:pysiology} \input{intro/pysiology.tex}
        \newpage

        \subsection{Continuum}
            \label{subsec:continuum} W.a.t. simulate blood flow in a time-dependent fluid domain $\OmgB \subset \R^{N+1}$;
we'll take $N \equivto 3$, the natural setting for the derivations that follow.
For each $t \in I_T \defined [0, T] \subset \R$, with $T > 0$, we define
\begin{flalign*}
    \quad \OmgBt &:= \{\vx \in \R^N : \vx \text{ lies inside the vessel at time } t\}. &&
\end{flalign*}
Formally, the fluid region is a time-dependent family of open sets
$\{\OmgBt\}_{\forallin{t}{I_T}} \subset \R^N$ occupied by blood at time $t$.
Let $\OmgB(0)$ r.t. the \emph{reference} configuration and $\OmgBt$
r.t. the \emph{current} configuration.
The fluid's computational domain r.t. all possible configurations of $\OmgBt$, i.e.,
the spatial-temporal lumen
\begin{flalign*}
    \quad & \OmgB \defined I_T \times \OmgBt \equivto \{ (t, \vx) \in \R^{N+1} ~:~ x \in \Omgt, ~ t \in I_T \}. &&
\end{flalign*}

\medskip

For each material particle initially at $\vxi \in \Omega_B(0)$,
we assume its motion in $\Omega_B$ is governed by a velocity field
$\vu : \Omega_B \to \R^N$.  Our aim here is to derive the governing PDE system
for $\vu$—the Navier–Stokes equations.
\medskip

\subsubsection{Blood Dynamics}
    \label{subsubsec:blood-dynamics}

    % Coordinate defs
    Let the flow map be
    \begin{flalign*}
        \quad & \vx: I_T \times \Omega_B(0) \to \R^N, \quad (t, \vxi) \mapsto \vx(t, \vxi) &&
    \end{flalign*}
    For each fixed $t \in I_T$, the \emph{Lagrangian map} is from the reference configuration to the current configuration, i.e.
    \begin{flalign*}
        \quad & \La_t : \Omega_B(0) \to \Omega_B(t), \quad \vxi \mapsto \vx(t, \vxi) \quad \text{ s.t. }
        \quad \La_t(\vxi) \equals \vx(t, \vxi). &&
    \end{flalign*}
    The variables $(t,\vx)$ and $(t,\vxi)$ are referred to as the Eulerian and Lagrangian
    coordinates, resp..

    \medskip

    \begin{remark}[Eulerian vs. Lagrangian]
        \label{rmk:eulerian-lagrangian}
        Informally, the Eulerian approach focuses our attention to $\vx \in \OmgBt$, namely,
        some fluid particle located at $\vx$ at a particular time $t$.
        Whereas, the Lagrangian approach tracks an individual fluid particle $\vxi \in \OmgB(0)$
        along it's trajectory $\vT_{\vxi} \defined \{ \vx(t,\vxi) : t \in I_T \}$ as it moves along a
        characteristic curve in $\OmgB$.
        \medskip
        For $\phi \in C^1(\OmgB)$ s.t. $(t, \vx) \mapsto \phi(t, \vx)$ in the Eulerian variables,
        we'll let $\hat{\phi}(t,\vxi) \defined \phi\bigl(t,\La_t(\vxi)\bigr)$ s.t. $\vx(t) = \La_t(\vxi)$. I.e.,
        we'll let $\phihat \equals \phi \circ \La_t$ and it follows that $\phi \equals \phihat \circ \inv{\La_t}$.
    \end{remark}

    \medskip

    % Principle kinematic quantity: velocity def. via Lagrangian Reference Frame
    \begin{definition}
        \label{def:lagrangian-velocity}
        Let the Lagrangian velocity field be defined as
        \begin{flalign*}
            \quad \vuhat \defined \pd_t \vx(t, \vxi) \mapsto \vuhat(t, \vxi)
                \quad \forallin{(t, \vxi)}{I_T \times \OmgB(0)} &&
        \end{flalign*}
    \end{definition}
    A change of coordinates in~\ref{def:lagrangian-velocity} yields the velocity field $\vu$
    in the Eulerian frame, i.e.,
    \begin{flalign*}
        \quad \vu \equals \vuhat \circ \inv{\La_t} \quad \iff \quad \vu(t, \vx)
            \equals \vuhat(t, \inv{\La_t}(\vx)) &&
    \end{flalign*}
    Then if $\vuhat$ and $\OmgB(0)$ are known, we arrive at the Cauchy problem
    for the trajectory $T_{\vxi}$ of each material particle $\vxi \in \OmgB(0)$:
    \begin{flalign*}
        \quad \begin{cases*}
            \pd_t{\vx}(t, \vxi) \equals \vuhat(t, \vxi), \forallin{t}{I_T} \\
            \vx(0, \vxi) \equals \vxi.
        \end{cases*} &&
    \end{flalign*}

    \td{State conditions for well-posedness in the sense of hadamard? Explain existence, uniqueness, stability of solutions of the resulting
    ODE system above via the Picard-Lindelof theorem.}


    \medskip

\subsubsection*{Material Derivatives}

    \begin{definition}
        \label{def:material-derivative}
        Let $\phi \in C^k(\OmgB)$ and $\phihat \equals \phi \circ \La_t$, then the \emph{material derivative}
        $D_t \phi$ is the derivative of $\phi$ w.r.t. $t$ in the Lagrangian frame expressed as a function in the Eulerian frame, i.e.,
        \begin{flalign*}
            \quad D_t \big( \phi(t, \vx) \big) &\defined \pd_t \phihat(t, \vxi), \quad \text{ s.t. } \vxi \equals \inv{\La_t}(\vx) &&\\
                & \equals \dfrac{d}{dt}\phi(t, \vx(t, \vxi)), \quad \forallin{\vxi}{\OmgB(0)}. &&
        \end{flalign*}
    \end{definition}
    By the multivariable chain rule,
    \begin{flalign*}
        \quad D_t \big( \phi(t, \vx) \big) &\equals \pd_t \phi(t,\vx) + \nabla \phi(t,\vx)\cdot \vu(t,\vx) &&\\
            &\equals \pd_t \phi + \inp{\nabla \phi}{\vu}_{\R^N} \quad \text{ in } \OmgB &&
    \end{flalign*}
    where one may see that the material derivative $D_t \phi$ measures the rate of variation of $\phi$ along trajectory $T_{\vxi}$
    expressed in the Eulerian frame. The literature also r.t. $D_t$ as the \emph{substantial derivative}, \emph{advective derivative}, \emph{lagrangian derivative}, or
    \emph{convective derivative}.

\medskip

Now differentiating $\vuhat$ w.r.t. $t$ yields the acceleration vector field $\vahat$ in the Lagrangian and Eulerian frame
repsectively:
\begin{flalign*}
    \quad \vahat(t, \vxi) &\defined \pd_t \vuhat(t, \vxi) = \pd_t^2 \vx(t, \vxi) \quad  \forallin{\vxi}{\OmgB(0)} &&\\
        &\iff \va \equals D_t \vu \equals \pd_t \vu + (\vu \cdot \nabla) \vu &&\\
        \va(t, \vx) &\equals \pd_t \vu(t, \vx) +  (\vu(t,\vx)\cdot\nabla)\vu(t,\vx) &&\\
        \va(t, \vx) &\equals \pd_t \vu(t, \vx) + \sum_{i=1}^{N} u_i(t, \vx) \pd_{x_i} \vu(t, \vx) \quad \forallin{(t, \vx)}{\OmgB}. &&
\end{flalign*}

\medskip
Let's denote the deformation gradient tensor $\vFhat_t$ and its counterpart $\vF_t$ be the gradient of $\La_t$ and $\inv{\La_t}$ in,
defined respectively as:
\begin{flalign*}
    \quad \vFhat_t(\vxi) &\defined \nabla_{\vxi} \La_t(\vxi) \quad \forallin{\vxi}{\OmgB(0)}. &&\\
        &\iff \vF_t(\vx) \equals \nabla_{\vxi} \La_t(\inv{\La_t}(\vx)) \equals \pd_{\vxi} \vx(t, \vxi) \quad \forallin{(t, \vx)}{\OmgB}. &&
\end{flalign*}
The tensor $\vF_t$ measures the spatial deformation of infinitesimal fluid elements as they move from
the reference configuration to the current configuration under $\La_t$ along trajectory $T_{\vxi}$.
We assume that, for each fixed $t$, the mapping
$\La_t : \OmgB(0) \to \OmgB(t)$ is a $C^1$–diffeomorphism with $\det \vFhat_t(\vxi) > 0 \forallin{\vxi}{\OmgB(0)}$\footnote{
    Recall the $\det \vFhat$ may be interpreted as the signed volume change of an infinitesimal parallelepiped
    spanned by the column vectors of $\vFhat$; thus, $\det \vFhat_t(\vxi) > 0$ ensures that the orientation of fluid elements
    is preserved under the deformation mapping $\La_t$ at time $t$. Moreover, it prevents unphysical scenarios
    such as interpenetration of matter consisting of negative volumes.
}.
Then $\vFhat_t(\vxi)$ is invertible, and we define
\begin{flalign*}
    \quad \Jhatdet_t(\vxi) & \defined \det\big(\vFhat_t(\vxi)\big) \quad \forallin{\vxi}{\OmgB(0)} &&\\
        &\iff \Jdet_t(\vx) \equals \det\big(\vFhat_t(\inv{\La_t}(\vx))\big) \quad \forallin{(t, \vx)}{\OmgB} &&
\end{flalign*}
We r.t. such measure of spatial deformation $\Jhatdet_t(\vxi)$ as the Jacobian determinant of $\La_t$ at
time $t$. The following lemma shows that $\pd_t \Jdet_t$ relates to the fluids divergence $\Div(\vu)$.

\medskip

\begin{lemma}
    \label{lem:jacobian-determinant-derivative}
    The Jacobian determinant $\Jdet_t(\vx)$ satisfies
    \begin{flalign*}
        \quad & D_t \Jdet_t(\vx) \equals \Jdet_t(\vx) \, \Div(\vu(t, \vx)) \quad \forallin{(t, \vx)}{\OmgB}. &&
    \end{flalign*}
    \begin{proof}
        Ref~\cite{book2}, lemma 2.1, pg. 20, for proof.
    \end{proof}
\end{lemma}

Literature often r.t. relation in lem.~\ref{lem:jacobian-determinant-derivative} as the
\emph{Euler expansion formula}, or \emph{Jacobi's equation}.

\medskip

% Reynolds Transport
Let $V(t)$ denote a measurable material volume at time $t$ that moves with the fluid flow, i.e.,
\begin{flalign*}
    \quad V(t) &\subset \OmgBt, \quad \text{for each } t \in I_T &&
\end{flalign*}
Equivalently, $V(t)$ is the image of some reference volume $V(0) \subset \OmgB(0)$ under the
Lagrangian mapping $\La_t$, i.e.,
\begin{flalign*}
    \quad V(t) & = \La_t\big(V(0)\big). &&
\end{flalign*}
The following theorem, known as the Reynolds Transport Theorem, relates the time derivative of an integral
over the moving volume $V(t)$ to integrals over $V(t)$ and its boundary $\p V(t)$.
\begin{theorem}[Reynolds Transport Theorem]
    \label{thm:reynolds-transport}
    Let $V(0) \subset \OmgB(0)$ be a material volume in the reference configuration,
    and $V(t) \subset \OmgBt$ its image in the current configuration under $\La_t$. For any
    sufficiently smooth scalar field $\phi: \OmgB \to \R$,
    \begin{flalign*}
        \quad & \frac{d}{dt} \int_{V(t)} \phi(t, \vx) \, dV \defined \int_{V(t)} \pd_t \phi(t, \vx) \, dV
            + \int_{\p V(t)} \phi(t, \vx) \, \inp{\vu}{\nhat} \, dS, &&
    \end{flalign*}
    where $\vu$ is the fluid velocity field and $\nhat$ the outward unit normal on $\p V(t)$.
\end{theorem}
\begin{proof}
    Ref~\cite{book2}, theorem 2.2. pg. 21.
\end{proof}

\subsection{NS-derivation}
In the following derivations we let $N \equiv 3$ so that $\OmgBt \subset \R^3$. We assume blood fluid is
a continuum that deforms continuously, i.e., at every point $\vx \in \OmgBt$ and time $t \in I_T$, the
blood's kinematic quantities are described by sufficiently smooth fields. At microscopic scales this
continuum hypothesis breaks down, since matter is a discrete collection of molecules, but at macroscopic
scales empirical evidence suggests such models remain accurate.

\medskip

For each $t \in I_T$, herein assume $\OmgBt$ is simply connected and boundary $\pOmgB(t) \in C^1$,
so, any closed curve in $\OmgBt$ can be continuously contracted to a point within $\OmgBt$; such an
assumption is reasonable in a healthy vessel with a smooth wall. This regularity allows us to globally
define geometric quantities such as surface differential operators on the fluids boundary $\pOmgB$, then
we may apply stokes theorem and Green's identities on the $C^1$ boundaries, useful for obtaining weak
integro-differential and variational formulations.

\medskip

\begin{remark}
    A weaker regularity condition is that $\pOmgB$ is Lipschitz. Here we must
    rely on standard trace theorems for Sobolev spaces (e.g.\ $H^1(\OmgBt)$). In particular,
    the outward unit normal is then defined a.e. on $\pOmgB(t)$, so we can meaningfully speak of
    normal and tangential components of vector fields on the boundary.
\end{remark}

\medskip
Let blood's velocity, thermodynamic pressure, and density fields be
\begin{flalign*}
    \quad & \vu: \OmgB \to \R^3,\;
        (t, x, y, z) \mapsto (u_1(t, x, y, z),\, u_2(t, x, y, z),\, u_3(t, x, y, z))^\top,
        \quad \units{\frac{cm}{s}} && \\
    \quad & p: \OmgB \to \R^+, \;
        (t, x, y, z) \mapsto p(t, x, y, z),
        \quad \units{\frac{N}{cm^2}} && \\
    \quad & \rho: \OmgB \to \R^+,\;
        (t, x, y, z) \mapsto \rho(t, x, y, z),
        \quad \units{\frac{kg}{cm^3}}. &&
\end{flalign*}
\begin{definition}[Incompressible fluid]
    \label{def:incompressible-fluid}
    A fluid is \emph{incompressible} if for each material element $V(0)\subset \Omega_B(0)$,
    the density of that element remains constant in time, i.e.
    \begin{flalign*}
        \quad \rho\bigl(t,\vx(t,\vxi)\bigr) \equals \rho_{V(0)} \in \R^+, \quad  \forallin{(t, \vxi)}{I_T \times V(0)}, &&
    \end{flalign*}
    where constant $\rho_{V(0)}$ can depend on initial fluid element $V(0)$.
\end{definition}
\begin{definition}[Incompressible flow]
    \label{def:incompressible-flow}
    A flow is \emph{incompressible} in $\OmgB$ if the material derivative of the density vanishes,
    \begin{flalign*}
        \quad D_t \rho = 0 \quad \text{in } \OmgB. &&
    \end{flalign*}
\end{definition}
In our model we further assume a uniform initial density, so def~\ref{def:incompressible-fluid} may be written as
\begin{flalign*}
    \quad & \rho_{V(0)} \equals \rho_0 \in \R^+ \quad \forallsubsets{V(0)}{\OmgB(0)}, &&
\end{flalign*}
for some constant $\rho_0$ independent of material element $V(0)$.
Therefore uniform density in the reference configuration implies constant density along
trajectory $T_{\vxi}$ to any current configuration, i.e.,
\begin{flalign*}
    \quad  \rho\bigl(t,\vx(t,\vxi)\bigr) & \equals \rho_0 \in \R^+, \quad  \forallin{(t, \vxi)}{I_T \times \OmgB(0)}. && \\
    \quad & \implies \rho(t, \vx) \equals \rho_0 \in \R^+, \quad  \forallin{(t, \vx)}{\OmgB}. &&
\end{flalign*}
Taking the material derivative yields $D_t \rho = 0$ in $\OmgB$ -- In summary, we've shown:
\begin{flalign*}
    \quad \text{Incompressibile Fluid~\ref{def:incompressible-fluid} }
        ~ \implies ~ \text{Incompressible Flow~\ref{def:incompressible-flow}}. &&
\end{flalign*}
Note, the converse is not generally true: an incompressible flow ($D_t\rho=0$) only preserves density
along $T_{\vxi}$, allowing $\rho=\rho(\vx)$ to vary spatially in the Eulerian frame. But particularly
here, uniform initial density in $\OmgB(0)$ implies constant density in $\OmgB$ under incompressible flow.
\newpage
\begin{remark}[Divergence-free condition]
    \label{rmk:incompressibility-divergence-free}
    Mass conservation for a fluid with density $\rho$ and velocity $\vu$
    yields the continuity equation
    \begin{flalign*}
        \quad \partial_t \rho + \Div(\rho \vu) = 0
        \quad \Longleftrightarrow \quad
        D_t \rho + \rho\,\Div \vu = 0
        \quad \text{in } \Omega_B. &&
    \end{flalign*}
    For an incompressible fluid with constant density $\rho \equiv \rho_0$,
    Definition~\ref{def:incompressible-fluid} implies $D_t \rho = 0$, so
    \begin{flalign*}
        \quad 0 = D_t \rho + \rho_0 \Div \vu
        \quad \Longrightarrow \quad \Div \vu = 0
        \quad \text{in } \Omega_B. &&
    \end{flalign*}
    Combining this with Lemma~\ref{lem:jacobian-determinant-derivative},
    \begin{flalign*}
        \quad D_t J_t = J_t \Div \vu, &&
    \end{flalign*}
    we obtain $D_t J_t = 0$, hence $J_t(\vx)$ is constant along material
    trajectories. With the natural choice $\La_0 = \mathrm{Id}$ we have
    $J_0 \equiv 1$, so incompressible flow is equivalent to
    \begin{flalign*}
        \quad & J_t(\vx) \equiv 1 \quad \text{and} \quad \Div \vu = 0. &&
    \end{flalign*}
    We refer to $\nabla \cdot \vu = \Div \vu = 0$ as the \emph{divergence-free condition} in $\Omega_B(t)$.
\end{remark}
By assumption blood is incompressible and newtonian, linear momentum balance follows from Newton's 2nd Law ($F = ma$)
for an infinitesimal fluid element $V(t) \subset \OmgBt$.
\begin{definition}
    \label{def:linear-momentum-balance}
    The linear momentum balance for a fluid element $V \in \OmgB$ under flow $\vu$ is
    \begin{flalign*}
        \quad & \frac{d}{dt} \int_{V(t)} \rho \, \vu \, dV \defined \int_{V(t)} \Divbf(\vT) \, dV + \int_{V(t)} \rho \, \vfb \, dV,
        \quad (\because ~ \rho \equiv \rho_0) &&
    \end{flalign*}
    where $\vT$ is the Cauchy stress tensor describing the fluids deformation, $\vfb$ may be some body force per unit mass $\units{\dfrac{cm}{s^2}}$.
\end{definition}
By applying the Reynolds Transport Theorem~\ref{thm:reynolds-transport} to~\ref{def:linear-momentum-balance},
we obtain the strong form of linear momentum balance in $\OmgB$:
\begin{flalign*}
    \quad & \int_{V(t)} \rho \, \pd_t \vu \, dV + \int_{\p V(t)} \rho \, \vu \, \inp{\vu}{\nhat} \, dS \defined \int_{V(t)} \Divbf(\vT) \, dV + \int_{V(t)} \rho \, \vfb \, dV, &&
\end{flalign*}
for any material volume $V(t) \subset \OmgBt$.
By the divergence theorem, we may rewrite the surface integral as a volume integral
\begin{flalign*}
    \quad & \int_{V(t)} \rho \, \pd_t \vu \, dV + \int_{V(t)} \Div(\rho \, \vu \otimes \vu) \, dV \defined \int_{V(t)} \Divbf(\vT) \, dV + \int_{V(t)} \rho \, \vfb \, dV, &&
\end{flalign*}
Since $V(t)$ is arbitrary, the integrands must be equal a.e. in $\OmgBt$, yielding differential form of linear momentum balance:
\begin{flalign*}
    \quad & \rho \, \pd_t \vu + \Div(\rho \, \vu \otimes \vu) \defined \Divbf(\vT) + \rho \, \vfb, \qquad \text{in} ~ \OmgBt &&
\end{flalign*}
By incompressibility assumption of blood fluid~\ref{def:incompressible-fluid} we have $\rho \equiv \rho_0$ constant in $\OmgBt$,
so the linear momentum balance simplifies to
\begin{flalign*}
    \quad & \rho \, \pd_t \vu + \rho \, \Div(\vu \otimes \vu) \defined \Divbf(\vT) + \rho \, \vfb, \qquad \text{in} ~ \OmgBt &&
\end{flalign*}
Using the vector calculus identity $\Div(\vu \otimes \vu) = (\nabla \cdot \vu) \, \vu + (\vu \cdot \nabla) \vu$ we have
\begin{flalign*}
    \quad \Div(\vu \otimes \vu) &= \Divbf\big(\vu \vu^\top\big) &&\\
    \quad &\equals \big(\nabla \cdot \vu\big) \, \vu + (\vu \cdot \nabla) \vu &&\\
    \quad &\equals 0 \cdot \vu + (\vu \cdot \nabla) \vu \quad (\because \Div \, \vu \equals 0 \text{ by remark~\ref{rmk:incompressibility-divergence-free}}) &&\\
    \quad \implies& \rho \, \pd_t \vu + \rho \, (\vu \cdot \nabla) \vu \defined \Divbf(\vT) + \rho \, \vfb, \qquad \text{in} ~ \OmgBt &&
\end{flalign*}
so the linear momentum balance~\ref{def:linear-momentum-balance} for incompressible fluid simplifies to the following form.
\begin{definition}
    \label{def:linear-momentum-balance-strong}
    The differential form of linear momentum balance for an incompressible fluid element $V \in \OmgB$ under flow $\vu$ is
    \begin{flalign*}
        \quad & \rho(\pd_t \vu + (\vu\cdot\nabla)\vu ) \defined \Divbf(\vT) + \rho \vfb, \qquad \text{in} ~ \OmgBt &&
    \end{flalign*}
\end{definition}

\medskip

The linear momentum balance equation above is not closed since the Cauchy stress tensor $\vT$ is
unknown. To close the system, we need a constitutive law relating $\vT$ to kinematic quantities such as
the velocity field $\vu$ and its derivatives.

\medskip

Let the rate-of-deformation tensor be defined as the symmetric
part of the velocity gradient, i.e.,
\begin{flalign*}
    \quad \vct{D}(\vu) \defined \tfrac{1}{2}\big(\nabla\vu+(\nabla\vu)^\top\big)
    \quad \text{s.t.} \quad \nabla\vu \defined \vect{
        \dfrac{\pd u_1}{\pd x} & \dfrac{\pd u_1}{\pd y} & \dfrac{\pd u_1}{\pd z} \\
        \dfrac{\pd u_2}{\pd x} & \dfrac{\pd u_2}{\pd y} & \dfrac{\pd u_2}{\pd z} \\
        \dfrac{\pd u_3}{\pd x} & \dfrac{\pd u_3}{\pd y} & \dfrac{\pd u_3}{\pd z}
    }. &&
\end{flalign*}
Note that $\vu \mapsto \vct{D}(\vu)$ captures spatial deformations of element $V$ in $\OmgB$ under flow $\vu$.
\begin{definition}
    \label{def:newtonian-fluid}
    A fluid is \emph{Newtonian} if its Cauchy stress tensor $\vT\;$ depends
    linearly on the rate-of-deformation tensor $\vct{D}(\vu)$.
\end{definition}
\begin{definition}
    \label{def:isotropic}
    A fluid is \emph{isotropic} if its constitutive response is independent of the coordinate system.
    Writing the Cauchy stress as $\vT\;=\vT\;(\vct{D})$,
    isotropy means that for every orthogonal rotator $\vct{Q}\in\mathrm{SO}(3)$,
    \begin{flalign*}
        \quad & \vct{Q}\, \vT(\vct{D})\,\vct{Q}^{\top} \defined \vT\;\!\big(\vct{Q}\,\vct{D}\,\vct{Q}^{\top}\big). &&
    \end{flalign*}
\end{definition}
Where $SO(3) \defined \{\vct{Q} \in \R^{3 \times 3} : \vct{Q}\,\vct{Q}^{\top} = \mathrm{I}, ~ \det(\vct{Q}) = 1\}$ is the special orthogonal group in $\R^3$.
\begin{remark}[Isotropy implication]
    \label{rmk:isotropy-implication}
    By \ref{def:isotropic}, the Cauchy stress $\vT$ of an isotropic fluid depends only on
    the invariants of $\vct{D}(\vu)$, i.e., the eigenvalues of $\vct{D}(\vu)$.
\end{remark}
\begin{remark}[Simple vs. complex fluids]
    In the terminology of continuum mechanics, a \emph{simple} fluid is one whose
    constitutive response at a point depends only on the instantaneous values of
    kinematic quantities (such as the rate-of-deformation tensor $\vct{D}(\vu)$),
    and not on their past history. The Newtonian, isotropic model in
    Definition~\ref{def:fluidsmodel} is a prototypical simple fluid.
    More complex models (e.g.\ viscoelastic fluids) incorporate memory effects
    so that the stress depends on the deformation history.
\end{remark}
\begin{remark}[Blood isotropy justification]
    \label{rmk:blood-isotropy-justification}
    Blood is often modeled as an isotropic fluid since its microstructure (RBCs, WBCs, platelets)
    are suspended in plasma and distributed uniformly in all directions at macroscopic scales.
    This uniform distribution leads to isotropic mechanical properties, meaning blood's response
    to deformation is independent of direction.
    \red{[\cite{book1}, sec. 3.1]} (Check source)
\end{remark}
\medskip
\begin{definition}[Newtonian, isotropic constitutive law]
    \label{def:fluidsmodel}
    For a Newtonian, isotropic fluid the Cauchy stress is
    \begin{flalign*}
        \quad & \vT \equals -\,p\,\mathrm{I} + 2\,\eta\,\vct{D}(\vu) + \lambda\,\Div(\vu)\,\mathrm{I}, &&
    \end{flalign*}
    where the bulk viscosity $\lambda \in \R^+$ is a material parameter quantifying the fluid's resistance
    to uniform compression.
\end{definition}
So newtonian isotropic fluids at rest (quiescent state $\vu\equiv\vct{0}$) sustains only hydrostatic stress:
\begin{flalign*}
    \quad \implies \vT \equals -\,p\,\mathrm{I} \quad \text{when} ~ \vu \equiv \vct{0}. &&
\end{flalign*}
Consequently, the constitutive law for newtonian fluids~\ref{def:fluidsmodel} simplifies under incompressibity assumption,
because then $\Div(\vu) \equals 0$ in $\OmgB$ (refer to~\ref{rmk:incompressibility-divergence-free}).
\begin{definition}[Incompressible stress tensor]
    \label{def:improssible-stress-tensor}
    The Cauchy stress~\ref{def:fluidsmodel} simplifies to
    \begin{flalign*}
        \quad \vT ~ = ~ -\,p\,\mathrm{I} + 2\,\eta\,\vct{D}(\vu). &&
    \end{flalign*}
\end{definition}
\begin{remark}[Dynamic vs. Kinematic Viscosity]
    In our model assumptions, \emph{dynamic viscosity} $\eta \in \R^+$ and  \emph{kinematic viscosity} $\mu \in \R^+$
    relate as
    \begin{flalign*}
        \quad \mu\; \defined\; \frac{\eta}{\rho} ~ = ~  \frac{\eta}{\rho_0} \; \in \; \R^+. &&
    \end{flalign*}
    Here $\eta$ quantifies the internal resistance of blood to shear deformation, i.e., $\eta \defined \dfrac{\tau}{\dot{\gamma}}$,
    with units $\units{Pa \cdot s}$. Moreover $\mu$ adjusts $\eta$ by the density $\rho$, capturing the viscous diffusion of momentum per unit mass, with units $\units{\frac{cm^2}{s}}$.
    Intuitively, $\eta$ measures how "thick" or "sticky" the fluid is, while $\mu$ measures how quickly momentum diffuses through the fluid due to viscosity.
\end{remark}

\medskip

\begin{remark}[Newtonian Blood Justification]
    \label{rmk:newtonian-justification}
    When diameter $d$ and hematocrit effects are needed, one may use a Non-Newtonian
    model with relative viscosity $\eta_r(H,d)$ that scales an absolute baseline $\eta$:
    \begin{flalign*}
        \quad \eta_{\mathrm{eff}} \defined \eta_r(H,d)\,\eta \quad \text{(effective viscosity)}&&
    \end{flalign*}
    An empirical fit from \cite{bviscosity}
    \begin{flalign*}
        \quad \eta_r \defined 1 + (\eta_{0.45}-1)\,
        \frac{(1-H)^{C}-1}{(1-0.45)^{C}-1}
        \text{ s.t. }
        \begin{cases}
            \eta_{0.45} = 6\,e^{-0.085\,d} + 3.2 - 2.44\,e^{-0.06\,d^{0.645}},\\[3pt]
            C = \big(0.8 + e^{-0.075\,d}\big)\!\left(\dfrac{1}{1 + 10^{-11}\,d^{12}} - 1\right)
                + \dfrac{1}{1 + 10^{-11}\,d^{12}},
        \end{cases} &&
    \end{flalign*}
    where $d \defined 2R/(1.0 \mu m)$ is the (scaled) vessel diameter.
    In large vessels, $\eta_r$ is often constant, justifying the Newtonian assumption.
    [\cite{book1}, sec. 3.1]
\end{remark}
\medskip
When modeling non-Newtonian effects (when $\eta \neq \text{ constant}$), the kinematic viscosity
$\mu(\cdot)$ is often chosen by Carreau model \cite{hemodynamics}
\begin{flalign*}
    \quad 2\,\mu(|\vct{D}|^2) = \eta_\infty + (\eta_0 - \eta_\infty) \cdot \big(1 + \kappa |\vct{D}|^2\big). &&
\end{flalign*}
Where $\eta_0$ and $~\eta_\infty$ are chosen to be the viscosity for very small
and very large shear rates, resp., and $\kappa \in \R^+$ and $n \in (-0.5, 0)$ are model parameters.
According to [\cite{book1}, pg. $38$], we often set
\begin{flalign*}
    \quad \eta_0 = 65.7 \cdot 10^{-3} ~Pa \cdot s, ~ \eta_\infty = 4.45\cdot10^{-3} ~Pa \cdot s,
        \kappa = 212.2 ~s^2, ~ \text{ and } n = -0.325 &&
\end{flalign*}
In the Newtonian case, we choose $\eta = \eta_\infty$ which allows us to determine $\mu$
as $\mu(|\vct{D}|^2) = \eta$.

        \newpage

        \subsection{Blood Model}
            \label{subsec:bloodmodel} One chooses a model based upon the specific application, computational resources,
and desired accuracy. We construct models of blood flow in
various geometries, starting from a single vessel, then extending our approach to
bifurcations and arterial networks. Our strategy involves a \emph{domain decomposition approach}.

\medskip

We seek solutions to inital and boundary value problems of Eq.~\ref{def:NSEquations}.
\begin{definition}
    \label{def:init_condition}
    Let $\vct{u_0} : \Omg(0) \to R^3, ~ \vct{x} \mapsto \vct{u_0}(x) ~ : ~ \vu(0, \vct{x}) = \vct{u_0}$.
    We refer to $\vct{u_0}$ as the initial condition of velocity feild $\vu$
\end{definition}
\begin{definition}
    Let $\vct{u^0}(t)$ and $\vct{u^1}(t)$ be the velocity feilds at $S_0$ and $S_1$.
\end{definition}
In practice, $\vct{u_0}(z)$ may be prescribed or determined from sensor data.

\subsection{Dimension-Reduced Models of Blood Flow}
    \label{subsec:dim-reduced-models}

    We start our disscusion with 1D and 0D models, reducing the
    d.o.f. in the NS system~\ref{def:NSEquations} by imposing further
    simplifying assumptions. Namely, w.a.t. compute average pressures
    and velocities after a relatively short simulation time by solving
    the 1D NS system in a compliant vessel with suitable side conditions.
    Because our model averages pressure and velocity over a surface-area, we obtain
    a uniform distribution of WSS on the vessel.

    \medskip

    One may start by introducing a rigid-vessel assumption, which leads to a no slip condition
    that $\vu \big|_{\pOmgt} \; = \; \vct{0}$. Instead we seek
    to model the link between blood flow and the deformation of the vessel wall.
    We begin our derivation of Dimension-Reduced models by assuming the following
    transformation exists of our fluid domain boundary $\Omgt$ to a simplified geometry.

    \medskip

    \begin{figure}[!htb]
        \label{fig:compliant-vessel}
        \centering
        \captionsetup{aboveskip=2pt,belowskip=2pt}
        \includegraphics[width=0.95\linewidth,height=0.45\textheight,keepaspectratio]{compliant-vessel.png}
        \caption{
            From \cite{book1} [Fig. 3.2, pg. 37]
        }
    \end{figure}

    \newpage

    We consider the fluid dynamics of the following fluid element contained in a portion of the lumen $\Omgt$
    \begin{figure}[!htb]`'
        \label{fig:fluid-element}
        \centering
        \captionsetup{aboveskip=2pt,belowskip=2pt}
        \includegraphics[width=0.8\linewidth,height=0.25\textheight,keepaspectratio]{fig33.png}
        \caption{
            From \cite{book1} [Fig. 3.3, pg. XX]
        }
    \end{figure}
    Let $S_1(t), ~ S_2(t)$ be the time-dependent shaded boundaries at $z = z_1$ and $z = z_2$ s.t.
    $0 < z_1 < z_2 < \ell$.
    Let $V_t$ be the fluid element of blood. The boundary of the fluid element is
    $\partial V_t \; = \; S_1(t) \cup S_2(t) \cup \partial V_{t,w}$ such that $\partial V_{t,w}$
    is the vessel wall in contact with the fluid element.
    According to the Reynold's transport theorem for scalar feild $\phi \in L^1(\Omgt)$.
    \begin{flalign*}
        \quad \dfrac{d}{dt} \int_{V_t} \phi dV \;
            = \; \int_{V_t} \dfrac{\partial \phi }{\partial t} dV \;
            + \; \int_{\partial V_t}  (\vct{u_b} \cdot \nhat) \phi dS &&
    \end{flalign*}
    where $\vct{u_b}$ is the velocity feild deforming the boundary $\partial V_t$
    (Pf. see \href{https://en.wikipedia.org/wiki/Reynolds_transport_theorem}{wiki}).
    If we assume the normal component of $\vct{u_b} = \vct{0}$ near the inlet
    and outlet boundaries $S_1$ and $S_2$ (resp.) of $\Omg$, then the motion of the vessel wall
    is coupled to the blood flow through the fluid element $V_t$. The velocity $\vct{u_b}$
    is equivalent to the velocity of the vessel wall $\partial \Omgt$ in contact with the boundary element $\partial V_t$.
    I.e., the vessel wall velocity $\vct{u_w} = \vct{u_b}$.
    Now let $\vct{w} = \vct{u_w} - \vu$ be the relative velocity of the vessel wall
    w.r.t. the velocity $\vu = (u_1, u_2, u_3)^\top$ of the blood element $V_t$. Then it follows that
    \begin{flalign*}
        \quad \; \int_{\partial V_t} \big(\vct{u_b} \cdot \nhat \big)\phi \; dS ~
            & = ~ \; \int_{\partial V_t} \big(\vct{u_w} \cdot \nhat \big)\phi \; dS &&\\
        \quad \; & = \; \int_{\partial V_t} \big(\vct{w} \cdot \nhat \big)\phi \; dS
            ~ + ~ \; \int_{\partial V_t} \big(\vu \cdot \nhat \big)\phi \; dS &&
    \end{flalign*}
    Let $\bar{\phi}$ denote the average value of $\phi$ defined over a surface $S$
    \begin{flalign*}
        \quad \bar{\phi} \defined \frac{1}{A} ~ \int_{S(z, t)} \phi \;dS\quad :  \quad  A(z, t) \defined \int_{S(z, t)} dS &&
    \end{flalign*}
    Now we may rewrite the volume integral in the LHS of RT theorem
    \begin{flalign*}
        \quad  \int_{V_t} \phi dV \; = \; \int_{z_1}^{z_2} \int_{S(z, t)} \phi \;dS \; dz
            \;=\; \int_{z_1}^{z_2} A \cdot \bar{\phi} \; dz &&
    \end{flalign*}
    where $z_1 < z_2$ are fixed $z$-coordinates for $S_1$ and $S_2$.
    Then we differentiate the integrands in the above equation w.r.t. $t$
    \begin{flalign*}
        \quad \int_{V_t} \dfrac{\partial \phi}{\partial t} dV \;=\; \int_{z_1}^{z_2} \dfrac{\partial}{\partial t} \bigg[A \cdot \bar{\phi} \bigg]\; dz, &&
    \end{flalign*}
    and we've rewritten the first term in the RHS of the reynolds system.
    The surface integral in the RHS may be written as
    \begin{flalign*}
        \quad \int_{\partial V_t}  (\vct{u_b} \cdot \nhat) \phi dS = \int_{\partial V_t}  (\vct{u_b} \cdot \nhat) \phi dS.... &&
    \end{flalign*}\todo{cleanup, ref. pg. 43}
    With a little more work, one may obtain:
    \begin{definition}
        \label{def:1dReynolds}
        The 1D Reynolds Transport theorem for both compressible and incompressible fluids:
        \begin{flalign*}
            \quad \frac{\partial }{\partial t} \bigg( A \bar{\phi} \bigg) \; + \; \frac{\partial}{\partial z}\big( A (\overline{\phi \cdot u_3}) \big)
            \; = \; \int_S \bigg( \dfrac{\partial \phi}{\partial t} + \nabla \cdot (\phi \vu) \bigg) dS
            \; + \; \int_{\partial S} \phi \vct{w} \cdot \nhat \; d \gamma &&
        \end{flalign*}
    \end{definition}

    \begin{remark}
        \label{1DMassConservation}
        By taking $f = \rho$ in \ref{def:1dReynolds}, mass conservation follows directly. Also, by our assumption
        that blood is incompressible, we have $\begin{cases*}\Div(\vu) = 0\\\rho =\text{const.}\\
        \end{cases*}$ and we simplify \ref{def:1dReynolds} as
        \begin{flalign*}
            \quad \frac{\partial A }{\partial t} \; + \; \frac{\partial}{\partial z}\big( A (\overline{u_3}) \big)
            \; = \; \int_{\partial S} \vct{w} \cdot \nhat \; d \gamma &&
        \end{flalign*}
        The RHS term above describing the transport process across the vessel wall. \todo{complete, pg. 45}
    \end{remark}

    \begin{remark}
        \label{1DMomentumConservation}
        By taking $f = u_3$ in \ref{def:1dReynolds}, momentum conservation follows directly. Also,
        by our assumption that blood is incompressible, we simplify \ref{def:1dReynolds} as
        \begin{flalign*}
            \quad \frac{\partial }{\partial t} \bigg( A u_3 \bigg) \; + \; \frac{\partial}{\partial z}\big( A (\overline{u_3^2}) \big)
            \; = \; \int_S \bigg( \dfrac{\partial u_3}{\partial t} + \nabla u_3 \cdot \vu \bigg) dS
            \; + \; \int_{\partial S} u_3 \vct{w} \cdot \nhat \; d \gamma &&
        \end{flalign*}
        The RHS term above describing the transport process across the vessel wall.
    \end{remark}


    \begin{remark}[Tube law from a thin elastic cylindrical wall]
        \label{rem:tube-law-derivation}
        We briefly justify the pressure--area relation used in \textsf{aq\_1d\_compliant.jl}
        \begin{definition}
            \label{def:tube-law}
            \begin{flalign*}
                \quad & p(A) - p_{ext} \;=\; \beta\big(\sqrt{A} - \sqrt{A_0}\big) &&
            \end{flalign*}
        \end{definition}
        used in the 1D $(A,Q)$ model.  Consider a straight cylindrical vessel with
        (local) lumen radius $R(x,t)$, reference radius $R_0$, wall thickness
        $h \ll R$, and internal pressure $p(x,t)$ relative to an external pressure
        $p_{ext}$ (assumed constant in space and time for simplicity).  The
        corresponding lumen area is
        \begin{flalign*}
            \quad & A(x,t) = \pi R(x,t)^2, &&
            \qquad
            A_0 = \pi R_0^2. &&
        \end{flalign*}

        Under the thin--wall assumption, balance of forces in the circumferential
        direction (Young--Laplace law) yields
        \begin{definition}
            \label{def:laplace}
            \begin{flalign*}
                \quad & (p - p_{ext})\,2\pi R \;=\; \sigma_\theta\,2h\pi &&
            \end{flalign*}
        \end{definition}
        where $\sigma_\theta$ is the circumferential (hoop) Cauchy stress in the
        vessel wall.  We model the wall as linearly elastic in the hoop direction,
        so that
        \begin{definition}
            \label{def:linear-elastic}
            \begin{flalign*}
                \quad & \sigma_\theta \;=\; E_{eff}\,\varepsilon_\theta &&
            \end{flalign*}
        \end{definition}
        with an effective circumferential modulus $E_{eff} > 0$ and circumferential
        strain
        \begin{flalign*}
            \quad & \varepsilon_\theta \;=\; \frac{\text{change in circumference} - {\text{reference circumference}}}{\text{reference circumference}} &&
            \quad & = \; \frac{2\pi R - 2\pi R_0}{2\pi R_0} &&
            \quad & = \; \frac{R - R_0}{R_0}. &&
        \end{flalign*}

        Equating \eqref{def:laplace} and \eqref{def:linear-elastic} gives
        \begin{flalign*}
            \quad & (p - p_{ext})\,2\pi R \;=\; E_{eff}\,\frac{R - R_0}{R_0}\,2h\pi &&\\
            \quad & \iff (p - p_{ext})\,R \;=\; E_{eff}\,\frac{R - R_0}{R_0}\,h &&
        \end{flalign*}
        Express $R$ and $R_0$ in terms of the areas $A$ and $A_0$:
        \begin{flalign*}
            \quad  R & = \sqrt{\frac{A}{\pi}} &&\\
            \quad & = \; \frac{\sqrt{A}}{\sqrt{\pi}}, &&\\
            R_0 = \sqrt{\frac{A_0}{\pi}}
            \quad & = \;
            \frac{\sqrt{A_0}}{\sqrt{\pi}},
        \end{flalign*}
        so that
        \begin{flalign*}
            \quad R - R_0
            \;=\;
            \frac{\sqrt{A}}{\sqrt{\pi}} - \frac{\sqrt{A_0}}{\sqrt{\pi}} &&\\
            \quad & = \; \frac{1}{\sqrt{\pi}}\big(\sqrt{A} - \sqrt{A_0}\big). &&
        \end{flalign*}
        Substituting into the expression for $p - p_{ext}$, we obtain
        \begin{flalign*}
            \quad & (p - p_{ext})\,\frac{\sqrt{A}}{\sqrt{\pi}}
            \;=\;
            E_{eff} h\,
            \frac{\dfrac{1}{\sqrt{\pi}}\big(\sqrt{A} - \sqrt{A_0}\big)}{R_0} &&\\
            \quad & \iff (p - p_{ext})\,\frac{\sqrt{A}}{\sqrt{\pi}}
            \;=\;
            E_{eff} h\,
            \frac{\dfrac{1}{\sqrt{\pi}}\big(\sqrt{A} - \sqrt{A_0}\big)}{R_0}. &&\\
        \end{flalign*}
        For moderate deformations where $A$ remains close to $A_0$, we approximate
        the factor $1/\sqrt{A}$ by its reference value $1/\sqrt{A_0}$, which yields
        \begin{flalign*}
            \quad & (p - p_{ext})\,\frac{\sqrt{A_0}}{\sqrt{\pi}}
            \;=\;
            E_{eff} h\,
            \frac{\dfrac{1}{\sqrt{\pi}}\big(\sqrt{A} - \sqrt{A_0}\big)}{R_0} &&\\
            \quad & \iff (p - p_{ext})
            \;=\;
            \frac{E_{eff} h}{R_0\sqrt{A_0}}\,
            \big(\sqrt{A} - \sqrt{A_0}\big). &&
        \end{flalign*}
        Defining the lumped stiffness parameter
        \begin{flalign*}
            \quad & \beta \;:=\; \frac{E_{eff} h}{R_0\sqrt{A_0}}. &&
        \end{flalign*}
        we arrive at the tube law \eqref{eq:tube-law} used in the 1D model:
        \begin{flalign*}
            \quad &
            p(A) - p_{ext}
            \;=\;
            \beta\big(\sqrt{A} - \sqrt{A_0}\big). &&
        \end{flalign*}
        In the numerical experiments below, we take $\beta$ and $A_0$ to be
        constant along the vessel, so that $p$ can be written as a function of
        $A$ alone.
    \end{remark}

    \medskip

    \subsubsection{0D Models}
        \label{subsubsec:0D}
        The 0D model, on the other hand, treats the vessel as a lumped parameter system,
        focusing on overall pressure and flow relationships without spatial resolution.


        \subsection{Navier-Stokes}
            \label{subsec:ns} When coupling our momentum balance equation with the divergence-free condition of $\vu$, we obtain the
Navier-Stokes (NS) equations. We present several equivalent forms of the NS equations below.
First, we write the NS system in conservative form.
\medskip
\begin{definition}[Conservative-Momentum Balance Form]
    \label{def:nsconservative}
    \begin{flalign*}
        \quad & \begin{cases*}
            \quad \pd_t(\rho \vu) \; +\;  (\rho\vu\cdot\nabla)\vu
                \equals -\,\nabla p \;\+\; \Divbf\!\big( 2\,\eta\,\vct{D}(\vu) \big) \;\+\; \rho\,\vf, \\
            \quad \Div(\vu) \equals 0,\qquad \rho \equivto \rho_0>0 \text{ (constant).}
        \end{cases*} &&
    \end{flalign*}
\end{definition}
Since $\rho \equivto \rho_0$ and $\eta \equivto \mu |\vct{D}|^2$, the advective form follows
from~\ref{def:nsconservative}.
\begin{definition}[Generalized-Newtonian Navier-Stokes (NS)]
    \label{def:NSEquations}
    \begin{flalign*}
        \quad & \begin{cases*}
            \quad \rho\big(\pd_t \vu + (\vu\cdot\nabla)\vu\big) \equals -\,\nabla p \;+\; \Divbf\!\Big( 2\,\mu(|\vct{D}|^2)\,\vct{D}\Big) \;+\; \rho\,\vf \\
        \quad \Div(\vu) = 0,
        \end{cases*} &&
    \end{flalign*}
\end{definition}
We divide $\rho$ and obtain the kinematic-viscosity form from~\ref{def:NSEquations}
as presented in \cite{book1}, \cite{book2}, and \red{\cite{hemodynamics} (confirm)}.
\begin{definition}[Kinematic-Viscosity Navier-Stokes (NS)]
    \label{def:kinematic-viscosity-ns}
    \begin{flalign*}
        \quad & \begin{cases*}
        \quad \pd_t \vu + (\vu\cdot\nabla)\vu \equals -\,\frac{\nabla p}{\rho}\, \;+\; \Divbf\!\Big( \frac{2}{\rho}\,\mu(|\vct{D}|^2)\,\vct{D}\Big) \;+\; \vct{f}, \\
        \quad \Div(\vu) = 0,
        \end{cases*} &&
    \end{flalign*}
\end{definition}
\medskip
Finally, we write the NS system in operator form. \red{(check arguments of F.
Rewrite with Laplacian instead of divergence of stress tensor?)}
{\small
\begin{definition}
    \label{def:standard-form-ns}
    We write Eq. \ref{def:NSEquations} in standard form
    \begin{flalign*}
        \quad & \begin{cases*}
            \quad F\big(\pd_t\vu,\,\nabla\vu,\,\nabla p,\, \vu,\, p;\, \rho, \mu\big) \equals \vct{f}, \\
            \quad \Div(\vu) \equals 0,
        \end{cases*} && \\
        \text{where} \quad & F\big(\pd_t\vu,\,\nabla\vu,\,\nabla p,\, \vu,\, p;\, \rho, \mu\big) \defined \pd_t \vu \;+\;  (\vu\cdot\nabla)\vu \;+\; \frac{\nabla p}{\rho}  \;-\;  \Divbf\!\Big( \frac{2}{\rho}\;\mu\;(|\vct{D}|^2)\,\vct{D}\Big) &&
    \end{flalign*}
\end{definition}}
\begin{remark}
    \label{classifying NS}
    The NS equations are a non-linear coupled system of PDEs.
    \medskip
    The first equation follows from the balance of linear momentum and it's
    terms are characterized as:
    \begin{itemize}
        \item  The convective term $(\vu\cdot\nabla)\vu = \vect{
            \inp{\vu}{\nabla \vu_1}\\
            \inp{\vu}{\nabla \vu_2}\\
            \inp{\vu}{\nabla \vu_3}
        }$ governs acceleration of fluid (non-linear).
        \item The diffusive term $\Divbf\!\Big( 2\,\mu(|\vct{D}|^2)\,\vct{D}\Big)$ describes
        the viscouselastic behavior (linear since $\mu$ is constant).
    \end{itemize}
    The second equation is the continuity equation, a consequence of the assumed fluid properties of blood
    that lead to the divergence-free condition on $\vu$.
    \medskip
    The total system comprises of four equations in four unknowns: the three components of the velocity field
    $\vu$ and the pressure field $p$.
\end{remark}
If pressure $p$ and the velocity $\vu$ are given, the Cauchy stress $\vT$ is computed
from Eq. \ref{def:improssible-stress-tensor}. It follows that the wall shear stress (WSS) at the vessel wall is:
\begin{flalign*}
    \quad \text{WSS} \defined \inp{\vct{t_{blood}}}{\vT\;\hat{n}} \quad : \quad \begin{cases*}
        t_{blood} \text{ is tangent of a flow line through a cross-sectional area} \\
        \hat{n} \text{ is outer normal of the cross-sectional area}
    \end{cases*}&&
\end{flalign*}
Forgoing the rigid-wall assumption allows us to model the relationship
between the vessel wall and blood flow. Applicable models are referred
to as fluid-structure interaction (FSI) models.\todo{discuss in a later section}

\medskip

\subsection{Existence and Uniqueness of NS}
    \label{subsec:existenceUniqueness}

    \begin{remark}[Global Regularity Problem for (NS)]\
        \label{rem:global-regularity}
        \textit{For any smooth, spatially localized initial data $ \mathbf{u}_0 $,
        does there exist a global smooth solution $ (\mathbf{u}, p) $ to NS~\ref{def:NSEquations}, when $N\equals 3$.
        Such question is one of the Millennium Prize Problems posed
        by the Clay Mathematics Institute in 2000, with a prize of one million
        dollars for a correct solution.}
    \end{remark}

    \begin{theorem}[Local Existence and Uniqueness]
        \label{thm:local-existenceUniqueness}
        Given smooth, localized initial data $\vct{u}_0$, there exists a maximal time $0 < T_* \leq \infty$
        for which a unique solution exists.
    \end{theorem}

    \medskip

    If $T_* < \infty$, a \textbf{blow-up} occurs:
    \begin{flalign*}
        \quad & \sup_{x \in \mathbb{R}^3} |\mathbf{u}(t, x)| \to +\infty \quad \text{as} \quad t \to T_*. &&
    \end{flalign*}
    Otherwise, if $T_* = \infty$, then $|\vct{u}| \to 0$ as $t \to \infty$. Numerical evidence suggests global regularity
    holds in most practical cases, but turbulent behavior can emerge for large initial data.

    \subsubsection*{Heuristic Considerations and Energy Balance}
        Starting from the incompressibility condition:
        \begin{flalign*}
            \quad & \Div(\vct{u}) = 0 && \\
            \quad & \iff ~ \rho \text{ is constant in } \Omega(t) && \\
            \quad & \iff ~ \text{ chain rule applies to } \Divbf\big(\frac{2}{\rho} \mu (|\vct{D}|^2) \vct{D}\big) &&\\
            \quad \implies &  \Divbf\big(\frac{2}{\rho} \mu (|\vct{D}|^2) \vct{D}\big) = \nabla \cdot \big(\frac{2}{\rho} \mu (|\vct{D}|^2) \vct{D}\big)&&\\
            \quad & = \langle\frac{2}{\rho} \mu (|\vct{D}|^2) \vct{D}, \nabla \rangle&&\\
            \quad  & = \frac{2}{\rho} \langle \mu (|\vct{D}|^2) \vct{D}, \nabla \rangle&&\\
            \quad & = \frac{2}{\rho} \nabla \cdot \big( \mu (|\vct{D}|^2) \vct{D}\big)&&\\
            \quad & \therefore \quad \frac{2}{\rho}~ \Divbf\big( \mu (|\vct{D}|^2) \vct{D}\big) = 0.&&
        \end{flalign*}
        This vanishes if $\mu$ is constant (Newtonian fluid) and $\rho$ is constant (incompressibility).
        So the diffusive term becomes:
        \begin{flalign*}
            \quad \frac{2\eta}{\rho} \Delta \vct{u}, \quad \text{with } \eta = \mu \rho.&&
        \end{flalign*}\todo{show simplification of diffusive term to laplacian}
        We heuristically compare dominant terms:
        \begin{enumerate}
            \item If $\eta \Delta \vct{u} \gg (\vct{u} \cdot \nabla) \vct{u}$, viscous dissipation dominates $\Rightarrow$ smooth, regular behavior.\
            \item If $(\vct{u} \cdot \nabla) \vct{u} \gg \eta \Delta \vct{u}$, nonlinearity dominates $\Rightarrow$ turbulence, potential blow-up.
        \end{enumerate}
        We construct rigorous energy estimates in Section~\ref{sec:math-models}.
    \medskip

\subsubsection{NS in Cylindrical Coordinates}
    \label{subsubsec:cylindrical-coords}
    The relationship between cartesian and cylidrical coordinates is
    \begin{flalign*}
        \quad (x, ~y, ~z) ~ \mapsto (r \sin(\theta), ~ r\cos(\theta), ~z), \quad r = \sqrt{x^2 + y^2}.&&
    \end{flalign*}

    \medskip

    Let $(r,\theta,z)$ denote cylindrical coordinates on $\R^3$ with
    orthonormal basis $(\e_r,\e_\theta,\e_z)$, and write the velocity as
    \begin{flalign*}
        \quad \vu(t, r,\theta,z) &\equivto u_r(t, r, \theta,z)\,\e_r \;+\; u_\theta(t, r, \theta,z)\,\e_\theta \;+\; u_z(t, r, \theta,z)\,\e_z. &&\\
        &\equals u_r\,\e_r \;+\; u_\theta\,\e_\theta \;+\; u_z\,\e_z
    \end{flalign*}
    where $u_r, u_\theta, u_z: \OmgB \to \R$ are the cylindrical
    velocity components in the radial, azimuthal, and axial directions respectively. Then the kinematic-viscosity form of NS~\ref{def:kinematic-viscosity-ns} in cylindrical coordinates
    involves the following differential operators:
    \begin{flalign*}
        \quad (\vu\cdot\nabla) &\equals u_r \,\pd_r \;+\; \frac{u_\theta}{r}\,\pd_\theta \;+\; u_z \,\pd_z, &&\\
        \nabla p &\equals \pd_r p\,\e_r \;+\; \frac{1}{r}\,\pd_\theta p\,\e_\theta \;+\; \pd_z p\,\e_z, &&\\
        \Div(\vu) &\equals \frac{1}{r}\,\pd_r(r u_r) \;+\; \frac{1}{r}\,\pd_\theta u_\theta \;+\; \pd_z u_z. &&
    \end{flalign*}
    The components of the rate-of-strain tensor $\vct{D} = \tfrac{1}{2}\big(\nabla\vu + (\nabla\vu)^\top\big)$
    with respect to $(\e_r,\e_\theta,\e_z)$:
    \begin{flalign*}
        \quad D_{rr} &= \pd_r u_r, & D_{\theta\theta} &= \frac{1}{r}\,\pd_\theta u_\theta + \frac{u_r}{r}, & D_{zz} &= \pd_z u_z, && \\
        \quad D_{r\theta} &= \frac{1}{2}\Big(\pd_r u_\theta - \frac{u_\theta}{r} + \frac{1}{r}\,\pd_\theta u_r\Big), &
        D_{rz} &= \frac{1}{2}\big(\pd_r u_z + \pd_z u_r\big), & D_{\theta z}&= \frac{1}{2}\Big(\frac{1}{r}\,\pd_\theta u_z + \pd_z u_\theta\Big) &&.
    \end{flalign*}
    The viscous (extra) stress is $\vct{\tau} = 2\,\mu(|\vct{D}|^2)\,\vct{D}$, so
    \begin{flalign*}
        \quad \tau_{ij} = 2\,\mu(|\vct{D}|^2)\,D_{ij}, \qquad i,j\in\{r,\theta,z\}. &&
    \end{flalign*}
    Writing $\vct{f} = f_r\,\e_r + f_\theta\,\e_\theta + f_z\,\e_z$, the three components of the momentum equation are
    \begin{flalign*}
        \text{(radial)}\quad
        &\pd_t u_r
        + u_r \pd_r u_r
        + \frac{u_\theta}{r}\pd_\theta u_r
        + u_z \pd_z u_r
        - \frac{u_\theta^2}{r} &&\\
        &\quad \equals -\,\frac{1}{\rho}\,\pd_r p
        + \frac{1}{\rho}\Big[
              \frac{1}{r}\pd_r(r \tau_{rr})
            + \frac{1}{r}\pd_\theta \tau_{r\theta}
            + \pd_z \tau_{rz}
            - \frac{1}{r}\tau_{\theta\theta}
          \Big]
        + f_r, && \\
        \text{(azimuthal)}\quad
        &\pd_t u_\theta
        + u_r \pd_r u_\theta
        + \frac{u_\theta}{r}\pd_\theta u_\theta
        + u_z \pd_z u_\theta
        + \frac{u_r u_\theta}{r} \\
        &\quad \equals -\,\frac{1}{\rho r}\,\pd_\theta p
        + \frac{1}{\rho}\Big[
              \frac{1}{r}\pd_r(r \tau_{r\theta})
            + \frac{1}{r}\pd_\theta \tau_{\theta\theta}
            + \pd_z \tau_{\theta z}
            + \frac{1}{r}\tau_{r\theta}
          \Big]
        + f_\theta, &&\\
        \text{(axial)}\quad
        &\pd_t u_z
        + u_r \pd_r u_z
        + \frac{u_\theta}{r}\pd_\theta u_z
        + u_z \pd_z u_z \\
        &\quad \equals -\,\frac{1}{\rho}\,\pd_z p
        + \frac{1}{\rho}\Big[
              \frac{1}{r}\pd_r(r \tau_{rz})
            + \frac{1}{r}\pd_\theta \tau_{\theta z}
            + \pd_z \tau_{zz}
          \Big]
        + f_z. &&
    \end{flalign*}
    The incompressibility condition in cylindrical coordinates is
    \begin{flalign*}
        \quad \frac{1}{r}\,\pd_r(r u_r) \;+\; \frac{1}{r}\,\pd_\theta u_\theta \;+\; \pd_z u_z \equals 0. &&
    \end{flalign*}



    \begin{figure}[!htb]
    \label{fig:compliant-vessel}
        \centering
        \captionsetup{aboveskip=2pt,belowskip=2pt}
        \includegraphics[width=0.45\linewidth,height=0.45\textheight,keepaspectratio]{compliant-vessel.png}
        \caption{
            From \cite{book1} [Fig. 3.2, pg. 37]: A compliant vessel with
            time-dependent radius $R(z,t)$ along the axial position $z$.
            Since $\pOmgB \in C^1$, the normal vector $\hat{n}$ is well-defined a.e. on the boundary.
            We assume such transformation is possible for blood vessels in our model.
        }
    \end{figure}
    Assume a vessel of length $\ell$ is aligned with the z-axis
    whose cross-section is circular with radius $R(z,t)$ at axial position $z$ and time $t$.
    Our fluid domain becomes
    \begin{equation*}
        \label{fluid-domain-cylidrical}
        \Omgt \; = \; \{(r, ~\theta ~, z) \in \R^3 ~ : ~r \in [0, R(z, t)], ~ \theta \in [0, 2\pi), ~ z \in [0, \ell) \}
    \end{equation*}
    where $R(z,t)$ is the vessel radius at axial position $z$ and time $t$.

    \medskip
    \td{WIP: continue from here...}

        \newpage



    % ------------------ APPENDIX
    \section{Appendix}
        \label{sec:appendix}

        \subsection*{Bibliography}
            \printbibliography

        \subsection*{Code Listings}

        \td{Code listings}
        \lstset{language=julia}
        \begin{lstlisting}[language=julia, caption={Algorithm 16.5}, label={lst:problem-sampling}]
        function foo()
            println("Hello World)
        end
        \end{lstlisting}



\end{document}
% ===================== END DOCUMENT ================================
% ===================================================================
