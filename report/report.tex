\documentclass[10pt]{article}



% ===================== PREAMBLE ====================================
% Contains packages, custom commands, and formatting
% ===================================================================
% Packages
% ---------- Basic Packages ----------
\usepackage[utf8]{inputenc}
\usepackage{array}
\usepackage{geometry}
\usepackage{setspace}
\usepackage{multicol}
\usepackage{xcolor}
\usepackage{graphicx}
\usepackage{float}
\usepackage{caption}
\usepackage{booktabs}
\usepackage{url}
\usepackage{hyperref}
\usepackage{lipsum}
\usepackage{fancybox}
\usepackage{todonotes}
\usepackage[dvipsnames]{xcolor}
\usepackage{mdframed} % boxed theorems
\usepackage{thmtools} % couples well with mdframed
\usepackage{tikz} \usetikzlibrary{arrows.meta, calc, decorations.pathmorphing, positioning}
\usepackage{listings}

% --------- Code Listings ----------
\captionsetup[lstlisting]{font={small}, labelfont={bf}, labelsep=colon}
\renewcommand{\lstlistingname}{Code}

% ---------- Color Commands ----------
\DeclareRobustCommand{\yellow}[1]{\begingroup\color{Yellow}\ignorespaces#1\endgroup\ignorespacesafterend}
\DeclareRobustCommand{\red}[1]{\begingroup\color{Red}\ignorespaces#1\endgroup\ignorespacesafterend}
\DeclareRobustCommand{\blue}[1]{\begingroup\color{Blue}\ignorespaces#1\endgroup\ignorespacesafterend}
\DeclareRobustCommand{\green}[1]{\begingroup\color{Green}\ignorespaces#1\endgroup\ignorespacesafterend}
\DeclareRobustCommand{\gray}[1]{\begingroup\color{Gray}\ignorespaces#1\endgroup\ignorespacesafterend}
\newenvironment{redblock}{\begingroup\color{Red}\ignorespaces}{\endgroup\ignorespacesafterend}


% ---------- Math and Theorem Environments ----------
\usepackage{algorithm}
\usepackage{algpseudocode}
\usepackage{amsmath, amssymb, amsthm, amsfonts, mathtools, bm, amsopn}
\usepackage{witharrows}
\usepackage[dvipsnames]{xcolor}
\usepackage{mdframed}       % boxed theorems
\usepackage{thmtools}       % couples well with mdframed
\usepackage{capt-of}        % if you prefer \captionof

% ================== THEOREM ENVIRONMENTS ==================
\numberwithin{equation}{subsection} % optional

% Main shared counter “theorem” per section:
\newtheorem{theorem}{Theorem}[subsection]

% All others share the same counter as “theorem”:
\newtheorem{lemma}[theorem]{Lemma}
\newtheorem{proposition}[theorem]{Proposition}
\newtheorem{corollary}[theorem]{Corollary}

\theoremstyle{definition}
\newtheorem{definition}[theorem]{Definition}
\newtheorem{example}[theorem]{Example}
\newtheorem{prop}[theorem]{Proposition}

\theoremstyle{remark}
\newtheorem{remark}[theorem]{Remark}

% Boxed Assumption using the custom style, sharing the same counter:
\declaretheorem[
  name=Assumption,
  style=boxed,
  sibling=theorem
]{assumption}
% % Define a theorem style that uses mdframed for the box
% \declaretheoremstyle[
%   headfont=\normalfont\bfseries,
%   mdframed={
%     linewidth=0.5pt,
%     roundcorner=4pt,
%     backgroundcolor={gray!5},
%     innertopmargin=6pt,
%     innerbottommargin=6pt,
%     innerleftmargin=8pt,
%     innerrightmargin=8pt
%   }
% ]{boxed}
% % Create the “assumption” environment, numbered within each section
% \declaretheorem[name=Assumption, style=boxed, numberwithin=section]{assumption}
% % Other logical environments
% \newenvironment{solution}
%   {\renewcommand\qedsymbol{$\blacksquare$}\begin{proof}[\textbf{Solution}]}
%   {\end{proof}}
% \newtheorem{lemma}{Lemma}
%   \renewcommand{\thelemma}{\Alph{lemma}}
% \newtheorem*{lem}{Lemma}
% \newtheorem{corollary}[lemma]{Corollary}
% \newtheorem*{theorem}{theorem}
% \theoremstyle{definition}
% \newtheorem{definition}{Definition}
% \theoremstyle{remark}
% \newtheorem*{remark}{Remark}
% \theoremstyle{claim}
% \newtheorem*{claim}{Claim}

% ---------- Page setup ----------
\geometry{letterpaper, margin=1in}
\setlength{\parindent}{0pt}
\linespread{1.5}
\renewcommand{\arraystretch}{1.25} % Adjust as needed

% ---------- Shaded box's to tag ideas ----------
\newcommand{\td}[1]{\todo[inline, color=red!20, size=\small]{#1}}
\newcommand{\note}[1]{\todo[inline, color=blue!20, size=\small]{#1}}
\newcommand{\remeber}[1]{\todo[inline, color=green!20, size=\small]{#1}}

% ---------- Sets and number systems ----------
\newcommand{\Z}{\mathbb{Z}}
\newcommand{\Q}{\mathbb{Q}}
\newcommand{\C}{\mathbb{C}}
\newcommand{\R}{\mathbb{R}}
\newcommand{\N}{\mathbb{N}}
\newcommand{\La}{\mathcal{L}}

% ---------- Summations ----------
\newcommand{\sumforall}{\sum_{\forall i}}
\newcommand{\sumfor}[1]{\sum_{\forall #1}}

% ---------- Generic Domains ----------
\newcommand{\Omg}{\Omega}
\newcommand{\OmgBar}{\overline{\Omega}}
\newcommand{\Omgt}{\Omega(t)}
\newcommand{\pOmg}{\partial \Omega}
\newcommand{\pOmgt}{\partial \Omega(t)}


%---------- Blood Domain ----------
\newcommand{\OmgB}{\Omega_B}
\newcommand{\OmgBt}{\Omega_B(t)}
\newcommand{\OmgBtBar}{\overline{\Omega}_B(t)}
\newcommand{\OmgBBar}{\overline{\Omega}_B}
\newcommand{\pOmgB}{\partial \Omega_B}
\newcommand{\pOmgBt}{\partial \Omega_B(t)}

%---------- Vessel Wall Domain ----------
\newcommand{\OmgW}{\Omega_W}
\newcommand{\OmgWt}{\Omega_W(t)}
\newcommand{\pOmgW}{\partial \Omega_W}

% ---------- Symbols ----------
\newcommand{\e}{\epsilon}
\newcommand{\p}{\rho}
\newcommand{\eps}{\epsilon}
\newcommand{\la}{\lambda}

% ---------- Linear Algebra ----------
\newcommand{\matr}[1]{\begin{pmatrix}#1\end{pmatrix}} % For matrices
\newcommand{\vect}[1]{\begin{bmatrix}#1\end{bmatrix}} % For column vectors
\newcommand{\vct}[1]{\bm{#1}} % For bold vectors
\newcommand{\transp}{^{\mathsf{T}}} % Transpose symbol
\newcommand{\sym}{\text{sym}} % Symmetric part of a matrix
\DeclarePairedDelimiter\ip{\langle}{\rangle} % Inner product
\DeclarePairedDelimiter\norm{\lVert}{\rVert} % Norm
\newcommand{\zero}{\mathbf{0}} % Zero vector/matrix
\newcommand{\ident}{\mathbf{I}} % Identity matrix
\newcommand{\id}[1]{\operatorname{id}_{#1}} % Identity map on set #1
\DeclareMathOperator{\tr}{tr} % Trace
\DeclareMathOperator{\detm}{det} % Determinant
\DeclareMathOperator{\rank}{rank} % Rank
\DeclareMathOperator{\nullity}{nullity} % Nullity

% Block Matrices and Vectors ----------
\newcommand{\blockmat}[2]{\begin{pmatrix} #1 & #2 \end{pmatrix}} % Block matrix with 2 columns
\newcommand{\augmatr}[2]{\left(\begin{array}{#1|#1}#2\end{array}\right)}
\newcommand{\bmfour}[4]{\left( \begin{array}{c|c} #1 & #2 \\ \hline #3 & #4 \end{array} \right)} % 2x2 block matrix with lines
\newcommand{\blockvect}[2]{\begin{bmatrix} #1 \\ #2 \end{bmatrix}} % Block vector with 2 rows

% Unit vectors ----------
\newcommand{\nhat}{\hat{\vct{n}}}
\newcommand{\that}{\hat{\vct{t}}}
\newcommand{\ihat}{\hat{\vct{i}}}
\newcommand{\jhat}{\hat{\vct{j}}}
\newcommand{\khat}{\hat{\vct{k}}}

% Generic vectors and tensors. (bold notation) ---------
\newcommand{\vu}{\vct{u}}
\newcommand{\vv}{\vct{v}}
\newcommand{\vT}{\vct{T}}
\newcommand{\vt}{\vct{t}}
\newcommand{\vTx}{\vct{T}(\vct{x})}
\newcommand{\vx}{\vct{x}}
\newcommand{\vxi}{\vct{\xi}}
\newcommand{\vfb}{\vct{f_b}}
\newcommand{\vf}{\vct{f}}
\newcommand{\vF}{\vct{F}}
\newcommand{\vA}{\vct{A}}
\newcommand{\vFhat}{\widehat{\vct{F}}}
\newcommand{\vJ}{\vct{J}}
\newcommand{\vJhat}{\widehat{\vct{J}}}
\newcommand{\Jdet}{J}
\newcommand{\Jhatdet}{\widehat{J}}
\newcommand{\phihat}{\widehat{\phi}}
\newcommand{\vw}{\vct{w}}
\newcommand{\va}{\vct{a}}
\newcommand{\vahat}{\widehat{\vct{a}}}
\newcommand{\vuhat}{\widehat{\vct{u}}}
\newcommand{\vub}{\widehat{\vct{u_b}}}
\newcommand{\vepsilon}{\vct{\epsilon}}
\newcommand{\vtheta}{\vct{\theta}}
\newcommand{\ej}{\vct{e_j}}
\newcommand{\eo}{\vct{e_1}}
\newcommand{\ef}{\vct{e_1}} % first basis vector
\newcommand{\el}{\vct{e_n}} % nth basis vector

% Learning Stuff
\newcommand{\vW}[1]{\vct{W_{#1}}} % Weight matrix
\newcommand{\vN}[1]{\vct{N_{#1}}} % Neural Network Layer output}
\newcommand{\vb}[1]{\vct{b_{#1}}} % Bias vector
\newcommand{\vsigma}{\vct{\sigma}} % Activation function
\newcommand{\vmL}{\mathcal{L}_{\text{obj}}} % Loss function
\newcommand{\vs}{\vct{s}} % Step direction
\newcommand{\vH}{\vct{H}} % Hessian matrix
\newcommand{\vd}{\vct{d}} % Direction vector 1

% Basis vectors
\newcommand{\generalbasis}{\{\vct{v_1}, \ldots, \, \vct{v_n}\}}
\newcommand{\stdbasis}{\{\vct{e_1}, \ldots, \, \vct{e_n}\}}

% ---------- Operators ----------
\newcommand{\equivto}{\; \equiv \;}
\newcommand{\defined}{\; := \;}
\newcommand{\equals}{\; = \;}
\newcommand{\ol}{\overline}
\newcommand{\pd}{\partial}
\newcommand{\dt}{\frac{d}{dt}}
\newcommand{\pdt}{\frac{\partial}{\partial t}}
\newcommand{\dx}{\, d\vct{x}}
\newcommand{\dxi}{\, d\vct{\xi}}
\newcommand{\dSx}{\, dS_{\vct{x}}}
\newcommand{\argmin}{\operatorname{argmin}}
\DeclareMathOperator{\col}{col}
\DeclareMathOperator{\spn}{span}
\DeclareMathOperator{\range}{range}
\DeclareMathOperator{\kernel}{ker}
\newcommand{\inv}[1]{#1^{-1}}
\DeclareMathOperator{\dia}{diag}

\newcommand{\del}{\Delta}
\newcommand{\Div}{\text{div}}
\newcommand{\Divbf}{\textbf{div}}
\newcommand{\inp}[2]{\langle #1, \; #2 \rangle}
\newcommand{\forallin}[2]{\forall #1 \in #2}
\newcommand{\forallsubsets}[2]{\forall \; #1 \subset #2}
\newcommand{\units}[1]{\; \big[ \; #1 \; \big]}
\newcommand{\blinform}[2]{a\big( #1, \; #2 \big)}

% ---------- Define custom colors ----------
\definecolor{bg}{rgb}{0.95,0.95,0.92}
\definecolor{codeblue}{rgb}{0.25,0.5,0.75}
\definecolor{keyword}{rgb}{0.5,0,0.5}
\definecolor{string}{rgb}{0,0.5,0}

% ---------- Define the style for IPython ----------
\lstdefinelanguage{ipython}{
  basicstyle=\ttfamily\footnotesize\color{black},
  keywordstyle=\color{keyword}\bfseries,
  stringstyle=\color{string},
  backgroundcolor=\color{bg},
  morekeywords={In,Out},
  frame=lines,
  numbers=left,
  numbersep=5pt
}

% ---------- Define the style for Julia ----------
\lstdefinelanguage{julia}{
  keywords={struct, abstract, primitive, typealias, bitstype, mutable, module, baremodule, import, importall, using, export, if, else, elseif, for, while, break, continue, function, return, type, global, local, const, let, do, try, catch, finally, end, quote, macro, where},
  basicstyle=\ttfamily\footnotesize\color{black},
  keywordstyle=\color{keyword}\bfseries,
  stringstyle=\color{string},
  backgroundcolor=\color{bg},
  frame=lines,
  numbers=left,
  numbersep=8pt,
  inputencoding=utf8,
  extendedchars=true,
  escapeinside={(*}{*)},
  literate={~}{{$\sim$}}1
           {₁}{{\textsubscript{1}}}1
           {₂}{{\textsubscript{2}}}1
           {₃}{{\textsubscript{3}}}1
           {ᵀ}{{$^{\mathrm{T}}$}}1
           {⋅}{{$\cdot$}}1
           {∈}{{$\in$}}1
           {ℝ}{{$\mathbb{R}$}}1
           {→}{{$\to$}}1
           {π}{{$\pi$}}1
           {≤}{{$\leq$}}1
           {≥}{{$\geq$}}1
           {≈}{{$\approx$}}1
           {ϵ}{{$\epsilon$}}1
           {κ}{{$\kappa$}}1
           {Δ}{{$\Delta$}}1
           {λ}{{$\lambda$}}1
}

% ---------- Define the style for Mathematica ----------
\lstdefinelanguage{Mathematica}{
  keywords={Module,With,Block,If,Then,Else,Which,Switch,For,While,Return,Do,Table,Plot,Map,Apply,Function,Set,SetDelayed,Clear,Quit,NonlinearModelFit,FindMinimum,FindMaximum,FindRoot,NIntegrate,Integrate,Simplify,FullSimplify,DSolve,RSolve,NSolve,NDSolve,Limit,Series,Assuming,Expand,Factor,TableForm,MatrixForm,Part,Length,Dimensions,Transpose,MapThread,MapAt,Flatten,Thread,Join,Outer,ConstantArray,Riffle,ArrayPlot,Plot3D,ContourPlot,ParametricPlot,MatrixPlot,Graphics,Graphics3D,Manipulate,Evaluate,Sin,Cos,Exp,Log,Sqrt},
  keywordstyle=\color{keyword}\bfseries,
  sensitive=true,
  morecomment=[l]{(*}, morecomment=[r]{*)},
  commentstyle=\itshape\color{gray},
  morestring=[b]",
  stringstyle=\color{string},
  basicstyle=\ttfamily\footnotesize\color{black},
  backgroundcolor=\color{bg},
  frame=lines,
  numbers=left,
  numbersep=5pt,
  showstringspaces=false,
  escapeinside={(*}{*)}
}

% ---------- Optionally define a style for Mathematica code cells ----------
\lstdefinestyle{mathematicaStyle}{
  language=Mathematica,
  frame=lines,
  backgroundcolor=\color{bg},
  basicstyle=\ttfamily\footnotesize,
  keywordstyle=\color{keyword}\bfseries,
  commentstyle=\itshape\color{gray},
  stringstyle=\color{string},
  numbers=left,
  numbersep=5pt,
  showstringspaces=false
}

\usepackage{csquotes}                                   % TODO: why isn't this in preamble.tex?
\usepackage[backend=biber,style=numeric]{biblatex}
\addbibresource{citations.bib}                          % BibTeX file



% ===================== TITLE PAGE ==================================
% ===================================================================
\title{Blood Flow in the Human Circulatory System}
\author{Daniel Henderson, Michigan Technological University \\ \texttt{dahender@mtu.edu}}
\date{\today}



% ===================== BEGIN DOCUMENT ==============================
% ===================================================================
\begin{document}
    \maketitle


    % ------------------ COVER PAGE
    \noindent Report of modern techniques for modeling the motion of
    blood within a Human's Macrocirculatory System. \medskip
    \textbf{Keywords:} \textit{
        computational hemodynamics, 0D blood-flow, 1D blood-flow, 2D-blood-flow,
        PINN's, finite element methods, discontinous galerkin, Lax-Wendroff, fluid-structure ineraction (FSI)
    }
    \medskip \tableofcontents \newpage



    % ------------------ PRELIMINARIES
    \section{Preliminaries}
        \label{sec:prelims}

        \subsection*{Notation}
            \label{subsec:notation} \begin{minipage}[t]{0.48\textwidth}
    \begin{tabular}{@{}p{0.4\linewidth} p{1.8\linewidth}@{}}
        $\therefore$ & consequently\\
        $\because$ & because\\
        $\implies$ & implies \\
        $\iff$ & if and only if \\
        $:=$ & defines\\
        $\equiv$ & equivalence \\
        $ \R$                             & set of real numbers \\
        $ \R^{+}$                         & set of positive real numbers \\
        $ \R^{-}$                         & set of negative real numbers \\
        $ \R^{n}$                         & n-dimensional real vector space \\
        $ \Omg \subset \R^{n}$          & a connected open subset of $\R^{n}$ \\
        $ \OmgBar$                      & the closure of $\Omg$ \\
        $ \partial \Omg$                & the boundary of $\Omg$ \\
        $C^{k}(\Omg)$                   & space of $k$ times continuously differentiable functions on $\Omg$ \\
        $C^{k}_{0}(\Omg)$               & space of $k$ times continuously differentiable functions with compact support in $\Omg$ \\
        $C^{k}_{0}(\overline{\Omg})$    & space of $k$ times continuously differentiable functions which have bounded and uniformly
                                            continuous derivatives up to order $k$ with compact support in $\Omg$ \\
        $C^{\infty}_{0}(\Omg)$          & space of smooth functions with compact support in $\Omg$ \\
        $L^{p}(\Omg)$                   & Lebesgue space of $p$-integrable functions on $\Omg$ \\
    \end{tabular}
\end{minipage}\hfill
\newpage
\begin{minipage}[t]{0.48\textwidth}
    \begin{tabular}{@{}p{0.4\linewidth} p{1.8\linewidth}@{}}
        $ dx $                            & Lebesgue measure on $\R^{n}$ \\
        $ dS $                            & surface measure on $\partial \Omg$ \\
        $ dV $                            & volume measure on $\Omg$ \\
        $ \nabla $                        & gradient operator \\
        $ \del = \nabla^2 = \nabla \cdot \nabla(\cdot)$ & Laplace operator \\
        $ \Div $                          & divergence of a vector field \\
        $ \Divbf $                        & divergence of a tensor \\
        $ \vct{v}_i$                      & $i$-th component of vector $\vct{v}$ \\
        $ \langle \cdot, \cdot \rangle_X$ & inner product on vector space $X$ \\
        $ \langle \vu, \vct{v} \rangle$ & inner product of vectors $\vu, \vct{v} \in \R^n$ \\
        $ \dfrac{\partial}{\partial \nhat} = \langle \nabla, \nhat \rangle$ & normal derivative on $\partial \Omg$ \\
        $\| \cdot \|$                     & $L^2$-norm \\
    \end{tabular}
\end{minipage}

        \newpage

        \subsection*{Symbols and Abbreviations}
            \label{subsec:syms-abbrevs} \begin{minipage}[t]{0.48\textwidth}
    \begin{tabular}{@{}p{0.4\linewidth} p{1.8\linewidth}@{}}
        a.e. & almost everywhere\\
        e.g. & "exempli gratia" (for example)\\
        i.e. & "id est" (that means) \\
        s.s. & sufficiently smooth \\
        s.t. & such that \\
        r.t. & refers to \\
        w.r.t. & with respect to\\
        m.b.s. & must be shown \\
        i.m.b.s. & it must be shown \\
        i.r.t.s. & it remains to show \\
        w.a.t.s. & we aim to show \\
        bpm & beats per minute \\
        wlog & without loss of generality \\
        ODE & Ordinary Differential Equation \\
        PDE & Partial Differential Equation \\
        PDES & System of Partial Differential Equations \\
        IC & Initial Condition \\
        BC & Boundary Condition \\
        0D & Zero dimensional \\
        1D & One dimensional \\
        2D & Two dimensional \\
        3D & Three dimensional \\
        FSI & Fluid-Structure Interaction \\
        SB & Stenotic Blockage \\
        RBC & Red Blood Cell \\
        CVD & Cardiovascular disease \& CVDs r.t. such diseases\\
    \end{tabular}
\end{minipage}

        \newpage

        \subsection*{Parameters and Units}
            \label{subsec:params} \begin{minipage}[t]{0.48\textwidth}
    \begin{tabular}{@{}p{0.4\linewidth} p{1.8\linewidth}@{}}
        $R$ & radius of vessel with diameter $2R$ \\
        $\eta$ & dynamic viscosity $\quad \units{Pa \cdot s}$ \\
        $\mu$ & kinematic viscosity $\quad \units{\frac{cm^2}{s}}$\\
        $\tau$ & shear stress \\
        $\dot{\gamma}$ & shear rate \\
        $\rho$ & density field $\quad \units{\frac{kg}{cm^3}}$\\
        $p$ & pressure field \\
        $\vu$ & velocity field $\quad \units{\frac{cm}{s}}$\\
        $W_0$ & Womersley number $\units{-}$ \\
        $Re$ & Reynolds number $\units{-}$ \\
        $Pe$ & Péclet number $\units{-}$ \\
        $c$ & concentration of a material element \\
        $D$ & diffusion coefficient $\quad \units{\frac{cm^2}{s}}$\\
        $t$ & time $\quad \units{s}$ \\
        $T$ & terminal time $\quad \units{s}, t > 0$ \\
        $\omega$ & angular frequency $\quad \units{\frac{rad}{s}}$ \\
        $\vct{f_b}$ & body force per unit volume $\quad \units{\frac{N}{cm^3}}$ \\
    \end{tabular}
\end{minipage}

        \newpage

        \subsection*{Mathematical Foundations}
            \label{subsec:maths} % domain disscussion
We assume Zermelo-Fraenkel set theory with the axiom of choice (ZFC), and according to Cohen\href{https://en.wikipedia.org/wiki/Paul_Cohen},
it's necessary to also posulate the existence of the continuum hypothesis (CH).
A \emph{domain} will r.t. an open and bounded subset of $\R^N$ with nonempty interior $\Omg^\circ$,
for all $N\in\{1,2,3,4\}$. By the Heine-Borel theorem, every domain $\Omg$ has a well-defined boundary
$\partial \Omg$ with compact closure $\OmgBar = \Omg \cup \pOmg$. The compact domains of $\R^N$ are
precisely the closed and bounded subsets of $\R^N$.

% % simply connected, lipschitz domain, and convex domain
% When every path between two points in $\Omg$ may be continously contrated to a point without leaving
% $\Omg$, we say that $\Omg$ is \emph{simply connected}. A \emph{Lipschitz domain} is a domain $\Omg$
% s.t for every point $x\in\pOmg$, there exists a neighborhood $U_x$ of $x$ s.t $\Omg \cap U_x$ is the
% region above the graph of a Lipschitz continuous function (after a suitable rotation and translation
% of coordinates). A \emph{convex domain} is a domain $\Omg$ s.t for every $x,y\in\Omg$, the line segment
% connecting $x$ and $y$ is contained in $\Omg$.

\medskip

% function space discussion
Given domain $\Omega$, the function space $\mathcal{F}(\Omg)$ is a linear
space (sp.) $(F(\Omg), \, K)$ of scalar valued functions $f:\Omg \to K$; e.g. $F = C$ and $K = R$
is $C^0(\Omg)$, the sp. of continuous real valued functions on $\Omg$. Let $C^{k}(\Omg)$ be the
sp. of $k$-tides continuously differentiable functions on $\Omg$. Then $C^k(\OmgBar)$ is the sp. of functions
$f \in C^k(\Omg)$ s.t $f$ and it's derivatives up to order $k$ may be continuously extended to the boundary $\pOmg$.
The space $L^p(\Omg)$ denotes the Lebesgue sp. of $p$-integrable functions on $\Omg$. If We denote
the space of Lipschitz continous functions on $\Omega$ as $Lip(\Omg)$.\todo{relate to $C(\Omg)$, cite brenners book}.


\medskip

% more on sp.s: topological to metric to normed to inner-product sp.,
%   then completeness and banach sp. discussion; Hilbert sp.s are banach.

\medskip

% definition for scalar feilds
For real-valued $f \in C^1(\Omg)$, the partial derivative w.r.t. the $i$-th coordinate of $\Omg$ is
\begin{flalign*}
    \quad \pd_i f = \frac{\pd f}{\pd x_i}, &&
\end{flalign*}
the gradient of $f$ is $\nabla f = \left(\pd_1 f, \ldots  \pd_N f\right)$
and the Laplacian of $f$ is $\Delta f = \sum_{i=1}^N \pd_i^2 f.$
\todo{introduce $C_0$}
\medskip

The integral of $f \in L^1(\Omg)$ is
\begin{flalign*}
    \quad \int_{\Omg} f(\vx) \, \dx \; \equiv \; \int_\Omg f ~ .&&
\end{flalign*}
\todo{introduce $L_{loc}$}
For a vector field $\vct{F} = (F_1, \ldots, F_N) \in C^1(\Omg; \R^N)$,
the divergence is defined as $\Div \vct{F} = \sum_{i=1}^N \pd_i F_i$.

\medskip

\td{Work In Progress (WIP) -- Complete Last. Incoporate following

\begin{flalign*}
    \quad & \phi:\Omg(t)\to\R \;\text{ r.t. a \emph{scalar field}}, & \\
    \quad & \vct f:\Omg(t)\to\R^N \;\text{ r.t. a \emph{vector field}}, &&\\
    \quad & \vct T:\Omg(t)\to\R^{N\times N} \;\text{ r.t. a \emph{(second-order) tensor field}}. &&
\end{flalign*}
The tensor $\vct T$ defines a linear map from $\R^N \to \R^N$, and once a basis is fixed,
$\exists ~ A \in \R^{N\times N}$ s.t. $\vct T(\vct x) = A \,\vct x$ for all $\vct x \in \Omg(t)$.}

        \newpage

    \section{Introduction}
        \label{sec:intro}
        Hemodynamics studies the kinematics of blood. Our interest is the kinematic motion
        of blood within the Human macrocirculatory system, i.e. the flow of blood in large
        vessels such as arteries and veins. Blood is observed as a complex fluid of formed elements
        suspended in plasma, thus, the rheological behavior of blood is non-trivial. We report techniques for modeling
        bloods' motion in a Human's macrocirculatory system.

        \medskip

        After stating our reports' motivation, we begin with physiological review of the of the Human's circulatory system
        in Sec.~\ref{subsec:pysiology}. Sec.~\ref{subsec:continuum} states the continuum hypothesis, a necessary posulate of
        fluid mechanics. By adopting the continuum hypothesis we treat blood as a continuous medium; the resulting hemodynamics problem becomes
        a problem in the motion of continous media. In Sec.~\ref{subsec:blood-model}, we discuss various assumptions one may
        impose on the rheology of bloods material properties. By adopting a particular rheological model, we obtain the necessary constitutive
        relations for a mathematical description of bloods motion. Treating blood as a simple fluid (single-contitute, homogenous, and isotropic
        mixture) yeilds a Newtonian rheological model of blood (i.m.b. justified such models are valid in large vessels, e.g.
        arteries and veins, where the shear rates are sufficiently high [\cite{fung1997biomechanics}, Ch.~2]).
        Assuming the density of a material element of blood remains constant as it flows within the vessel, is
        r.t. the incompressibility condition. Together, these assumptions result in an incompressible-newtonian rheological model of blood.
        Finally, in Sec.~\ref{subsec:navier-stokes} the Navier-Stokes (NS) system of Partial Differential Equations (PDEs) governing
        the motion of an incompressible-newtonian continuum of blood are derived. The NS equations serve
        as the foundation for all subsequent modeling techniques reported herein.

        \medskip

        \paragraph{Motivation}
        Coronary artery stenosis (CAS) is the narrowing of the coronary arteries due
        to the buildup of plaque. Such narrowing can restrict blood flow to the heart muscle,
        which may lead to various cardiovascular problems. Current methods for predicting a stenotic
        blockage (SB) in a coronary artery are rudimentary, and often SB predicition doesn't mean obstruction [\cite{ehab332}].
        Additionally, current clinical methods for assessing the severity of a SB rely on imaging techniques such as angiography,
        intravascular ultrasound (IVUS), and optical coherence tomography (OCT) to visualize the arteries and identify areas of narrowing.
        These methods provide valuable information about the anatomy of the arteries, but they do not
        provide direct information about the functional significance of the CAS. Functional assessment of CAS typically involves measuring the fractional flow reserve (FFR),
        which is the ratio of the blood pressure downstream of the stenosis to the blood pressure upstream of the stenosis during maximum blood flow.
        However, measuring FFR requires the use of a pressure wire, which can be invasive and carries some risks.
        Therefore, there is a need for non-invasive methods to assess the functional significance of CAS.


    % ------------------ APPENDIX
    \section{Appendix}
        \label{sec:appendix}

        \subsection*{Bibliography}
            \printbibliography

        \subsection*{Code Listings}

        \td{Code listings}
        \lstset{language=julia}
        \begin{lstlisting}[language=julia, caption={Algorithm 16.5}, label={lst:problem-sampling}]
        function foo()
            println("Hello World)
        end
        \end{lstlisting}



\end{document}
% ===================== END DOCUMENT ================================
% ===================================================================
