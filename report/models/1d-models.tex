The 1D model reduces the spatial dimension of the NS system by assuming
axial symmetry of the vessel and averaging pressure and velocity
over cross-sectional areas orthogonal to the vessel centerline.
One may start by introducing a rigid-vessel assumption, which leads to a no slip condition
that $\vu \big|_{\pOmgBt} \; = \; \vct{0}$. Instead we seek
to model the link between blood flow and the deformation of the vessel wall.
We begin our derivation of Dimension-Reduced models by assuming the following
transformation exists of our fluid domain boundary $\Omgt$ to a simplified geometry.

\medskip

\begin{figure}[!htb]
    \label{fig:compliant-vessel}
    \centering
    \captionsetup{aboveskip=2pt,belowskip=2pt}
    \includegraphics[width=0.95\linewidth,height=0.45\textheight,keepaspectratio]{compliant-vessel.png}
    \caption{
        From \cite{book1} [Fig. 3.2, pg. 37]
    }
\end{figure}

\newpage

We consider the fluid dynamics of the following fluid element contained in a portion of the lumen $\Omgt$
\begin{figure}[!htb]`'
    \label{fig:fluid-element}
    \centering
    \captionsetup{aboveskip=2pt,belowskip=2pt}
    \includegraphics[width=0.8\linewidth,height=0.25\textheight,keepaspectratio]{fig33.png}
    \caption{
        From \cite{book1} [Fig. 3.3, pg. XX]
    }
\end{figure}
Let $S_1(t), ~ S_2(t)$ be the time-dependent shaded boundaries at $z = z_1$ and $z = z_2$ s.t.
$0 < z_1 < z_2 < \ell$.
Let $V(t)$ be the fluid element of blood. The boundary of the fluid element is
$\pd V(t) \; = \; S_1(t) \cup S_2(t) \cup \pd V(t)_w$ such that $\pd V(t)_w$
is the vessel wall in contact with the fluid element.
According to the Reynold's transport theorem for scalar feild $\phi \in L^1(\Omgt)$
\begin{flalign*}
    \quad \dfrac{d}{dt} \int_{V(t)} \phi(t, \vx) \dx
        & \equals \int_{V(t)} \pd_t \phi(t, \vx) + \inp{\nabla}{\phi(t, \vx)\vub(t, \vx)} \dx &&\\
        & \equals \int_{V(t)} \pd_t \phi(t, \vx) \dx
        ~ + ~ \int_{\pd V(t)_w}  \inp{\vub(t, \vx)}{\nhat} \phi(t, \vx) \dSx, &&
\end{flalign*}
where $\vub$ is the velocity feild on the boundary $\pd V(t)$
(Pf. see \href{https://en.wikipedia.org/wiki/Reynolds_transport_theorem}{wiki}).
If we assume the normal component of $\vub = \vct{0}$ near the inlet
and outlet boundaries $S_1$ and $S_2$ (resp.) of $\Omg$, then the motion of the vessel wall
is coupled to the blood flow through the fluid element $V(t)$. The velocity $\vub$
is equivalent to the velocity of the vessel wall $\partial \Omgt$ in contact with the boundary element $\pd_t$.
I.e., the vessel wall velocity $\vct{u_w} = \vub$.
Now let $\vct{w} = \vct{u_w} - \vu$ be the relative velocity of the vessel wall
w.r.t. the velocity $\vu = (u_1, u_2, u_3)^\top$ of the blood element $V(t)$. Then it follows that
\begin{flalign*}
    \quad \; \int_{\pd_t} \big(\vub \cdot \nhat \big)\phi \; \dSx ~
        & = ~ \; \int_{\pd_t} \big(\vct{u_w} \cdot \nhat \big)\phi \; \dSx &&\\
    \quad \; & = \; \int_{\pd_t} \big(\vct{w} \cdot \nhat \big)\phi \; \dSx
        ~ + ~ \; \int_{\pd_t} \big(\vu \cdot \nhat \big)\phi \; \dSx &&
\end{flalign*}
Let $\bar{\phi}$ denote the average value of $\phi$ defined over a surface $S$
\begin{flalign*}
    \quad \bar{\phi} \defined \frac{1}{A} ~ \int_{S(z, t)} \phi \;\dSx\quad :  \quad  A(z, t) \defined \int_{S(z, t)} \dSx &&
\end{flalign*}
Now we may rewrite the volume integral in the LHS of RT theorem
\begin{flalign*}
    \quad  \int_{V(t)} \phi dV \; = \; \int_{z_1}^{z_2} \int_{S(z, t)} \phi \;\dSx \; dz
        \;=\; \int_{z_1}^{z_2} A \cdot \bar{\phi} \; dz &&
\end{flalign*}
where $z_1 < z_2$ are fixed $z$-coordinates for $S_1$ and $S_2$.
Then we differentiate the integrands in the above equation w.r.t. $t$
\begin{flalign*}
    \quad \int_{V(t)} \dfrac{\partial \phi}{\partial t} dV \;=\; \int_{z_1}^{z_2} \dfrac{\partial}{\partial t} \bigg[A \cdot \bar{\phi} \bigg]\; dz, &&
\end{flalign*}
and we've rewritten the first term in the RHS of the reynolds system.
The surface integral in the RHS may be written as
\begin{flalign*}
    \quad \int_{\pd_t}  (\vub \cdot \nhat) \phi \dSx = \int_{\pd_t}  (\vub \cdot \nhat) \phi \dSx.... &&
\end{flalign*}\todo{cleanup, ref. pg. 43}
With a little more work, one may obtain:
\begin{definition}
    \label{def:1dReynolds}
    The 1D Reynolds Transport theorem for both compressible and incompressible fluids:
    \begin{flalign*}
        \quad \frac{\partial }{\partial t} \bigg( A \bar{\phi} \bigg) \; + \; \frac{\partial}{\partial z}\big( A (\overline{\phi \cdot u_3}) \big)
        \; = \; \int_S \bigg( \dfrac{\partial \phi}{\partial t} + \nabla \cdot (\phi \vu) \bigg) \dSx
        \; + \; \int_{\partial S} \phi \vct{w} \cdot \nhat \; d \gamma &&
    \end{flalign*}
\end{definition}

\begin{remark}
    \label{1DMassConservation}
    By taking $\phi = \rho$ in \ref{def:1dReynolds}, mass conservation follows directly. Also, by our assumption
    that blood is incompressible, we have $\begin{cases*}\Div(\vu) = 0\\\rho =\text{const.}\\
    \end{cases*}$ and we simplify \ref{def:1dReynolds} as
    \begin{flalign*}
        \quad \frac{\partial A }{\partial t} \; + \; \frac{\partial}{\partial z}\big( A (\overline{u_3}) \big)
        \; = \; \int_{\partial S} \vct{w} \cdot \nhat \; d \gamma &&
    \end{flalign*}
    The RHS term above describing the transport process across the vessel wall. \todo{complete, pg. 45}
\end{remark}

\begin{remark}
    \label{1DMomentumConservation}
    By taking $f = u_3$ in \ref{def:1dReynolds}, momentum conservation follows directly. Also,
    by our assumption that blood is incompressible, we simplify \ref{def:1dReynolds} as
    \begin{flalign*}
        \quad \frac{\partial }{\partial t} \bigg( A u_3 \bigg) \; + \; \frac{\partial}{\partial z}\big( A (\overline{u_3^2}) \big)
        \; = \; \int_S \bigg( \dfrac{\partial u_3}{\partial t} + \nabla u_3 \cdot \vu \bigg) \dSx
        \; + \; \int_{\partial S} u_3 \vct{w} \cdot \nhat \; d \gamma &&
    \end{flalign*}
    The RHS term above describing the transport process across the vessel wall.
\end{remark}
